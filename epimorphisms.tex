\documentclass[12pt]{article}
\usepackage[english]{babel}
\usepackage{amssymb,amsthm,amscd,amsmath, txfonts,mathrsfs,enumerate}
\usepackage{xcolor}
\usepackage{tikz-cd}
\usepackage{amsmath}
\usepackage{amsfonts}
\usepackage{amssymb}
\usepackage{latexsym}
\usepackage{eufrak}
\usepackage{makeidx}
%\usepackage{enumitem}




\input xy
\xyoption{all}


\usepackage{vmargin}
\setmarginsrb{28mm}{20mm}{28mm}{25mm}%
          {10mm}{7mm}{10mm}{10mm}

\usepackage[all]{xy}
%\usetikzlibrary{arrows,shapes,positioning,decorations.markings, decorations.pathreplacing,decorations.pathmorphing}

\usepackage{hyperref}
%% hyperref should be the last package loaded

\usepackage[makeroom]{cancel}



\newtheorem{lemma}{Lemma}[section]
\newtheorem{proposition}[lemma]{Proposition}
\newtheorem{theorem}[lemma]{Theorem}
\newtheorem{corollary}[lemma]{Corollary}
\newtheorem{conjecture}[lemma]{Conjecture}
\newtheorem{problem}[lemma]{Problem}
\newtheorem{fact}[lemma]{Fact}
\newtheorem{claim}[lemma]{Claim}
\newtheorem{question}[lemma]{Question}
\newtheorem{fatti}[lemma]{Facts}
\newtheorem*{claim*}{Claim}

\newtheorem{introtheorem}{Theorem}
\renewcommand{\theintrotheorem}{\Alph{introtheorem}} 

\theoremstyle{definition}
\newtheorem{definition}[lemma]{Definition}
\newtheorem{remark}[lemma]{Remark}
\newtheorem{construction}[lemma]{Construction}
\newtheorem{example}[lemma]{Example}
\newtheorem{examples}[lemma]{Examples}
\newtheorem*{notaz}{Notation}


\newcommand{\bvector}[2]{\textstyle{\binom{#1}{#2}}}
\newcommand*{\norm}[1]{\rVert #1 \rVert}
\newcommand*{\ec}[2]{\mathscr{E}_{#1}(#2)}

% NOTES
\def\NB{$\clubsuit$}
\def\NBA{$\blacklozenge$}
\def\NBI{$\spadesuit$}
\def\NBD{$\maltese\ $}

% \mathbb
\def\TT{\mathsf U}
\def\SS{\mathbb S}
\def\N{\mathbb N}
\def\R{\mathbb R}
\def\Z{\mathbb Z}
\def\Q{\mathbb Q}
\def\T{\mathbb T}
\def\K{\mathbb K}
\def\A{\mathcal A}
\def\D{\mathfrak D}
\def\X{\mathfrak X}
\def\Y{\mathfrak Y}
\def\atr{\mathfrak{atr}}
\def\Coim{\mathrm{Coim}}
\def\coeq{\mathrm{coeq}}

% Operators
\def\Ker{\mathrm{Ker}}
\def\P{\mathcal P}
\def\Pf{\P_{fin}}
\def\Aut{\mathrm{Aut}}
\def\Mod{\mathrm{Mod}}
\def\Se{\mathfrak S}
\def\G{\mathcal{G}}

\def\hti{\widetilde{h}}

%\newcommand{\LM}{\mathcal L}
\newcommand{\F}{\mathcal F}
%\newcommand*{\Mon}[1]{\mathcal M_{#1}}
%\newcommand{\LCG}{\textbf{\textup{LC}}}
%\newcommand{\LC}{{}_{\K}\LCG}
%\newcommand{\LLC}{{}_{\K}\textbf{\textup{LLC}}}
%\newcommand{\LCA}{\textbf{\textup{LCA}}}
%\newcommand{\Set}{\textbf{\textup{Set}}}
%\newcommand{\CC}{\LCA_{cc}}
%\newcommand{\Vect}{{}_{\K}{\mathbf Vect}}
%\newcommand{\Flow}{\mathrm{Flow}}
\newcommand{\bcL}{\mathbf{\mathcal{L}}}
%\newcommand{\Lqm}{\bcL_{qm}}
\newcommand{\barLqm}{\overline{\bcL}_{qm}}
\newcommand{\Lpqm}{\mathcal{L}_{pqm}}
%\newcommand{\Ab}{\textbf{\textup{Ab}}}
%\newcommand{\TDLC}{\textbf{\textup{TDLC}}}
\newcommand{\Spec}{\mathrm{Spec}}
\newcommand{\re}{\mathrm{reg}}
% Functors
\newcommand{\LCO}{\mathcal{LCO}}
\newcommand{\CO}{\mathcal{CO}}
\newcommand{\clop}{closure operator}

\def\Triple{\mathbb T}

\def\Inv{\mathrm{Inv}}

%\DeclareMathOperator{\td}{t.d.}
%\DeclareMathOperator{\dist}{dist}
%\DeclareMathOperator{\cardm}{card}
\def\Im{\mathrm{Im}}
%\DeclareMathOperator{\supp}{Supp}

\newcommand*{\card}[1]{\left\vert #1 \right\vert}
%\newcommand*{\abs}[1]{\left\lvert #1 \right\rvert}
%\newcommand*{\Abs}[1]{\bigl\lvert #1 \bigr\rvert}
%\newcommand*{\set}[1]{\left\{ #1 \right\}}
%\newcommand*{\tuple}[1]{\langle #1 \rangle}

% Entropies
\newcommand{\halg}{h_{alg}}
\newcommand{\hset}{\texorpdfstring{\ensuremath{h_{set}}}{h-set}}
\newcommand{\Halg}{H_{alg}}
\def\ent{\mathrm{ent}}
\def\M{\mathcal{M}}
\newcommand\eps{\varepsilon}
% Categories
\newcommand{\LM}{\mathcal L}
\newcommand{\LCG}{\textbf{\textup{LC}}}
\newcommand{\LC}{{}_{\K}\LCG}
\newcommand{\LLC}{{}_{\K}\textbf{\textup{LLC}}}
\newcommand{\LCA}{\textbf{\textup{LCA}}}
\newcommand{\Set}{\textbf{\textup{Set}}}
\newcommand{\CC}{\LCA_{cc}}
\newcommand{\Vect}{{\K}\text{-}\textbf{\textup{Vect}}}
\newcommand{\Flow}{\mathrm{Flow}}
\newcommand{\Lqm}{\mathcal{L}_{qm}}
\newcommand{\LL}{\overline{\mathcal{L}}_{qm}}
\newcommand{\Ab}{\textbf{\textup{Ab}}}
\newcommand{\Grp}{\textbf{\textup{Grp}}}
\newcommand{\Ring}{\textbf{\textup{Ring}}}
\newcommand{\Modu}{R\text{-}\textbf{\textup{Mod}}}
\newcommand{\Mon}{\textbf{\textup{Mon}}}
\newcommand{\TFA}{\textbf{\textup{TFAb}}}
\newcommand{\Cat}{\textbf{\textup{Cat}}}
\renewcommand{\Top}{\textbf{\textup{Top}}}
\newcommand{\Haus}{\textbf{\textup{Haus}}}
\newcommand{\FHaus}{\textbf{\textup{FHaus}}}
\newcommand{\Cmon}{\textbf{\textup{Cmon}}}
\newcommand{\Semi}{\textbf{\textup{Semi}}}
\newcommand{\Csemi}{\textbf{\textup{Csemi}}}
\newcommand{\TDLC}{\textbf{\textup{TDLC}}}
\newcommand{\SL}{\mathcal{SL}}
\newcommand{\VSL}{\mathcal{VSL}}
\newcommand{\TAb}{\mathbf{TAb}}
\newcommand{\QQ}{\mathcal Q}
\newcommand{\SLatt}{\mathbf{SLatt_0}}
\newcommand{\SLattnobottom}{\mathbf{SLatt}}
\newcommand{\eim}[1]{\mathbf{EM}{(#1)}}
\newcommand{\CHaus}{\mathbf{CHaus}}
\newcommand{\Bool}{\mathbf{Bool}}
\newcommand{\CSLatt}{\mathbf{CSLatt}}
\newcommand{\MM}{\mathfrak{M}}
\newcommand{\BL}{\mathbf{BLatt}}
% Functors
%\newcommand{\LCO}{\mathcal{LCO}}
%\newcommand{\CO}{\mathcal{CO}}

%Miscellanea
\newcommand{\argu}{\hbox to 7truept{\hrulefill}}

\DeclareMathOperator{\Ufor}{U}
\DeclareMathOperator{\Cfun}{C}
\DeclareMathOperator{\Dfun}{D}
\DeclareMathOperator{\Ifun}{I}
\DeclareMathOperator{\Ffun}{F}
\DeclareMathOperator{\eq}{eq}
\DeclareMathOperator{\Gfun}{G}
\DeclareMathOperator{\Mor}{Mor}
\DeclareMathOperator{\wid}{width}
\DeclareMathOperator{\Conv}{Conv}
\DeclareMathOperator{\Dclop}{D}
\DeclareMathOperator{\Tclop}{T}
\DeclareMathOperator{\Iclop}{I}
\DeclareMathOperator{\Cclop}{C}
%\newcommand{\Spdf}{\texorpdfstring{$S$}{S}}
%\newcommand{\Gpdf}{\texorpdfstring{$G$\xspace}{G\xspace}}
%% Pdf-LaTeX lancia un warning se trova una formula dentro il titolo di una
%% sezione. La soluzione per evitarlo e' usare il comando \texorpdfstring
%% usage: instead of writing \section{$S$-sfift}, write \section{\Spdf-shift}
%%%%%%%%%%%%%%%%%%%%%%%%%%%%%%%%%%%%%
\def\rel{\mathfrak{R}}
%% <x,y> tuple
\newcommand*{\Pa}[1]{\bigl(#1\bigr)}
\newcommand*{\PA}[1]{\left(#1\right)}
% \newcommand{\PF}[1]{[#1]^{<\omega}}
\newcommand{\et}{\ \& \ }
\newcommand{\SDiff}{\mathbin{\Delta}}

\newcommand{\rest}{\mathbin\restriction}

\newcommand*{\ns}[1]{\tensor[^*]{#1}{}}
% {{}^*{#1}}
\newcommand{\nsA}{\ns A}
%{{}^*\!A}
\newcommand{\nsV}{\ns{\mathbb V}}

%\newcommand{\av}{{\bar a}}
%\newcommand{\cv}{{\bar c}}
%\newcommand{\kv}{{\bar k}}
%\newcommand{\nv}{{\bar n}}
%\newcommand{\Aver}{\mathcal H}
%\newcommand{\Afam}{{\mathcal A}}
%\newcommand{\FolSeq}{\mathfrak F}
%\DeclareMathOperator{\td}{t.d.}
%\DeclareMathOperator{\dist}{dist}
%\DeclareMathOperator{\cardm}{card}
%\newcommand{\eqeps}{\mathrel{=_{\eps}}}
%\newcommand{\leqeps}{\mathrel{\leq_{\eps}}}
%\newcommand{\geqeps}{\mathrel{\geq_{\eps}}}
%\newcommand{\eps}{\varepsilon}
%\newcommand{\beps}{{\bar\varepsilon}}
%\newcommand{\nb}{{\bar n}}
%
%\newcommand{\ellinf}{\ell_{\infty}}
%\newcommand{\ellone}{\ell_{1}}
%\newcommand*{\pair}[1]{\langle #1 \rangle}


\renewcommand{\theequation}{\thesection.\arabic{equation}}
\numberwithin{equation}{section}

\newlength{\bibitemsep}\setlength{\bibitemsep}{.2\baselineskip plus .05\baselineskip minus .05\baselineskip}
\newlength{\bibparskip}\setlength{\bibparskip}{0pt}
\let\oldthebibliography\thebibliography
\renewcommand\thebibliography[1]{%
  \oldthebibliography{#1}%
  \setlength{\parskip}{\bibitemsep}%
  \setlength{\itemsep}{\bibparskip}%
}


\def\gen{generalized\ }
\def\nbd{neighbourhood\ }
%\def\Gen{Generalized\ }
\def\qm{quasi-metric\ }
\def\qms{quasi-metrics\ }


\DeclareFontFamily{OT1}{pzc}{}
\DeclareFontShape{OT1}{pzc}{m}{it}{<-> s * [1.200] pzcmi7t}{}
\DeclareMathAlphabet{\mathcal}{OT1}{pzc}{m}{it}
\newcommand{\vin}{\rotatebox[origin=c]{-90}{$\dashv$}}
\newcommand{\catname}[1]{\mathbf{#1}}
\newcommand{\homodd}[0]{\catname{HoMod}}
\newcommand{\until}[0]{\mathsf{until}}
\newcommand{\slice}[2]{(\catname{#1}\downarrow{#2})}
\newcommand{\reach}[1]{\mathsf{reach}_{#1}}
\newcommand{\abs}[1]{\left\lvert#1\right\rvert}
\newcommand{\nat}{\xrightarrow{\bullet}}
\newcommand{\im}[1]{\mathrm{Im}(#1)}
\newcommand{\chain}[1]{\catname{Ch}(\catname{#1})}
\newcommand{\homo}[1]{\mathcal{K}(\catname{#1})}
\newcommand{\colim}[0]{\mathrm{colim}}
\newcommand{\rela}[2]{\mathcal{Rel}_\mathcal{#2}(\catname{#1})}
\newcommand{\sub}[1]{\mathsf{Sub}_{\catname{#1}}}
\newcommand{\subm}[1]{\mathcal{M}/{{#1}}}
\newcommand{\set}[1]{\mathsf{Inj}/{\abs{#1}}}
\newcommand{\dom}{\mathrm{dom}}
\newcommand{\cod}{\mathrm{cod}}
\newcommand{\co}[1]{\{\abs{#1}\}}
\newcommand{\equ}[2]{[#1 \equiv #2]}
\newcommand{\id}[1]{\mathsf{id}_{#1}}
\newcommand{\sat}[0]{\mathsf{cl}}
\newcommand{\escape}[3]{#2 \mathfrak{E}_{#1} #3}
\newcommand{\es}[3]{ \mathsf{isER}_{#1}(p,#2, #3)}
\newcommand{\esca}[4]{ \mathsf{eRoute}_{#1} (#4, #2, #3)}
\newcommand{\cov}{\vartriangleleft}
\newcommand{\pos}[1]{\mathsf{Pos}_{\mathfrak{#1}}}
\newcommand{\rk}[2]{\mathrm{rk}_{#1}({#2})}
\newcommand{\lind}[1]{\mathcal{L}({#1})}
\newcommand{\class}[1]{\catname{Cl}({#1})}
\newcommand{\hoclass}[1]{\catname{HoCl}({#1})}
\newcommand{\propo}[0]{\mathsf{Prop}}
\newcommand{\term}[2]{\mathbf{Term}({#1}, {#2})}
\newcommand{\formu}[1]{\mathbf{Form}_{#1}}
\newcommand{\ex}[1]{\mathcal{ex(#1)}}
\newcommand{\inte}[1]{\llbracket{#1}\rrbracket}
\newcommand{\typ}[1]{\mathsf{Type}({#1})}
\newcommand{\coalg}[1]{\catname{CoAlg}(\mathcal{#1})}
\newcommand{\modd}[0]{\catname{Mod}}
\newcommand{\des}[1]{Des({#1})}
\newcommand{\prof}[2]{\mathsf{Proof}({#1},{#2})}
\newcommand{\sur}[3]{{#2}\mathfrak{S}_{#1}{#3}}
\newcommand{\er}[3]{\mathsf{EscR}_{#1}({#2},{#3})}
\newcommand{\ab}[0]{=-Adj}
\newcommand{\ba}[0]{=-E}
\newcommand{\su}[2]{\mathfrak{s}_{#1}^{#2}}
\newcommand{\ipos}[1]{\catname{Pos}(\catname{#1})}
\newcommand{\Su}[2]{\mathfrak{S}_{#1}^{#2}}
\newcommand{\Pro}[2]{\mathfrak{P}_{#1}^{#2}}
\newcommand{\surr}[0]{\mathsf{surr}}
\newcommand{\propag}[0]{\mathsf{propag}}
\newcommand{\nxt}[0]{\mathfrak{s}}
\newcommand{\integ}[1]{\mathsf{int}_{#1}}
\newcommand{\scat}[1]{\mathsf{nec}_{\gamma_{#1}}}
\newcommand{\diam}[1]{\mathsf{cl}_{\gamma_{#1}}}
\newcommand{\fix}[2]{\mathsf{Fix}_{(#1, \cdot)}({#2})}
\newcommand{\stab}[2]{\mathsf{Stab}_{(#1, \cdot)}({#2})}
\newcommand{\stabi}[2]{\mathsf{Stab}_{(#1, \cdot_{#1})}({#2})}
\newcommand{\inva}[1]{\mathsf{Inv}_{(#1, \cdot)}}
\newcommand{\invar}[1]{\mathsf{Inv}_{(#1, \cdot_{#1})}}
\newcommand{\fuz}[0]{\mathcal{FzSub}}
\newcommand{\cont}[0]{\mathsf{cont}}
\newcommand{\bound}[0]{\mathsf{ext}}


\def\E{\mathcal{E}}
\def\epi{\mathsf{epi}}
\def\reg{\mathsf{reg}}
\newcommand{\rg}[2]{\mathsf{reg}_{#1}([#2])}
\newcommand{\ep}[2]{\mathsf{repi}_{#1}([#2])}





\title{A categorical approach to the epimorphism problem}

\author{}
\begin{document}

\maketitle


\setcounter{tocdepth}{3}
\tableofcontents


\section*{Introduction}\label{Background}
\addcontentsline{toc}{section}{Introduction}

In most of the categories one usually encounters in algebra (e.g., categories of semigroups, monoids, posets, semi-lattices, lattices, groups, Abelian groups, modules, rings, vector spaces, etc. but also topological spaces, topological groups, etc.), there is an obvious way to interpret what a ``surjective morphism'' or an ``injective morphism'' is. The reason for this is that, for any morphism in such commonly used categories, there is an underlying ``set-theoretic'' function. Now, it is well-known that, in the category of sets, the monomorphisms are exactly the injective functions, while the epimorphisms are exactly the surjective functions. Hence, the following question seems natural:

\bigskip\noindent
{\bf Question. }{\em
Let $\X$ be a category where the expressions ``surjective morphism'' and ``injective morphism'' have a clear meaning. When is it true that surjective morphisms coincide with epimorphisms and/or injective morphisms coincide with monomorphisms?}

\medskip
In this section we introduce some hypotheses on a category $\X$ that allow one to make the above question precise. We will see, in particular, that surjective morphisms are always epimorphisms and injective morphisms are always monomorphisms (see Lemma \ref{faithful}) and, in any category ``of algebraic origin'', monomorphisms and injective morphisms do always coincide by abstract non-sense (see Corollary \ref{mono_is_injective} and the following discussion). Hence, the right question to ask in those categories is whether or not all epimorphisms are surjective: this is sometimes called {\em the epimorphism problem}. \marginpar{Ho riscritto la sezione 1 incorporando il materiale della sezione 3 sulle categorie concrete a cui è dedicata la sottosezione 1.2. }
 

\section{Some categorical constructions}

 In this subsection we introduce the categorical constructions --like co/products, co/equalizers and co/kernel pairs-- that will be necessary in the rest of the section. The most important result of this subsection is Proposition \ref{mono=trivial_ker_prop}, that characterizes monomorphisms and epimorphisms in terms of the vanishing of their kernel and cokernel pair, respectively. 

\subsection{Monomorphisms and epimorphisms}
\reversemarginpar \marginpar{Visto che stiamo richiamando alcune nozioni base e fissando la notazione tanto vale richiamare anche queste due} \normalmarginpar
In this section we will recall the definition of monomorphisms and epimorphisms and some useful facts about them. 

\begin{definition}
	An arrow $m:X_1\rightarrow X_2$ in a category $\X$ is said to be a \emph{monomorphism} (or a \emph{mono}, or \emph{monic}) if, for every $f,g:Y\rightarrow X_1$, $f=g$ whenever $m\circ f=m\circ g$. Dually $m$ is an \emph{epimorphism} (or a \emph{epi}, or \emph{epic}) if, for every $f,g:X_2\rightarrow Y$, $f=g$ whenever $f\circ m=g\circ m$.
 \end{definition}

\begin{remark}
	Equivalently, we can say that $m$ is monic if and only each component of the natural transformation $m_*:\X(-, X_1)\rightarrow \X(-, X_2)$ is injective. Dually the injectivity of each component of  $m^*:\X(X_2, -) \rightarrow \X(X_1, -)$  is equivalent to $m$ being epic.
\end{remark}

\begin{example}\label{sets}
	In $\Set$ monics coincide with injective functions and epic with surjective ones.
\end{example} 

Notice that while every isomorphism is both monic and epic, the viceversa is not true.

\begin{example}\label{pos}
	Let $(P, \leq)$ be a poset regarded as a category, the set of arrows between any two elements has at most cardinality $1$, therefore every arrow is both monic and epic.
\end{example}
Monics and epis enjoy some closure properties with respect to composition.

\begin{proposition}\label{composition}Let $\X$ be a category, then:
	\begin{enumerate}[\rm (1)]
		\item if $g$ and $f$ are monomorphisms then $g\circ f$ is monic;
		\item if $g\circ f$ is monic then $f$ is monic, dually if $g\circ f$ is epic than $g$ is epic. 
	\end{enumerate}
The dual propositions hold too.
	\end{proposition}
\begin{proof}
	\begin{enumerate} [\rm (1)]
		\item If $(g\circ f)\circ h=(g\circ f)\circ k$ then, since $g$ is monic, $f\circ h=f\circ k$ and now we can conclude using the fact that $f$ is monic too.
		\item Suppose that $f\circ k=f\circ h$ for some pair of arrows $h$ and $k$, then 
		\[(g\circ f)\circ k= g\circ (f\circ k)=g\circ (f\circ h)=(g\circ f)\circ k\]
		and we can conclude since $g\circ f$ is monic. \qedhere 
	\end{enumerate}
\end{proof}


\begin{lemma}\label{faithful}Forgetful functors reflects monomorphisms and epimorphisms: if  $F\colon \X \to \Y$ is faithful and  $F(\phi)$ is monic in $\Y$, then $\phi$ is monic in $\X$, and similarly for epis.
\end{lemma}
\begin{proof}
Let $f$ and $g$ be arrows such that $\phi\circ f=\phi\circ g$, then
\[
F(\phi)\circ F(f)=F(\phi \circ f)=F(\phi \circ g)=F(\phi)\circ F(g)
\]
By hypothesis $F(\phi)$ is monic, thus $F(f)=F(g)$, hence $f=g$ by faithfulness.
The thesis for epis follows by duality noting that $F^{op}:\X^{op}\rightarrow \Y^{op}$ is faithful if $F$ is.
\end{proof}


\subsection{Concrete categories}
In this work we are interested in categories of sets endowed with some extra structure, the corresponding formal notion corresponding is the one of \emph{concrete categories}.
\begin{definition}
	A {\em concrete category} is a pair $(\X,|-|)$, where $|-|\colon \X\to \Set$ is a faithful functor, usually called the \emph{forgetful} (or the \emph{underlying set}) functor. 
	
	In a concrete category $(\X,|-|)$, a morphism $\phi\colon X\to Y$ in $\X$ is said to be:
	\begin{itemize}
		\item {\em injective} if the underlying function $|\phi|\colon |X|\to |Y|$ is injective;
		\item {\em surjective} if the underlying function $|\phi|\colon |X|\to |Y|$ is surjective.
	\end{itemize}
\end{definition}

By Lemma \ref{faithful} and Example \ref{sets} we can deduce at once that injective arrows are monomorphisms and surjective arrows are epimorphisms. The converse is in general not true as shown by the following example.

\begin{example}\label{rings}Consider the inclusion $i:\Z\rightarrow \Q$ in the category $\Ring$ of rings. $i$ is a monic since it is injective. On the other hand for every unital ring $R$ there exists at most one arrow $\mathbb{Q}\rightarrow R$ and thus $i$ is also epic.
\end{example}

We end this section with a definition.
\begin{definition}We say that two concrete categories $(\X, |-|_\X)$ and $(\Y, |-|_Y)$ are \emph{concretely equivalent (isomorphic)} if there is an equivalence (isomorphism) $F:\X\rightarrow \Y$ such that the following diagram commute.
\[
\xymatrix@C=20pt{
\X\ar[rr]^{F}\ar[dr]_{|-|_\X}&&\Y\ar[dl]^{|-|_\Y}\\&\Set}
\]


\end{definition}
 
\subsection{Review of limits and colimits} \marginpar{Mi serve la definizione generale di limite e colimite per trattare meglio la (co)completezza nelle categorie di algebre}
 In this section we recall the definition of limits and colimits and examine in details some kinds of them which will be useful in the next sections.
 
 \begin{definition}
 Let $F:\D\rightarrow \X$ be a functor, a \emph{cone} for $F$ is a couple $(D, \{f_D\}_{D\in \D})$ where $X\in \X$, $f_D: X\rightarrow F(D)$ and such that $f_{D_1}\circ F(\phi)=f_{D_2} $ for every $\phi:D_1\rightarrow D_2 $ in $\D$. A \emph{limit} for $F$ is a cone $(L, \{l_D\}_{D\in \D})$ which is universal: for every other cone $(X, \{f_D\}_{D\in \D})$ there exists a unique $\phi:X\rightarrow L$ such that $l_D\circ \phi=f_D$ for every object $D$ of $\D$. 
 
 Dually a a \emph{cocone} for $F$ is a couple $(D, \{f_D\}_{D\in \D})$ with $f_D: F(D)\rightarrow X$ and such that $f_{D_2}\circ F(\phi)=f_{D_1} $ for every $\phi:D_1\rightarrow D_2 $ in $\D$. A \emph{colimit} for $F$ is a cocone $(L, \{l_D\}_{D\in \D})$ which is couniversal: for every other cocone $(X, \{f_D\}_{D\in \D})$ there exists a unique $\phi:L\rightarrow X$ such that $\phi\circ l_D=f_D$ for every object $D$ of $\D$.
 \end{definition}


\begin{remark}
	Limits are unique up to isomorphisms. Let $(L, \{l_D\}_{D\in \X})$ and $(L', \{l_D\}_{D\in \X})$ be two limits for the same functor $F:\D\rightarrow \X$. By their universal property there exists $\phi:L\rightarrow L'$ and $\psi:L'\rightarrow L$ such that
	$l'_D\circ \phi= l_D$ and $l_D\circ \psi =l'_D$, thus
	\begin{align*}
	l'_D\circ (\phi\circ \psi)&=(l'_D\circ \phi)\circ \psi=l_D\circ \psi=l'_D\\
	l_D\circ (\psi \circ \phi)&=(l_D\circ \psi)\circ \phi=l'_D\circ \phi=l_D
	\end{align*}
	and now the uniqueness clause in the universal property entails $\phi\circ \psi=\id{L'}$ and $\psi \circ \phi=\id{L}$. The same holds by duality for colimits. 
\end{remark}
 
 We will often identify a (co)limit $(L, \{l_D\}_{D\in \D})$ with its vertex $L$ whenever the structural arrows $l_D$ are clear from the context. We will also use $\lim(F)$ (or $\colim(F)$) to denote such vertex.
 \begin{example}Let $\D$ be the empty category and $F:\D\rightarrow \X$ the empty functor. Then a limit for it is just an object $1$ such that there exists a unique arrow $X\rightarrow 1$ for any object $X$, such an object is called \emph{terminal}. Dually, the colimit of the empty functor is an object $0$ such that there exists a unique arrow $0\rightarrow X$ into any $X$, such an object is called \emph{initial}.
 \end{example}
 
 \begin{example}\label{intersection}
 	Let $\{A_i\}_{i\in I}$ be a family of subsets of a set $S$ and take the category with $I\cup \{\infty\}$ as objects and arrows given only by the identities and a unique arrow $f_i:i\rightarrow \infty$ for each $i$, then we have a functor sending $i$ to $A_i$, $\infty$ to $S$ and $f_i$ to the inclusion $A_i\rightarrow S$, which has the intersection $\bigcap_{in\in I}A_i$ as a  limit (see also Section \ref{subobjects} for more details).
 \end{example}
 
 In many situations the functor $F$ of which we are computing the (co)limit will be clear from the context. In such situations we will speak directly of (co)limits of a family of objects and arrows: for instance the example above can be rephrased directly saying that the family of inclusions $A_i\rightarrow S$ has their intersection as limit.
 
 \begin{remark}It is possible to show that any functor $F:\D\rightarrow \Set$ in which $\D$ is \emph{small} (i.e. has a set of arrows insted of a proper class) admits a limit and a colimit (see for instance \cite{cats,mac}, see also \cite[Proposition 2.7.1]{borceux1} for the smallness requirement).
 \end{remark}
 
 \begin{remark}Take two functors $F:\X\rightarrow \X'$ and $G:\X'\rightarrow \Y$ and suppose that there exists limiting cones $(\lim(F), \{l_X\}_{X\in \X})$, $(\lim(G\circ F),\{l'_X\}_{X\in \X})$ for  $F$ and $G\circ F$ respectively. Then $(G(\lim(F)), \{G(l_X)\}_{X\in \X} )$ is a cone for $G\circ F$ and thus exists a canonical \emph{comparison arrow} $\phi:G(\lim(F))\rightarrow \lim (G\circ F)$. Dually, if the colimits involved exist, there exists a comparison arrow $\psi: \colim(G\circ F)\rightarrow \colim F$.
 \end{remark}
 
 \begin{definition}
 Let $F:\D\rightarrow \X$ and $G:\X\rightarrow \Y$ be functors, $G$ \emph{preserves limits along $F$} if, whenever $F$ has a limit, the comparison arrow $\phi:G(\lim(F))\rightarrow \lim (G\circ F)$ is an isomorphism. $G$ \emph{preserves limits of shape $\D$} if preserves limits along every $F:\D\rightarrow \X$, finally $G$ \emph{preserves limits} (or it is \emph{left exact}) if it preserves limits of every shape $\D$.
 
 Dually,  $G$ \emph{preserves colimits along $F$} if, whenever $F$ has a colimit, the comparison arrow $\psi:\colim(G\circ F)\rightarrow G(\colim (F))$ is an isomorphism. $G$ \emph{preserves colimits of shape $\D$} if preserves colimits along every $F:\D\rightarrow \X$ and $G$ \emph{preserves colimits} (or it is \emph{right exact}) if it preserves limits of every shape $\D$.
 \end{definition}
 
 \begin{lemma}\label{adj} Let $L:\Y\rightarrow \X$ and $R:\X\rightarrow \Y$ be functors such that $L\dashv R$, then $R$ preserves limits and $L$ preserves colimits.
 \end{lemma}
\begin{proof}It's enough to show the first half of the lemma, the other one follows by duality. Let $F:\D\rightarrow \X$ be a functor with limit $(\lim(F), \{l_{D}\}_{D\in \D})$ and take a cone $(C, \{f_D\}_{D\in D})$ over $R\circ D$. By the adjunction $L\dashv R$ we have arrows $\bar{f}_D:L(C)\rightarrow F(D)$ which form a cone with vertex $L$: applying naturality of the adjunction, for every $\varphi:D_1\rightarrow D_2$ in $D$ we have that
$F(\varphi)\circ \bar{f}_{D_1}$ has $R(F(\varphi))\circ f_{D_1}=f_{D_2}$ as mate and so $F(\varphi)\circ \bar{f}_{D_1}=f_{D_2}$. Therefore we get an arrow $\bar{\phi}:L(C)\rightarrow \lim(F)$ which corresponds, under the adjunction, to $\phi:C\rightarrow R(\lim(F))$. Now, again by naturality of the adjunction, the mate of $R(l_D)\circ \phi$ is $l_D\circ \bar{\phi}=\bar{f}_D$ and we can conclude that $R(l_D)\circ \phi=f_D$ for every object $D$ of $\D$.

We are left with  uniqueness: if $\phi$ and $\psi$ are such that $R(l_D)\circ \phi=f_D=R(l_D)\circ \psi$ then $\bar{f}_D$ is equal to both $l_D\circ \phi$ and $l_D\circ \psi$, which entails $\phi=\psi$.
\end{proof}
 
 \begin{remark}\label{absolute}
 	There exist (co)limits preserved by all functors, such colimits are called \emph{absolute}. Now, let $\X$ be a category with a terminal object $1$ and notice that the colimit of every $F:\X\rightarrow \X'$ is given by $F(1)$ with  the images of the unique arrows $X\rightarrow 1$ as colimiting maps, so this colimit is absolute. Dually the limit of a functor $F$ whose domain has an initial object  is absolute. 
 \end{remark}
 
\subsubsection{Finite co/products}
We start with (co)products:  they are (co)limits on the discrete category with two objects. Explicitly we have the following definition.   
\begin{definition}
Given two objects $X_1$ and $X_2$ in a category $\X$, their {\em product}  is a pair $(X_1\times X_2,(\pi_i\colon X_1\times X_2\to X_i)_{i=1,2})$ that satisfies the following universal property: for each pair of maps $(\phi_i\colon Y\to X_i)_{i=1,2}$, there exists a unique map
\[
\Delta_{(\phi_1,\phi_2)}\colon Y\to X_1\times X_2
\]
that we call the {\em diagonal of $\phi_1$ and $\phi_2$}, such that $\pi_i\circ\Delta_{(\phi_1,\phi_2)}=\phi_i$, for $i=1,2$.

Dually,  the {\em coproduct} of $X_1$ and $X_2$ is a pair $(X_1\sqcup X_2,(\iota_i\colon X_i\to X_1\sqcup X_2)_{i=1,2})$ that satisfies the following universal property: for each pair of maps $(\psi_i\colon  X_i\to Y)_{i=1,2}$, there exists a unique map 
\[
\nabla_{(\psi_1,\psi_2)}\colon  X_1\sqcup  X_2\to Y
\]
that we call the {\em codiagonal of $\psi_1$ and $\psi_2$},such that $\nabla_{(\psi_1,\psi_2)}\circ \iota_i=\psi_i$, for $i=1,2$.

For each object $X$ in $\X$, the diagonal (resp., codiagonal) of two identities $\id X$ is denoted by $\Delta_X\colon X\to X\times X$ (resp., $\nabla_X\colon X\sqcup X\to X$).
\end{definition}

\begin{example}
The usual product of groups, rings, topological spaces, $R$-modules are examples of products in the corresponding categories $\Grp$, $\Ring, \Top, \Modu$.
\end{example}

\begin{example}\marginpar{Visto che è una survey ho inserito vari esempi elementari qua e là}In $\Modu$ the product $X_1\times X_2$ of two object is also their coproduct, with $i_1:X_k\rightarrow X_1\times X_2$ and $i_2:X_2\rightarrow X_1\times X_2$ given by
	\begin{equation*}
		i_1(x)=(x,0)\qquad i_2(y)=(0,y)
	\end{equation*} 
 \end{example}
And $\nabla_{(\psi_1,\psi_2)}:X_1\times X_2\rightarrow Y$ is simply
\begin{equation*}
\nabla_{(\psi_1,\psi_2)}(x,y)=\psi_1(x)+\psi_2(y)
\end{equation*} 

\subsubsection{Equalizers and coequalizers}
Consider the category $\X$ generated by the commutative diagram
\[\xymatrix@C=30pt{
A \ar@<-.5ex>[r]\ar@<.5ex>[r] & B
}
\]
Taking (co)limits on this category we get \emph{(co)equalizers}.
\begin{definition}
Let $f,\, g\colon X\rightrightarrows Y$ be a pair of parallel morphisms in a given category $\X$. The {\em equalizer} (resp., {\em coequalizer}) is a (co)universal arrow $e\colon E\to X$ such that $f\circ e=g\circ e$ (resp., $c\colon Y\to C$ such that $c\circ f=c\circ g$):
\[
\xymatrix{E\ar[r]^e&X\ar@<-.5ex>[r]_g \ar@<.5ex>[r]^f&Y}\quad \left(\text{resp.}, \xymatrix{X\ar@<-.5ex>[r]_g \ar@<.5ex>[r]^f&Y\ar[r]^c&C}\right).
\]
In other words, $e\colon E\to X$ is the equalizer of $f$ and $g$ if, and only if, for each morphism $d\colon D\to X$ such that $f\circ d=g\circ d$, there exists a unique morphism $d'\colon D\to E$, such that $e\circ d'=d$. The universal property for the coequalizer is dual.
\end{definition}

\begin{example}\label{ex_eq_of_projections}
	Let $\X$ be a category, $X$ an object in $\X$, and $(X\times X, (\pi_i\colon X\times X\to X)_{i=1,2})$ a product of two copies of $X$. Then, the diagonal map $\Delta_X\colon X\times X$ is the equalizer of $\pi_1$ and $\pi_2$:
	\[
	\xymatrix@C=40pt{
		X\ar[r]^-{\Delta_X}&X\times X\ar@<-.5ex>[r]_-{\pi_1} \ar@<.5ex>[r]^-{\pi_2}&X}.
	\]
	Dually, codiagonal maps are coequalizers of the canonical maps to the coproduct. 
\end{example}
 
\begin{proposition}\label{reg_mono} Let $\X$ be a category and $e\colon E\to X$ is the equalizer of $f$ and $g$ then $e$ is a monomorphism. Dually if $c$ is the coequalizer of two arrows then $c$ is an epimorphism.
\end{proposition}
\begin{proof}
	Let $h,k:A\rightarrow E$ be two arrows and suppose that $e\circ h=e\circ k$. Since $f\circ e=g\circ e$ we have that $e\circ h$ equalizes $f$ and $g$, hence there exists a unique arrow $m:A\rightarrow E$ such that $e\circ h=e\circ m$, but both $h$ and $k$ enjoy this property, thus , by uniqueness,
$h=m=k$.
 \end{proof}

%It is easy to prove that every equalizer is a monomorphism and every coequalizer is an epimorphism, although these implications are not reversible in general. 

\begin{proposition}\label{equalizer_of_equals}
Let $\X$ be a category, $f,\, g\colon X\rightrightarrows Y$ a pair of parallel morphisms and suppose that $e\colon E\to X$ is the equalizer of $f$ and $g$. Then, $f=g$ if and only if $e$ is an isomorphism. 
\end{proposition} 
\begin{proof}
 ($\Leftarrow$) If $e$ is an isomorphism, then  $f=(f\circ e)\circ e^{-1}=(g\circ e)\circ e^{-1}=g$.
	
	\noindent($\Rightarrow$) If $f=g$, then $f\circ \id X=g\circ \id X$ and so, by definition of equalizer, there exists a unique morphism $e'\colon X\to E$ such that $e\circ e'=\id X$, in particular $e\circ e'\circ e=e$, and the previous proposition entails that  $e'\circ e=\id E$.
\end{proof}

Before proceeding further, let us register the following elementary property of co/equalizers that is often useful in computations:

\begin{lemma}\label{easy_lemma_monos_and_eq}
Take a category $\X$ and  arrows $f,\, g\colon X\rightrightarrows Y$  admitting a (co)equalizer in $\X$:
\[
\xymatrix{E\ar[r]^e&X\ar@<-.5ex>[r]_g \ar@<.5ex>[r]^f&Y}\quad \left(\text{resp.,} \xymatrix{X\ar@<-.5ex>[r]_g \ar@<.5ex>[r]^f&Y\ar[r]^c&C}\right).\]
Suppose now that $\phi\colon Y\to Z$ is a monomorphism, then $e\colon E\to X$ is also an equalizer for the pair of maps $\phi\circ f$, $\phi\circ g\colon X\rightrightarrows Z$ (similarly, if $\psi\colon W\to X$ is an epimorphism, then $c\colon Y\to C$ is a coequalizer for the corresponding compositions).
\end{lemma}
\begin{proof}
Let us verify that the map $e\colon E\to X$ satisfies the universal property that makes it the equalizer of the pair $\phi\circ f$, $\phi\circ g\colon X\rightrightarrows Z$. Indeed, suppose that $e'\colon E'\to X$ is a morphism such that $\phi\circ f\circ e'=\phi\circ g\circ e'$. Since $\phi$ is monic, this implies that $f\circ e'=g\circ e'$ and so, since $e\colon E\to X$ is the equalizer of the pair $f,\, g\colon X\rightrightarrows Y$, there exists a unique morphism $d\colon E'\to E$ such that $e\circ d=e'$, concluding the proof.
\end{proof}
\marginpar{Inserito questa nuova definizione}
We end this section with a definition that will be useful in Section \ref{monads}.
\begin{definition}Let $f,g: X\rightrightarrows Y$ be arrows in a category $\X$, we say that they have a \emph{split coequalizers} if:
	\begin{itemize}
		\item  $f$ has a right inverse $t$;
		\item $f$ and $g$ have a coequalizer $e:Y\rightarrow Z$ with a right inverse $s$;
		\item the equality $s\circ e=g\circ t$ holds.
	\end{itemize}
\end{definition}

\begin{remark}\label{split}Notice that a split coequalizer is the same as the colimit on the category $\D$ generated by the diagram
	\[
	\xymatrix@C=30pt{
		A\ar@<-.5ex>[r]_{a}\ar@<.5ex>[r]^{b}&B\ar[r]^{c} \ar@/_.7cm/[l]_{h}&C\ar@/_0.7cm/[l]_{k} 
	}
	\]
	and subjected to the equations $c\circ a=c\circ b$, $c\circ k=\id{C}$, $b\circ h=\id{B}$ and $k\circ c=a\circ h$.
 Since $C$ is the terminal object in this category, it follows that split coequalizers are preserved by any functor.
\end{remark}

\begin{example}\label{split_set}\marginpar{Aggiunti esempio e osservazione per dopo}
Let $E\subseteq X\times X$ be an equivalence relation and $p_1, p_2:E\rightrightarrows X$ be the restriction of the projections, then one coequalizer is given by the arrow $e:X\rightarrow B$ with $B$ the quotient of $X$ by $E$. Notice that this quotient is split: a right inverse $s$ to $e$ is give taking, for any $x\in B$ a representative $s(x)\in e^{-1}(x)$. Moreover, by contruction $(y, s(e(y)))\in E_{(f,g)}$ for every $y\in Y$, so there exists a function
\[t:Y\rightarrow E_{(f,g)} \qquad y\mapsto (y, s(e(y))) \] 
and clearly
\[p_1\circ t=\id{Y}\qquad p_2\circ t=s\circ e\]
\end{example}



\begin{remark}\label{split_set2}
	The previous example exploits two crucial properties of $\Set$:
	\begin{itemize}
		\item the fact that every epimorphism has a right inverse: this is one form of the axiom of choice
		\item the fact that $\Delta_{(\id{Y},s\circ e)}$ factors through the original relation $E$: this is usually summed up under the observation that every equivalence relation is the kernel pair of its coequalizer.
	\end{itemize} 
\end{remark}
\subsubsection{Pullbacks and pushouts}
In this section we examine limits and colimits of functor having as codomain the category generated by the diagram
\[\xymatrix@C=30pt{
& A\ar[d]\\
B\ar[r]& C}
\]
\begin{definition}\label{pb}
Let $f\colon X\to Z$ and $g\colon Y\to Z$ be two morphisms in a category $\X$. The {\em pullback} of $f$ and $g$ is given by an object $P$ and two maps $g'\colon P\to X$ and $f'\colon P\to Y$ that make the following diagram commutative:
\begin{equation}\label{pullback_eq}
\xymatrix@R=10pt@C=40pt{
Q\ar@{.>}[dr]_{\exists!\ \varphi}\ar@/_-15pt/[rrd]^{g''}\ar@/_15pt/[rddd]_{f''}\\
&P\ar@{}[ddr]|-{\text{P.B.}}\ar[dd]_{f'}\ar[r]^{g'}&X\ar[dd]^f\\
\\
&Y\ar[r]_g&Z
}
\end{equation}
and such that, for each object $Q$ and each pair of morphisms $g''\colon Q\to X$ and $f''\colon Q\to Y$ such that $f\circ g''=g\circ f''$, there is a unique morphism $\varphi\colon P\to Q$, making the above diagram commute. 

The {\em pushout} of two morphisms $X\leftarrow Z\to Y$ is defined dually. %as the pullback in the opposite category. 
\end{definition}

\begin{remark}\label{eq_pb}\reversemarginpar \marginpar{Ho inserito questa osservazione che semplifica la dimostrazione del Lemma \ref{mono_are_pullback_stable}} \normalmarginpar
The uniqueness half of Definition \ref{pb}, entails that, given two arrows $\varphi, \psi:Q\rightarrow P$, if $g'\circ \varphi = g'\circ \psi$ and $f'\circ \varphi=f'\circ \psi$ then $\varphi = \psi$. 
\end{remark}

\begin{proposition}\label{pb_eq}\marginpar{Ho promosso i commento a proposizione.}
	Let $\X$ be a category with products and equalizers, then it has also pullbacks. Dually, if $\X$ has coproducts and coequalizers, then it has pushouts. 
\end{proposition}
\begin{proof}Take two arrows $f:X\rightarrow Z$ and $g:Y\rightarrow Z$ and consider the equalizer
	\[
	\xymatrix@C=50pt{
		P:=\eq(f\circ \pi_X,g\circ \pi_Y)\ar[r]^-e&X\times Y\ar@<-.5ex>[r]_-{g\circ\pi_Y} \ar@<.5ex>[r]^-{f\circ \pi_X}&Z.
	}
	\]
thus we have a square
\[
\xymatrix@R=10pt@C=40pt{
	P\ar@{}[ddr]\ar[dd]_{\pi_Y\circ e}\ar[r]^{\pi_X\circ e}&X\ar[dd]^f\\
	\\
	Y\ar[r]_g&Z.
}\]
Now, if $f^{'}:Q\rightarrow X$ and $g^{'}:Q\rightarrow Y $ are such that $f\circ f'=g\circ g'$, then 
\[
(f\circ \pi_X)\circ \Delta_{(f', g')}=(g\circ \pi_Y)\circ \Delta_{(f', g')}
\]
so there exists a unique $\phi:Q\rightarrow P$ such that $e\circ \phi = \Delta_{(f', g')}$, from which the thesis follows. 
\end{proof}
\iffalse 
Consider the pullback square in \eqref{pullback_eq}. An alternative way to characterize the pullback $P$ of the morphisms $f$ and $g$ is through products and equalizers:
\[
\xymatrix@C=50pt{
P:=\eq(f\circ \pi_X,g\circ \pi_Y)\ar[r]^-e&X\times Y\ar@<-.5ex>[r]_-{g\circ\pi_Y} \ar@<.5ex>[r]^-{f\circ \pi_X}&Z.
}
\]
In this construction, the morphism $f'\colon P\to Y$ and $g'\colon P\to X$ are recovered by the formulas $f':=\pi_Y\circ e$ and $g':=\pi_X\circ e$. Hence, if $\X$ has products and equalizers, then it has also pullbacks and, dually, if $\X$ has coproducts and coequalizers, then it has pushouts. 
\fi 
 In the following we will collect some useful facts about the interaction of pullbacks and monomorphisms (dually of pushouts and epimorphisms).


\begin{lemma}\label{mono_are_pullback_stable}
	Consider the following pullback diagram in a category $\X$:
	\begin{equation*}
	\xymatrix@R=10pt@C=40pt{
		P\ar@{}[ddr]|-{\text{P.B.}}\ar[dd]_{f'}\ar[r]^{g'}&X\ar[dd]^f\\
		\\
		Y\ar[r]_g&Z.
	}
	\end{equation*}
	If $f$ is a monomorphism, then $f'$ is a monomorphism. One usually refers to this property by saying that the class of monomorphisms is {\em pullback-stable}. Dually, the class of epimorphisms is {\em pushout-stable}.
\end{lemma} 
\begin{proof}
	Consider two parallel morphisms $a,\, b\colon Q\rightrightarrows P$ such that $f'\circ a=f'\circ b$. Then, 
	\[
	(f\circ g')\circ a=(g\circ f')\circ a=g\circ (f'\circ a)=g\circ (f'\circ b)=(g\circ f')\circ b=(f\circ g')\circ b,
	\]
	and thus $f\circ (g'\circ a)=f\circ (g'\circ b)$. $f$ is monic, so the previous equality implies $g'\circ a=g'\circ b$ and the thesis now follows from Remark \ref{eq_pb}.
	\iffalse 
	Letting $f'':=f'\circ a\colon Q\to Y$ and $g'':=g'\circ b\colon Q\to X$ we get the following commutative diagram:
	\[
	\xymatrix@R=10pt@C=40pt{
		Q\ar@{.>}[dr]|-{\exists!\ \varphi}\ar@/_-15pt/[rrd]^{g''}\ar@/_15pt/[rddd]_{f''}\\
		&P\ar@{}[ddr]|-{\text{P.B.}}\ar[dd]_{f'}\ar[r]^{g'}&X\ar[dd]^f\\
		\\
		&Y\ar[r]_g&Z.
	}
	\]
	By the universal property of the pullback, there exists a unique $\varphi\colon Q\to P$ rendering the diagram commutative, therefore $a=\varphi=b$. 
	\fi 
\end{proof}

Equalizers enjoy  a similar stability property.
\begin{lemma}\label{reg} \marginpar{Ho spostato qui questo lemma dalla sezione su reg}
	Let $r\colon R\rightarrow Y$ be the equalizer of $f, g: Y\rightrightarrows Z$ and suppose that the square
	\[\xymatrix{P\ar[r]^{q}\ar[d]_{p}&R\ar[d]^{r}\\X\ar[r]_{\phi}&Y}
	\]
	is a pullback. Then $p$ is the equalizer of $f\circ \phi$ and $f\circ \phi$.
\end{lemma}
\begin{proof}Let $t\colon T\to X$ be a morphism such that $(f\circ\phi)\circ t=(g\circ\phi)\circ t$ and consider the following commutative diagram:
	\[
	\xymatrix@R=30pt@C=40pt{
		T\ar@/_20pt/[ddrr]_t\ar@{.>}@/_-20pt/[drrr]^{\exists!\, t'}\ar@{.>}[drr]_{\exists!\, s}\\
		&&P\ar@{}[dr]|{\text{P.B.}}\ar[r]^{q}\ar[d]_{p}&R\ar[d]^r\\
		&&X\ar[r]_{\phi}&Y\ar@<-.5ex>[r]_-{g} \ar@<.5ex>[r]^-{f}&Z 
	}
	\]
	Using that $r=\eq(f,g)$ and the equality $f\circ(\phi\circ t)=g\circ(\phi\circ t)$, we deduce that there exists a unique morphism $t'\colon T\to R$ such that $\phi\circ t=r\circ t'$. By the universal property of pullbacks there exists also a unique morphism $s\colon T\to P$ such that $q\circ s=t'$ and $p\circ s=t$. Finally, suppose that there exists a second morphism $s'\colon T\to P$ such that $p\circ s'=t$. But then, \[\phi\circ t=\phi\circ (p\circ s')=(\phi\circ p)\circ s'=(r\circ q)\circ s'=r\circ (q\circ s')\] and, since $t'$ is the unique morphism such that  $\phi\circ t=r\circ t'$, we get that $p_r\circ s'=t'$. Finally, by the universal property of the pullback, we deduce that $s=s'$, concluding the proof.
	%
	%
	%	Let $h$ and $k$ be the arrows $X\rightarrow C$ of which $m$ is the equalizer. We have the solid diagram
	%	\begin{center}
	%		\begin{tikzpicture}
	%			\node(E)at(-2,1){$D$};
	%			\node(F)at(4,-2){$C$};
	%			\node(A)at(0,0){$P$};
	%			\node(B)at(2,0){$E$};
	%			\node(C)at(0,-2){$X$};
	%			\node(D)at(2,-2){$Y$};
	%			\draw[->](A)--(C)node[pos=0.5, right]{$\bar{m}$};
	%			\draw[->](B)--(D)node[pos=0.5, right]{$m$};	
	%			\draw[->](A)--(B)node[pos=0.5, above]{$\bar{f}$};
	%			\draw[->](C)--(D)node[pos=0.5, above]{$f$};	
	%			\draw[->](E)..controls(-1.5,-1)and(-1,-1.5)..(C)node[pos=0.5, left, xshift=-0.1cm]{$d$};
	%			
	%			\draw[dashed,->](E)..controls(1,1.5)and(1.5,1)..(B)node[pos=0.5, above, ]{$l$};
	%			\draw[->](D.350)--(F.190)node[pos=0.5, below]{$h$};
	%			\draw[->](D.20)--(F.160)node[pos=0.5, above]{$k$};
	%			\draw[dashed, ->](E)--(A)node[pos=0.5, above ]{$r$};			
	%		\end{tikzpicture}
	%	\end{center}
	%	and we claim that $\bar{m}$ is the equalizer of $h\circ f$ and $k\circ f$. Let $d:D\rightarrow X$ such that $h\circ f\circ d = k\circ f \circ d$, by the property of $e$ as equalizer we have a $l:D\rightarrow E$ such that $ e\circ l=f\circ h$ which, by the pullback property, yields a $r:D\rightarrow A$ such that $\bar{f}\circ r=l$ and $\bar{m}\circ r=d$.  On the other hand, if $r':D\rightarrow A$ is such that $\bar{m}\circ r'=d$ then 
	%	\begin{align*}e\circ \bar{f}\circ r'&=h\circ \bar{m}\circ r'=h\circ d
	%	\end{align*}
	%	and so $f\circ r'=d'$, from which we deduce $r'=r$.
\end{proof}
Finally we can fully characterize monics (epics) in terms of pullbacks (pushouts).

\begin{proposition}\label{mono_pullback}\marginpar{Aggiunta questa proposizione che rende più modulare la dimostrazione di prop \ref{mono=trivial_ker_prop}}
An arrow $m:X\rightarrow Y$ in a category $\X$ is monic if and only if the square
\[
\xymatrix@R=10pt@C=40pt{
	X\ar@{}[ddr]\ar[dd]_{\id{X}}\ar[r]^{\id{X}}&X\ar[dd]^m\\
	\\
	X\ar[r]_m&Y
}
\]
is a pullback. Dually $m$ is epic if and only if the following diagram is a pushout.
\[
\xymatrix@R=10pt@C=40pt{
	X\ar@{}[ddr]\ar[dd]_m\ar[r]^m&Y\ar[dd]^{\id{Y}}\\
	\\
	Y\ar[r]_{\id{Y}}&Y
}
\]
\end{proposition}
\begin{proof} ($\Rightarrow$) If $f, g:Q\rightarrow X$ are such that $m\circ f=m\circ g$, then, since $m$ is monic, $f=g$ and so $f$ is the unique arrow such the following diagram commutes
	\begin{equation*}
	\xymatrix@R=10pt@C=40pt{
		Q\ar[dr]^-{f}\ar@/_-15pt/[rrd]^{f}\ar@/_15pt/[rddd]_{g}\\
		&X\ar@{}[ddr]\ar[dd]_{\id{X}}\ar[r]^{\id{X}}&X\ar[dd]^m\\
		\\
		&X\ar[r]_m&Y
	}
\end{equation*} 

	\noindent ($\Leftarrow$) Let  $f, g:Q\rightarrow X$ be such that $m\circ f=m\circ g$, by the pullback property there exists a unique $\varphi:Q\rightarrow X$ such that  both $f$ and $g$ are equal to $\id{X}\circ \varphi$ and this implies $f=\varphi=g$.
\end{proof}

\begin{corollary}\label{pb_mono}If $F:\X\rightarrow \Y$ preserves pullbacks then it sends monomorphisms to monomorphisms. Dually a pushout preserving functor preserves epics.
\end{corollary}

Let $(\X, |-|)$ be a concrete category, the previous corollary, with Lemma \ref{faithful}, implies that if $|-|$ preserves pullbacks then monics are exactly the injective arrows. Dually epics are exactly the surjective ones if $|-|$ preserves pushouts.

\begin{example}
 A particularly nice concrete category is the category $\Top$ of all topological spaces and continuous maps. In fact, the forgetful functor $\Top\to \Set$ has both a left adjoint (that associates to any set $S$ the topological space $(S,\delta_S)$, where $\delta_S$ is the discrete topology) and a right adjoint (that associates to any set $S$ the topological space $(S,\tau_S)$, where $\tau_S$ is the indiscrete\marginpar{Ho preferito mettere prima la definizione vera di kernel pair e promuovere la definizione precedente e il Remark 1.27 a proposizione} topology). Hence, the forgetful functor $\Top\to \Set$ preserves pullbacks and pushouts (actually all limits). This shows that, in $\Top$, monomorphisms are exactly the injective continuous maps, while epimorphisms are exactly the surjective continuous maps.
\end{example}


\subsubsection{Kernel pairs and cokernel pairs}

Let $\phi\colon X\to Y$ be a morphism in a given category $\X$. We have shown that it  is possible to describe monomorphisms and epimorphisms using special pushouts and pullbacks (see Proposition \ref{mono_pullback}). In this section we will show that it is possible to use some special co/equalizers in order to do the same  whenever we can construct finite co/products and co/equalizers in  $\X$.


Let us now introduce the two notions we are really interested in:

\begin{definition}
	Let $\phi:X\rightarrow Y$ be a morphism in a category $\X$, its \emph{kernel pair} $(P, f, g)$ is the pullback of $\phi$ along itself. Dually its \emph{cokernel pair} is its pushout along itself.
\end{definition}
\begin{example}In $\Set$ a kernel pair of a function $\phi:X\rightarrow Y$ is given by the relation $R\subseteq X\times X$ given by:
	\[
	R:=\{(x_1,x_2)\in X\times X \mid \phi(x_1)=\phi(x_2)\}
	\]
\end{example}

\begin{proposition}\label{kp_equ} Let $\phi:X\rightarrow Y$ be an arrow in a category $\X$ with binary products and coproducts. Then triple $(P, f, g)$ with $f, g:P\rightarrow X$ is a kernel pair for $\phi$ if and only if $(P, \Delta_{(f,g)})$ is an equalizer for
	\[\xymatrix@C=50pt{
		X\times X\ar@<-.5ex>[r]_-{\pi_2} \ar@<.5ex>[r]^-{\pi_1} \ar@/^15pt/[rr]^{\phi\circ\pi_1}\ar@/_15pt/[rr]_{\phi\circ\pi_2}&X\ar[r]^-{\phi}&Y.
	}
	\]
	
	Dually $(P, f, g)$ with $f,g:Y\rightarrow P$ is a cokernel pair for $\phi$ if and only if  $(P,\nabla_{(f,g)})$ is a coequalizer for
\[
\xymatrix@C=50pt{
	X\ar@/^15pt/[rr]^{\iota_1\circ \phi}\ar@/_15pt/[rr]_{\iota_2\circ \phi}\ar[r]^-{\phi}&Y\ar@<-.5ex>[r]_-{\iota_2} \ar@<.5ex>[r]^-{\iota_1}&Y\sqcup Y
}\]
\end{proposition}
\begin{proof}($\Rightarrow$) By hypothesis we have a pullback square
	\[\xymatrix@R=10pt@C=40pt{
		P\ar@{}[ddr]\ar[dd]_{f}\ar[r]^{g}&X\ar[dd]^\phi \\
		\\
		X\ar[r]_\phi &Y
	}\]
	so 
	\[(\phi \circ \pi_1) \circ \Delta_{(f,g)}=\phi \circ (\pi_1\circ \Delta_{(f,g)})=\phi \circ f=\phi \circ g=\phi \circ (\pi_2\circ \Delta_{(f,g)})=(\phi \circ \pi_2) \circ \Delta_{(f,g)} 	\]
	Now, for every $h:Q\rightarrow X\times X$ such that $(\phi \circ \pi_1)\circ h=(\phi \circ \pi_2)\circ h$ we have a commutative diagram
	\[\xymatrix@R=10pt@C=40pt{
		Q\ar@/_-15pt/[rrd]^{\pi_2\circ h}\ar@/_15pt/[rddd]_{\pi_1\circ h}\\
		&P\ar@{}[ddr]\ar[dd]_{f}\ar[r]^{g}&X\ar[dd]^\phi\\
		\\
		&X\ar[r]_\phi&Y
	}\]
	and thus there exists a unique $\varphi:Q\circ P$ such that $f\circ \varphi =\pi_1 \circ h$ and $g\circ \varphi =\pi_2\circ h$, but these two last equalities are equivalent to $\Delta_{(f,g)}\circ \varphi =h$.
	
	\noindent ($\Leftarrow$) If $(P, \Delta_{(f,g)})$ is an equalizer for $\phi \circ \pi_1$ and $\phi \circ \pi_2$ then
	\begin{equation*}\phi \circ f= \phi \circ (\pi_1\circ \Delta_{(f,g)})=(\phi \circ \pi_1)\circ \Delta_{(f,g)}=(\phi \circ \pi_2)\circ \Delta_{(f,g)}=\phi \circ (\pi_2\circ \Delta_{(f,g)})=\phi \circ g
	\end{equation*}
	Now, take $h,k:Q\rightarrow X$ such that $\phi \circ h=\phi \circ k$, then $(\phi \circ \pi_1)\circ \Delta_{(h,k)}=(\phi \circ \pi_2)\circ \Delta_{(h,k)}$. Thus there exists a unique $\varphi$ such that $\Delta_{(f,g)}\circ \varphi=\Delta_{(h,k)}$, which amounts to $f\circ \varphi=h$ and $g\circ \varphi =k$.
\end{proof}

Given the previous proposition, we will use the expressions ``kernel pair'' and ``cokernel pair'' directly for the arrows $\Delta_{(f,g)}$ and $\nabla_{(f,g)}$. 

Before proceeding further, let us name those categories $\X$ where all these constructions can be done.

\begin{definition}
A category $\X$ is said to be {\em finitely bicomplete} if every pair of objects admits both a product and a coproduct in $\X$, and every pair of parallel morphisms admits both an equalizer and a coequalizer.
\end{definition}

\iffalse 

\begin{definition}
Let $\phi\colon X\to Y$ be a morphism in a finitely bicomplete category $\X$, and consider the following diagrams
\[
\xymatrix@C=50pt{
X\ar@/^15pt/[rr]^{\iota_1\circ \phi}\ar@/_15pt/[rr]_{\iota_2\circ \phi}\ar[r]^-{\phi}&Y\ar@<-.5ex>[r]_-{\iota_2} \ar@<.5ex>[r]^-{\iota_1}&Y\sqcup Y
}
\qquad\text{and}\qquad\xymatrix@C=50pt{
X\times X\ar@<-.5ex>[r]_-{\pi_2} \ar@<.5ex>[r]^-{\pi_1} \ar@/^15pt/[rr]^{\phi\circ\pi_1}\ar@/_15pt/[rr]_{\phi\circ\pi_2}&X\ar[r]^-{\phi}&Y.
}
\]
The {\em cokernel pair}  and the {\em kernel pair} of $\phi\colon X\to Y$ are defined, respectively, as the coequalizer of $\iota_1\circ \phi$ and $\iota_2\circ \phi$, and the equalizer  of the pair $\phi\circ\pi_1$ and $\phi\circ\pi_2$.
%\[
%\gamma\colon X\sqcup X\to X\sqcup_M X
%\] 
%of the two maps $\mu_1,\,\mu_2\colon M\rightrightarrows X\sqcup X$ described above. 
\end{definition}
\fi 
\iffalse 
This construction can be described in several equivalent ways and, depending on the choice, it may receive different names. For example, it is common to use pushouts to define cockernel pairs, as sketched in the following remark:
\begin{remark}
One can equivalently describe 
%\maltese\footnote{As above, it remains the doubt that one need not consider only pushouts with coinciding arrows $c$ and $d$ (see the modified diagram above, where the arrow names $c$ and $d$ is added now), as in the definition of cockernel pairs 
%given above yields, where only this special kind of pushouts (with $c=d$) are taken into account. Again, maybe there is some technical error somewhere ?  \\
%% (see also \cite[Proposition 11.33]{cats}). 
%Usually (in other sources) cokernel pairs are defined as {\bf pairs} $c,d$ with the pushout property. 
% This dually applies to kernel pairs (see the previous footnote), which appear in the above definition as a single morphism, whereas elsewhere in the literature 
% they are {\bf pairs} of morphisms defined as the pullback of the pair of equal maps $\phi: X\to Y\leftarrow  X$. In my view we need to carefully 
% revise these definitions (I definitely prefer the PO/PB approach as in this remark), or in case the definitions are just fine (which I do not believe), give examples in the basic categories of interest (at least $\Set$, $\Ab$). 
%  Examples and counter-examples are the real ``meat" in mathematics, we cannot only give definitions and prove theorems/lemmas/etc.
%  risking to declare something ``trivial" whereas example can show it is  simply wrong or vacuous.
% } 
 the cokernel pair of a map $\phi\colon X\to Y$ is by taking the pushout of the diagram $Y\leftarrow X\to Y$:
\[
\xymatrix@C=50pt{
X\ar@{}[dr]|{\text{P.O.}}\ar[d]_{\phi}\ar[r]^{\phi}&Y\ar[d]^c\\
Y\ar[r]_d&C.
}
\]
\end{remark}
\fi 
\begin{proposition}\label{mono=trivial_ker_prop}
Let $\X$ be a finitely bicomplete category. The following are equivalent for a morphism $\phi\colon X\to Y$ in $\X$:
\begin{enumerate}[\rm (1)]
\item $\phi$ is a monomorphism (resp., an epimorphism);
\item the kernel pair of $\phi$ is $\Delta_X\colon X\to X\times X$ (resp., the cokernel pair of $\phi$ is $\nabla_Y\colon Y\sqcup Y\to Y$).
\end{enumerate}
\end{proposition}
\begin{proof}
	By Lemma \ref{mono_pullback}, $\phi$ is monic if and only if the square
	\[
	\xymatrix@R=10pt@C=30pt{
		X\ar@{}[ddr]\ar[dd]_{\id{X}}\ar[r]^{\id{X}}&X\ar[dd]^\phi\\
		\\
		X\ar[r]_\phi&Y
	}
	\]
	is a pullback, by Proposition \ref{kp_equ} this is equivalent to $\Delta_X$ being the kernel pair of $\phi$.\qedhere 
	
	
	\iffalse 
	
(1)$\Rightarrow$(2).
Let $(X\times X, (\pi_i)_{i=1,2})$ be the product of two copies of $X$. By Lemma \ref{easy_lemma_monos_and_eq}, whenever $\phi$ is a monomorphism, the equalizer of the maps $\phi\circ \pi_1$ and $\phi\circ \pi_2$ (that is, the kernel pair of $\phi$) is the same as the equalizer of the maps $\pi_1$ and $\pi_2$, that is, $\Delta_X\colon X\to X\times X$ (see Example \ref{ex_eq_of_projections})

\smallskip\noindent
(2)$\Rightarrow$(1). If $\Delta_X\colon X\to X\times X$ is the kernel pair of $\phi$ then the square


is a pullback and we can conclude using Lemma \ref{mono_pullback}.
Let $(X\times X, (\pi_i)_{i=1,2})$ be the product of two copies of $X$, suppose that  $\Delta_X\colon X\to X\times X$ is the kernel pair of $\phi$ and take two maps $f,\, g\colon Z\rightrightarrows X$ such that $\phi\circ f=\phi\circ g$, as in the following diagram
\[
\xymatrix@R=30pt@C=50pt{
X\ar[r]^{\Delta_X}&X\times X\ar@<-.5ex>[r]_-{\pi_1} \ar@<.5ex>[r]^-{\pi_2}&X\ar[r]^\phi&Y\\
&Z\ar@{.>}[lu]^{\exists!h'}\ar@{.>}[u]^{\exists!h}\ar@<-.5ex>[ur]_(.4){f} \ar@<.5ex>[ur]^(.4){g}}.
\]
By the universal property of the product, there exists a unique map $h\colon Z\to X\times X$ such that $\pi_1\circ h=f$ and $\pi_2\circ h=g$. On the other hand, we have the following equality: 
\[
\phi\circ\pi_1\circ h=\phi\circ f=\phi\circ g=\phi\circ \pi_2\circ h
\]
that, by the universal property of the equalizer, implies that there exists a unique map $h'\colon Z\to X$ such that $h=\Delta_X\circ h'$. To conclude just note that, since $\pi_i\circ \Delta_X=\id {X}$,
\[
f=\pi_1\circ h=\pi_1\circ \Delta_X\circ h'=\id X\circ h'=\pi_2\circ \Delta_X\circ h'=\pi_2\circ h=g.
\qedhere
\]
\fi 
\end{proof}
\begin{corollary}\label{mono_is_injective}
	Let $(\X,|-|)$ be a concrete category and suppose that:
	\begin{enumerate}[\rm (1)]
		\item the category $\X$ is finitely bicomplete; 
		\item the forgetful functor $|-|\colon \X\to \Set$ preserves finite products and equalizers.
	\end{enumerate}
	Then, for every monomorphism $\phi\colon X\to Y$ in $\X$, the map $|\phi|\colon |X|\to |Y|$ is injective.
\end{corollary}
\begin{proof}
	By Proposition \ref{mono=trivial_ker_prop}, $\phi$ is a monomorphism if, and only if, its kernel pair is the diagonal $\Delta_X\colon X\to X\times X$. Since the forgetful functor is supposed to commute with finite products, it is not difficult to verify that $|\Delta_X|=\Delta_{|X|}\colon |X|\to |X|\times |X|$. Furthermore, since the forgetful functor also commutes with equalizers, it has to send kernel pairs to kernel pairs, and so the kernel pair of $|\phi|$ is exactly $\Delta_{|X|}$, showing that $|\phi|$ is a monomorphism in $\Set$, that is, $|\phi|$ is injective.
\end{proof}
\begin{remark}If $\X$ is bicomplete and $|-|$ preserves finite products and equalizers then Proposition $\ref{pb_eq}$ entails that $|-|$ preserves pullbacks too, so we can get the applying Corollary \ref{pb_mono}.
\end{remark}
\begin{remark}\label{Div:vs:left:adjoint} 
	The relevance of item (2) in the above corollary is clear from the following example. Let $\mathfrak D iv$ denote the complete subcategory of $\Ab$ 
	with objects all divisible groups. Then $\mathfrak D iv$ satisfies (1) (actually it is bicomplete), but the monomorphisms in $\mathfrak D iv$ 
	need not be injective (e.g., $m: \Q \to \Q/\Z$ is a non-injective monomorphism), this means the (2) fails (indeed, the equalizer of $m$ and the zero morphism $0: \Q \to \Q/\Z$, 
	i.e., the kernel of $m$, is the 0 subgroup of $\Q$, which differs from the
 equalizer formed in $\Set$ of $|m|$ and $|0|$, which is $\Z$). 
\end{remark}
 
\subsection{Some facts about monads}\label{monads} \reversemarginpar \marginpar{Questo dovrebbe anche semplificare la dimostrazione della proposizione 3.15}\normalmarginpar 	\marginpar{Ho inserito una sezione sulle monadi: questo serve a rendere preciso e formale i riferimenti a categorie di origine algebrica''. }

In this section we will recall some notion about \emph{monads} and \emph{monadic categories}, a particular kind of concrete categories which subsume all categories of algebraic origin'' such as $\Grp, \Ring$ or $\Modu$.

\begin{definition}A \emph{monad} on a category $\X$ is a triple $(T, \eta, \mu)$ where $T:\X\rightarrow \X$ is a functor and $\eta:\id{\X}\rightarrow T$, $\mu:T\circ T\rightarrow T$ are natural transformations such that the following diagrams commute. 

	\[
	\xymatrix@C=20pt{
		T\circ T \circ T\ar[d]_{\mu T}\ar[r]^{T\mu}&T\circ T\ar[d]^\mu 
		\\
		T\circ T\ar[r]_\mu&T
	}	\quad 
\xymatrix@C=20pt{
	T\ar[r]^{T\eta}\ar[dr]_{\id{T}}&T\circ T\ar[d]^\mu  &T \ar[l]_{\eta T}\ar[dl]^{\id{T}}
	\\
	 &T
}
	\]
\end{definition}
\begin{proposition}Let $U:\X\rightarrow \Y$ be a functor with a left adjoint $F$. Let also $\eta$ and $\epsilon$ be the unit and the counit of the adjunction, then $(U\circ F, \eta, U(\epsilon_{F(-)} )$ is a monad on $\Y$.
	
\end{proposition}
\begin{proof}
	The first square is obtained applying $U$ to the naturality square
	\[
	\xymatrix@C=40pt{F(U(F(U(F(Y))))\ar[r]^-{F(U(\epsilon_{F(Y)}))}\ar[d]_-{\epsilon_{F(U(F(Y)))}} & F(U(F(Y)))\ar[d]^{\epsilon_{F(Y)}}\\
	F(U(F(Y)))\ar[r]_{\epsilon_{F(Y)}}& F(Y)}
	\]
For the two triangles, we can start with the triangular identities of the adjunction:
\[\xymatrix{F(Y)\ar@/^0.75cm/[rr]^{\id{F(Y)}}\ar[r]_-{F(\eta_Y)} & F(U(F(Y))) \ar[r]_-{\epsilon_{F(Y)}} & F(Y)\\
U(X) \ar@/_0.75cm/[rr]_{\id{F(Y)}}\ar[r]^-{\eta_{U(X)}}& U(F(U(X)))\ar[r]^-{U(\epsilon_X)}& U(X)}\]
Applying $U$ to the first and instatiating the second with $X=F(Y)$ we get the thesis.	
\end{proof}

\begin{definition} Given a monad $(T, \eta, \mu)$ on a category $\X$, an \emph{Eilenberg-Moore algebra} for $T$ is an arrow $\xi:T(X)\rightarrow X$ ($X$ an object of $\X$) such that 
	\[\xymatrix{X\ar[r]^{\eta_X} \ar[dr]_{\id{X}} & T(X)\ar[d]^{\xi}\\ & X} \qquad \xymatrix{ T(T(X))\ar[r]^-{\mu_X} \ar[d]_{T(\xi)} & T(X)\ar[d]^{\xi}\\ T(X)\ar[r]_{\xi} & X}
	\]
commute. A morphism between	$\xi_1:T(X)\rightarrow X$ and $\xi_2:T(Y)\rightarrow Y$ is an arrow $\phi: X\rightarrow Y$ such that the following square commutes
\[\xymatrix{T(X)\ar[r]^{T(\phi)} \ar[d]_{\xi_1}& T(Y)\ar[d]^{\xi_2}\\X\ar[r]_{\phi}& Y}\]
We will denote with $\eim{T}$ the resulting category of Eilenberg-Moore algebras.
\end{definition}

Clearly we have a faithful forgetful functor $U_T:\eim{T}\rightarrow \X$ which sends $\xi:T(X)\rightarrow X$ to $X$ and is the identity on arrows. This functors has always a left adjoint, which sends an object to the \emph{free algebra} on it.

\begin{proposition} Let $(T, \eta, \mu)$ be a monad on the category $\Y$, then the forgetful functor $U_T:\eim{T}$ has a left adjoint $F_T$ which sends $Y$ to $\mu_Y:T(T(Y))\rightarrow T(Y)$.
\end{proposition}
\begin{proof} The axioms of monad entail at once that $\mu_Y$ is an Eilenberg-Moore algebra. Let show that $\eta$ has the universal property of the unit of an adjunction. Let $\xi:T(X)\rightarrow X$ be an Eilenberg-Moore algebra and $\phi:Y\rightarrow X$ a morphism of $\Y$ and consider the composition $\psi=\xi\circ T(\phi):T(Y)\rightarrow X$. Pasting together the naturality diagrams of $\eta$, $\mu$ and those in the definition of Eilenberg-Moore algebras we get:
	\[\xymatrix{Y\ar[r]^{\phi} \ar[d]_{\eta_Y} & X \ar[d]_{\eta_X} \ar[dr]^{\id{X}} & \\T(Y) \ar[r]_{T(\phi)}& T(X)\ar[r]_{\xi} & X} \qquad 
	\xymatrix{T(T(Y))\ar[r]^{T(T(\phi))} \ar[d]_{\mu_Y} & T(T(X))\ar[r]^{T(\xi)} \ar[d]_{\mu_X} & T(X)\ar[d]^{\xi}\\T(Y) \ar[r]_{T(\phi)}& T(X) \ar[r]_{\xi}& X}
	\]
showing $\psi$ is a morphism $\mu_{Y}\rightarrow \xi $ and that $U_T(\psi)\circ\eta_{Y} = \phi$. We are left with uniqueness, but if $\theta :\mu\rightarrow \xi$ is a morphism in $\eim(T)$ such that $U_T(\theta)\circ \eta_(Y)=\phi$ then
\[\xymatrix{T(Y)\ar[r]^{T(\eta_Y)} \ar[dr]_{\id{T(Y)}}\ar@/^0.75cm/[rr]^{T(\phi)}& T(T(Y)) \ar[d]^{\mu_Y}\ar[r]^{T(\theta)}& T(X)\ar[d]^{\xi}\\ & T(Y)\ar[r]_{\phi} &X} \]
commutes and thus $\phi=\xi \circ T(\phi)$.	
 \end{proof} 
Clearly $U_T\circ F_T=T$, whenever a functor $U:\X\rightarrow \Y$ has a left adjoint $F$ such that $U\circ F=T$ we can canonically compare $\X$ with $\eim{T}$.  
\begin{proposition}
	Let $U:\X\rightarrow \Y$ be a functor with a left adjoint $F$ and $(T, \eta, \mu)$ the induced monad. Then there exists a \emph{comparison functor} $K:\X\rightarrow \eim{T}$ which sends an object $Y$ to $U(\epsilon_X):U(F(U(X)))\rightarrow U(X)$
	where $\epsilon$ is the counit of $F\dashv U$.	
\end{proposition}
\begin{proof}
	First of all we have to verify that $U(\epsilon_X)$ is an Eilenberg-Moore algebra. One of the axioms is just one of the triangular identities, the other is obtained applying $U$ to the naturality square
	\[
	\xymatrix@C=40pt{F(U(F(U(X))))\ar[r]^-{\epsilon_{F(U(X))}}\ar[d]_{F(U(\epsilon_X))} & F(U(X))\ar[d]^{\epsilon_X}\\ F(U(X)) \ar[r]_{\epsilon_X} & X }
	\]
	Given $\phi:X_1\rightarrow X_2$, if we apply $U$ to the naturality square
	\[	\xymatrix@C=40pt{F(U(X_1))\ar[r]^-{F(U(\phi))}\ar[d]_{\epsilon_{X_1}} & F(U(X_2))\ar[d]^{\epsilon_{X_2}}\\ X_1\ar[r]_{\phi} & X_2 } \] 
	 we get that $U(\phi)$ is an arrow of $\eim{T}$, so we can conclude the proof defining $K(f):=U(f)$.
\end{proof}

\begin{definition}A concrete category $(\X, |-|)$ is \emph{(strictly) monadic over $\Set$} if $|-|$ has a left adjoint $F$ and the comparison functor of the previous lemma is an proposition (isomorphism).
\end{definition}
\begin{remark}Notice that, by definition, a (strictly) monadic category $(\X, |-|)$ is concretely equivalent (isomorphic) to $(\eim{T}, U_T)$ for some monad $T:\Set\rightarrow \Set$.
\end{remark}

 Let  $(\X, |-|)$ be monadic over $\catname{Set}$, since right adjoints preserve pullbacks, Lemma \ref{faithful} and Corollary \ref{pb_mono} entail that monomorphisms in $\X$ are precisely the injective arrows.
\begin{remark} We can deduce from Remark \ref{Div:vs:left:adjoint} that the category $\mathfrak D iv$ of divisible group is not monadic over $\catname{Set}$ since the forgetful functor is not a right adjoint.
	The reader may be tempted to consider, for a set $S$, the divisible group $\Q^{(S)}$ as a free object in $\mathfrak D iv$ associated to $S$, but easy examples show that the assignment $S \mapsto \Q^{(S)}$ fails to produce a functor, which witnesses again the failure of the property (2) in Corollary~\ref{mono_is_injective}.
\end{remark}


Our interest in monadic categories is justified by the following classical result (\cite{bor2,stone,algth,ManAlg,lawvere}).
\begin{theorem}Every finitary algebraic theory $\mathbb{T}$ gives rise to a monad on the category of $\Set$ whose category of Eilenberg-Moore algebras is equivalent to the category of model of $\mathbb{T}$.
\end{theorem}

In particular every category of ``algebraic origin'' such as the categories of rings, groups, monoids, etc.\dots are monadic over $\Set$.

\begin{remark} The word finitary in the above theorem means that every operation in the algebraic theory must have a finite ariety. The result still holds if we ask for all the arieties to be bounded by some cardinal $\kappa$. Notice that without this hypothesis an algebraic theory can fail to produce a monad (an example is given by the theory of complete boolean algebras, see \cite{stone}). 
\end{remark}

\begin{remark}\label{equational}A partial converse can also be proved: every category monadic over $\Set$ is concretely isomorphic to the category of models for some algebraic theory (\cite{linton2,linton3,stone}), eventually with operations of arbitrary large arieties (for instance this is the case for complete semilattices, see examples below). In general it is possible to fully characterize monads coming from a finitary algebraic theory in terms of preservation of certain colimits, called \emph{filtered} (see \cite{pres,algth} ). Such monads are called \emph{finitary}.
\end{remark}
 
We can now list some examples of monadic categories.

\begin{example}\begin{itemize}
		\item The left adjoint to the forgetful functor $\Mon\rightarrow \Set$ associates to a set $S$ the free monoid of words in the alphabet $S$ the resulting monad is the well known \emph{Kleene star}.
		
		\item The left adjoint to the forgetful functor $\Grp\to \Set$ associates to a set $S$ the free group of words in the alphabet $S\sqcup S^{-1}$, where $S^{-1}$ is the set of formal inverses $\{s^{-1}:s\in S\}$ of the elements in $S$.
		\item The left adjoint to the forgetful functor $\Ab\to \Set$ associates to a set $S$ the free Abelian group $\Z^{(S)}$, that is the sum of a number of copies of $\Z$ equals to the cardinality of $S$.
		
		\item  The left adjoint to the forgetful functor $\SLatt\to \Set$ associates to a set $S$ the $0$-join-semilattice of finite subsets of $S$, partially ordered by inclusion. The resulting monad  $\mathcal{P}_f$ sends a set $S$ to the set of its finite subsets.
		
		\item It is a theorem of Manes that the category $\CHaus$ of compact Hausdorff toplogical spaces is monadic over $\Set$. The resulting monad $U$ sends a set $S$ to the set $U(S)$ of the \emph{ultrafilters on} $S$ \cite{Rich, ManAlg}. 
		
		\item Let $\CSLatt$ be the category of complete join-semilattices, the left adjoint to the forgetful functor $\CSLatt\to \Set$ associates to a set $S$ the complete join-semilattice subsets of $S$, partially ordered by inclusion. The resulting monad $\mathcal{P}$ sends a set $S$ to its powerset.
	\end{itemize}
\end{example}

\begin{remark}The last two examples are not finitary. It can be shown that the algebraic theory of compact Hausdorff spaces requires an operation of arity $\aleph_0$ (assigning a limit to a succession), while the theory of complete semilattices requires operations of arbitrarily large arity (assigning to a subset its supremum).
\end{remark}

\subsubsection{Limits and colimits in categories of Eilenberg-Moore algebras}
In this section we examine the existence of limits and colimits in categories of Eilenberg-Moore algebras. As a result we will show how to compute limits and colimits in categories monadic over $\Set$.

\begin{proposition}\label{limcolim}
	Let $(T, \eta, \mu)$ be a monad on $\X$ and $F:\D\rightarrow \eim{T}$ a functor, then
	\begin{enumerate}[\rm (1)]
		\item if $U_T\circ F$ has a limit $(L, \{l_D\}_{D\in \D})$ then there exists a unique $\xi:T(L)\rightarrow L$ in $\eim{T}$ which makes every $l_D$ an arrow of $\eim{T}$ and moreover $\xi$ is a limit for $F$;
		\item if $U_T\circ F$ has a colimit $(L, \{l_D\}_{D\in \D})$ which is preserved by $T$ and by $T\circ T$ then there exists a unique $\xi:T(L)\rightarrow L$ in $\eim{T}$ which makes every $l_D$ an arrow of $\eim{T}$ and moreover $\xi$ is colimit for $F$.
	\end{enumerate} 
\end{proposition}
\begin{remark}If $T$ preserves all colimits of a certain shape $\D$, then, in item (2), the preservation of the same kind of colimits by $T\circ T$ follows for free.
\end{remark}
\begin{proof}
	For every $D\in \D$ let $F(D)$ be the algebra $\xi_D:T(X_D)\rightarrow X_D$.
	\begin{enumerate}[\rm(1)]
		\item Each $l_D$ must be a morhpism of $\eim{T}$, so we must have a commutative square
		\[
		\xymatrix@C=20pt{
			T(L)\ar[r]^{\xi} \ar[d]_{T(l_D)}& L\ar[d]^{l_D}\\
			T(X_D)\ar[r]_{\xi_D} & X_D	
		}
		\]
		Thus $\xi$ must be the unique arrow $T(L)\rightarrow L$ such that $l_D\circ \xi=\xi_D\circ T(l_D)$. $\xi$ define an actual object of $\eim{T}$. On one hand
		\begin{align*}
		&l_D\circ \xi \circ T(\xi)= \xi_D\circ T(l_D)\circ T(\xi)=\xi_D\circ T(l_D\circ \xi)=\xi_D\circ T(\xi_D\circ T(l_D))\\=&\xi_D\circ T(\xi_D)\circ T(T(l_D))=\xi_D \circ \mu_{X_D}\circ T(T(l_D))=\xi_D \circ T(l_D)\circ \mu_{L}=l_D\circ \xi \circ \mu_L
		\end{align*}
		from which it follows that $\xi\circ T(\xi)=\xi \circ \mu_L$. On the other hand we have a commutative diagram
		\[
		\xymatrix{
			L\ar[r]^{\eta_L}\ar[d]_{l_D}&T(L)\ar[r]^{\xi}\ar[d]_{T(l_D)} &L\ar[d]_{l_D}\\
			X_D\ar@/_0.5cm/[rr]_{\id{X_D}}\ar[r]^{\eta_{X_D}} & T(X_D)\ar[r]^{\xi_D} & X_D
		}
		\]
		therefore $l_D\circ (\xi\circ \eta_L)=l_D$ and thus $\xi\circ \eta_L=\id{L}$.
		
		We are left with the limiting property. Consider a cone on $F$ with vertex $\theta:T(Q)\rightarrow Q$ and structural arrows $f_D:\theta \rightarrow \xi_D$, then $(Q, \{f_D\}_{D\in \D})$ is a cone for $U_T\circ F$ and thus there is a unique $f:Q\rightarrow L$. If we show that $f$ defines an arrow of $\eim{T}$ we are done.	We have
		\[l_D\circ \xi\circ T(f) =\xi_D\circ T(l_D)\circ T(f)=\xi_D\circ T(l_D\circ f)=\xi_D\circ T(f_D)=f_D\circ \theta =l_D\circ (f\circ \theta)\]
		from which it follows that $\xi\circ T(f)=f\circ \theta$.
		
		\item By hypothesis $(T(L_D), T(l_D)_{D\in \D})$ is a colimit for $T\circ U_T\circ F$. Now if $l_D$ is a morhpism of $\eim{T}$ then we must have a commutative square
		\[
		\xymatrix@C=20pt{
			T(X_D)\ar[r]^{\xi_D} \ar[d]_{T(l_D)} & X_D\ar[d]^{l_D}\\
			T(L)\ar[r]_{\xi} & L	
		}
		\]
		and thus $\xi$ must be the unique arrow $T(L)\rightarrow L$ such that $ \xi \circ T(l_D)=l_D\circ \xi_D$.   We have to show that $\xi$ is in $\eim{T}$,
		On one hand we have that:
		\begin{align*}
		& \xi \circ T(\xi)\circ  T(T(l_D))= \xi \circ T(\xi\circ T(l_D))=\xi \circ T(l_D\circ \xi_D)=(\xi \circ T(l_D))\circ T(\xi_D)\\=&l_D\circ (\xi_D\circ T(\xi_D))=(l_D\circ \xi_D)\circ \mu_{X_D}) =\xi\circ (T(l_D)\circ \mu_{X_D})=\xi\circ \mu_L\circ T(T(l_D))
		\end{align*}
		while on the other the following diagram commutes
		\[
		\xymatrix{
			X_D\ar[d]_{l_D}\ar@/^0.5cm/[rr]^{\id{X_D}}\ar[r]_{\eta_{X_D}} & T(X_D)\ar[r]_{\xi_D} \ar[d]_{T(l_D)} & X_D \ar[d]^{l_D}\\
		L\ar[r]_{\eta_L}&T(L)\ar[r]_{\xi} &L	
	}
		\]
		Therefore, 	since  $(T(T(L)), \{T(T(l_D))\}_{D\in \D})$ is a colimit for $T\circ T\circ U_T\circ F$ we can conclude that
		\[\xi \circ T(\xi)=\xi\circ \mu_L\qquad \xi\circ \eta_L=\id{L}\]. The colimiting property is proven as in the previous point: consider a cocone on $F$ with vertex $\theta:T(Q)\rightarrow Q$ and arrows $f_D:\xi_D \rightarrow \theta$, then $(Q, \{f_D\}_{D\in \D})$ is a cocone for $U_T\circ F$ which induces a unique $f:L\rightarrow Q$ which is an arrow of $\eim{T}$:
		\[\theta\circ T(f)\circ T(l_D) =\theta\circ T(f_D)=f_D\circ \xi_D=f\circ (l_D\circ \xi_D)=(f\circ \xi) \circ T(l_D)   \qedhere\] 
	\end{enumerate}	
\end{proof}

\begin{corollary}\label{split_coeq}
	Let $(T, \eta, \mu)$ be a monad on a category $\X$, $\xi_1:T(X)\rightarrow X$ and $\xi_2:T(Y)\rightarrow Y$ two objects of $\eim{T}$ and $f,g:\xi_1 \rightrightarrows \xi_2$ two arrows between them such that $U_T(f)$ and $U_T(g)$ admit a split coequalizer $e:Y\rightarrow Z$ in $\X$. Then there exists a unique $\theta:T(Z)\rightarrow Z$ in $\eim{T}$ such that $e:\xi_2\rightarrow \theta$ is a coequalizer of $f$ and $g$. 
\end{corollary}
\begin{proof}By hypothesis in $\X$ we have a diagram 	
	\[
	\xymatrix@C=40pt{
		X\ar@<-.5ex>[r]_{g}\ar@<.5ex>[r]^{f}&Y\ar[r]^{e} \ar@/_.7cm/[l]_{t}&Z\ar@/_0.7cm/[l]_{s} 
	}
	\]
	in which $s$ and $t$ are, respectively, the left inverses of $e$ and $f$ and $s\circ e=g\circ t$ and $e$ is the coequalizer of $f$ and $g$. Split coequalizers are absolute (see Remarks \ref{absolute} and \ref{split}), thus 
	\[
	\xymatrix@C=40pt{
		T(X)\ar@<-.5ex>[r]_{T(g)}\ar@<.5ex>[r]^{T(f)}&T(Y)\ar[r]^{T(e)} \ar@/_.7cm/[l]_{T(t)}&T(Z)\ar@/_0.7cm/[l]_{T(s)} 
	} \qquad 	\xymatrix@C=40pt{
	T(T(X))\ar@<-.5ex>[r]_{T(T(g))}\ar@<.5ex>[r]^{T(T(f))}&T(T(Y))\ar[r]^{T(T(e))} \ar@/_.7cm/[l]_{T(T(t))}&T(T(Z))\ar@/_0.7cm/[l]_{T(T(s))} 
}
	\]
	are split coequalizers: in particular this implies that $T(e)$ and $T\circ T(e)$ are equalizers of $f$ and $g$ and we can conclude by point (2) of Proposition \ref{limcolim}.
\end{proof}

\begin{remark}Notice that the coequalizer of $f$ and $g$ in $\eim{T}$ does not need to be a split one: there is no need for $s$ and $t$ to be images of morphisms in $\eim{T}$.
\end{remark}
\begin{theorem}\label{bicomp}
	For any monad $(T, \eta, \mu)$ on $\Set$, $\eim{T}$ has coequalizers and binary coproducts. In particular every monadic category $(\X, |-|)$ over $\Set$ is finitely bicomplete.
\end{theorem} 
\begin{proof}Let's split the proof in two steps.
	\begin{itemize}
		\item $\eim{T}$ has coequalizers. Let  $\xi_1:T(X)\rightarrow X$ and $\xi_2:T(Y)\rightarrow Y$ be two Eilenberg-Moore algebras and $f,g:\xi_1\rightrightarrows\xi_2$ be two parallel arrows between. By point (1) of $\ref{limcolim}$ we can consider the morphism $\Delta_{f,g}:\xi_1\rightarrow \xi$ where $\xi:T(Y\times Y)\rightarrow Y\times Y$ is the product of $\xi_1$ and $\xi_2$. Now, we define the set $E_{(f,g)}\subseteq Y\times Y$ as the intersection of all $R\subseteq Y\times Y$ such that:
		\begin{itemize}
			\item $R$ is an equivalence relation;
			\item $R=U_T(\theta)$ for some Eilenberg-Moore algebra $\theta:T(R)\rightarrow R$;
			\item the inclusion $R\subseteq Y\times Y$ is an arrow $i_R:\theta_R\rightarrow \xi$ in $\eim{T}$;
			\item $\Delta_{(f,g)}=i_R\circ \phi_R$ for some $\phi_R\in \eim{T}$.
		\end{itemize}
	By Example \ref{intersection} and Proposition \ref{limcolim} there exists a unique $\alpha:T(E_{(f,g)})\rightarrow E_{(f,g)}$ which is the limit of the diagram made by the arrow $i_R:\theta_R\rightarrow \xi$. Let $i:\alpha \rightarrow \xi$ be the morphism of $\eim{T}$ given by the inclusion (which is equal to  $i_R\circ \pi_R$ for every inclusion $i_R$ and projection $\pi_R:\alpha\rightarrow \theta_R$ ), composing with the projections $\pi_1: \xi \rightarrow \xi_1$  and $\pi_2: \xi \rightarrow \xi_2$ we get a pair of parallel arrows $i_1, i_2:\alpha\rightrightarrows \xi_2$. In $\Set$ we have a coequalizer diagram:
	\[
	\xymatrix@C=30pt{
		E_{(f,g)}\ar@<-.5ex>[r]_{i_2}\ar@<.5ex>[r]^{i_1}&Y\ar[r]^{e}&Q 
	}
	\] 
	Notice that $E_{(f,g)}$ is an equivalence relation, hence by Example \ref{split_set} the previous coequalizer is a split one, therefore  Corollary \ref{split_coeq} implies that $i_1, i_2:\alpha \rightrightarrows \xi_2$ have a coequalizer $\theta$. Now it is enough to show the following claim.
	\begin{claim*}
		Let $\psi:\xi_2\rightarrow \beta$ be a morphism in $\eim{T}$, then $\psi\circ i_1=\psi \circ i_2$ if and only if $\psi \circ f=\psi \circ g$.
	\end{claim*}
\begin{proof}
	$(\Rightarrow)$ By construction there exists $\phi: X\times X\rightarrow E_{(f,g)}$ such that 
	\[\Delta_{(f,g)}=i\circ \phi=\Delta_{(i_1\circ \phi, i_2\circ \phi)}\]
 which implies $f=i_1\circ \phi$ and $g=i_2\circ \phi$, therefore
	\[\psi \circ f=\psi\circ (i_1\circ \phi)=(\psi\circ i_1)\circ \phi= (\psi\circ i_2)\circ \phi = \psi \circ (i_2\circ \phi)=\psi \circ g\]

	\noindent $(\Leftarrow)$ Define
	\[K=\{(x,y)\in Y \mid \psi(x)=\psi(y)\}\]
	then $K$, with the restriction of the two projections, is the kernel pair of $\psi$ in $\Set$ and, by Proposition $\ref{limcolim}$, there exists
	$\kappa: T(K)\rightarrow K$ which is the kernel pair of $\psi$ in $\eim{T}$. Notice that $K$ is an equivalence relation which, by hypothesis, contains the image of $\Delta_{(f,g)}$, thus $E_{(f,g)}\subseteq K$, but then $\psi\circ i_1=\psi\circ i_2$.
\end{proof}
	
	\item $\eim{T}$ has binary coproducts. Take two algebras $\xi_1:T(X_1)\rightarrow X_1$ and $\xi_2:T(X_2)\rightarrow X_2$. Define two parallel arrows $\iota, \gamma:T(X_1)+(X_2)\rightrightarrows T(X_1+X_2)$ putting
	\[\iota:=\nabla_{(T(\iota_1), T(\iota_2))} \qquad \gamma:=\eta_{X_1+X_2}\circ \nabla_{(\iota_1\circ \xi_1 \iota_2\circ \xi_2)}
	\]
	
	Now, $T(X_+X_2)=U_T(\mu_{X_1+X_2})$, the adjoint $\phi, \psi:F_T(T(X_1)+T(X_2))\rightrightarrows \mu_{X_1+X_2}$ to $\iota$ and $\gamma$ are given by
	\[ \phi=\mu_{X_1+X_2}\circ T(\iota) \qquad \psi=T(\nabla_{(\iota_1\circ \xi_1 \iota_2\circ \xi_2)})
	\]
	Let $e:\mu_{X_1+X_2}\rightarrow \xi$ be their coequalizer, where $\xi:T(X)\rightarrow X$. Notice that, by naturality, since $e\circ \phi =e\circ \psi$, then $U_T(e)\circ \iota=U_T(e)\circ \gamma $, in particular, precomposing with the injections $T(X_i)\rightarrow T(X_1)+T(X_2)$ this implies that
	\[e\circ \eta_{X_1+X_2}\circ \iota_i\circ \xi_i= e \circ T(\iota_i)\]. Take  $h_i:X_i\rightarrow X$ to be the composition
	\[\xymatrix@C=30pt{X_i\ar[r]^-{\iota_i}&{X_1+X_2}\ar[r]^-{\eta_{X_1+X_2}}& T(X_1+X_2)\ar[r]^-{e} & X}
	\] 
	the previous equality shows that, for $i=1,2$, we have a commutative diagram
	
	\[\xymatrix@C=40pt{
		T(X_i)\ar[r]^-{T(\iota_i)}\ar[dd]_-{\xi_i }& T(X_1+X_2)\ar[r]^-{T(\eta_{X_1+X_2})}\ar[dr]_-{\id{T(X_1+X_2)}} & T(T(X_1+X_2))\ar[r]^-{T(e)}\ar[d]^-{\mu_{T(X_1)+T(X_2)}} & T(X)\ar[dd]^-{\xi} \\
		& &T(X_1+X_2)\ar[dr]^-{e}&\\
		X_i\ar[r]_{\iota_i} &X_1+X_2\ar[r]_-{\eta_{X_1+X_2}} & T(X_1+X_2)\ar[r]_-{e}&X}
	\]
	and so $h_1$ and $h_2$ are morphisms of $\eim{T}$. We claim that $(\xi, h_1, h_2)$ is a coproduct for $\xi_1$ and $\xi_2$. Let $\alpha:T(Y)\rightarrow Y$ be an algebra and $f_i:\xi_i\rightarrow \alpha$, two morphisms into it, we can take the arrow $\alpha\circ T(\nabla_{(f_1,f_2)}):\mu_{X_1+X_2}\rightarrow \alpha$. Notice that, in $\Set$, we have
	\begin{align*}&\nabla_{(f_1,f_2)}\circ \nabla_{(\iota_1\circ \xi_1, \iota_2\circ \xi_2)}\circ j_i= 	\nabla_{(f_1,f_2)}\circ \iota_i\circ \xi_i=f_i\circ \xi_i\\=&\alpha \circ T(f_i)=\alpha \circ T(	\nabla_{(f_1,f_2)}) \circ T(\iota_i)=\alpha \circ T(	\nabla_{(f_1,f_2)})\circ \iota \circ j_i
	\end{align*}
	\iffalse 
	\begin{align*}
	\nabla_{(f_1,f_2)}\circ \nabla_{(\iota_1\circ \xi_1, \iota_2\circ \xi_2)}\circ j_i&= 	\nabla_{(f_1,f_2)}\circ \iota_i\circ \xi_i=f_i\circ \xi_i=\alpha \circ T(f_i)=\alpha \circ \nabla_{(T(f_1), T(f_2))}\circ j_i\\
	\alpha \circ T(	\nabla_{(f_1,f_2)})\circ \iota \circ j_i&=\alpha \circ T(	\nabla_{(f_1,f_2)}) \circ T(\iota_i)=\alpha \circ T(f_i)=\alpha \circ \nabla_{(T(f_1), T(f_2))}\circ j_i
	\end{align*} 
\fi 
	where $j_i$ is the inclusion $T(X_i)\rightarrow T(X_1)+T(X_2)$. This entails that 
	\[	\nabla_{(f_1,f_2)}\circ \nabla_{(\iota_1\circ, \xi_1 \iota_2\circ \xi_2)}=\alpha \circ T(	\nabla_{(f_1,f_2)})\circ \iota\]
Now, we can compute
	\begin{align*}&\alpha\circ T(\nabla_{(f_1,f_2)})\circ \phi =\alpha\circ T(\nabla_{(f_1,f_2)})\circ \mu_{X_1+X_2}\circ T(\iota)=\alpha \circ \mu_Y\circ T(T(\nabla_{(f_1,f_2)}))\circ T(\iota)\\=&\alpha \circ T(\alpha)\circ T(T(\nabla_{(f_1,f_2)}))\circ T(\iota)=\alpha \circ T(\alpha \circ T(	\nabla_{(f_1,f_2)})\circ \iota)\\=&\alpha \circ T(\nabla_{(f_1,f_2)})\circ T(\nabla_{(\iota_1\circ, \xi_1 \iota_2\circ \xi_2)})=\alpha \circ T(\nabla_{(f_1,f_2)})\circ \psi 
	\end{align*}
Thus there exists  a unique $f:\xi\rightarrow \alpha$ such that $f\circ e=\alpha\circ T(\nabla_{(f_1,f_2)})$. Now, for $i=1,2$:
\[f\circ h_i\circ \xi_i=f\circ e\circ T(\iota_i)=\alpha\circ  T(\nabla_{(f_1,f_2)})\circ T(\iota_i)=\alpha \circ T(f_i)=f_i\circ \xi_i \]
But $\xi_i\circ \eta_{X_i}=\id{X_i}$, thus, by Proposition \ref{composition}, $\xi_i$ is epi and $f\circ h_i=f_i$. We are left with uniqueness: let $l:\xi\rightarrow \alpha$ another arrow such that $l\circ h_i=f_i$, we have that:
\[l\circ e\circ \iota \circ j_i= l\circ e \circ T(\iota_i)= h_i\circ \xi_i=f_i\circ \xi_i=\nabla_{(f_1, f_2)}\circ \iota_i\circ\xi_i=\nabla_{(f_1, f_2)}\circ \nabla_{(\iota_1\circ \xi_1, \iota_2\circ \xi_2)}\circ j_i\]
and thus \[l\circ e\circ \iota=\nabla_{(f_1, f_2)}\circ \nabla_{(\iota_1\circ \xi_1, \iota_2\circ \xi_2)}\]
Therefore
\begin{align*}
&l\circ e=r\circ e \circ T(\nabla_{\iota_1, \iota_2})=l\circ e\circ T(\nabla_{(\iota_1\circ \xi_1, \iota_2\circ \xi_2)})\circ T(\nabla_{\eta_{X_1},\eta_{X_2}})\\=&l \circ e \circ \mu_{T(X_1)+T(X_2)}\circ T(\iota)\circ T(\nabla_{\eta_{X_1},\eta_{X_2}})\\=&l\circ \xi\circ T(e) \circ T(\iota)\circ T(\nabla_{\eta_{X_1},\eta_{X_2}})T(\iota)\circ T(\nabla_{\eta_{X_1},\eta_{X_2}})\\=&\alpha \circ T(l)\circ T(e) \circ T(\iota)\circ T(\nabla_{\eta_{X_1},\eta_{X_2}})\\=&
\alpha \circ T(\nabla_{(f_1, f_2)})\circ T(\nabla_{(\iota_1\circ \xi_1, \iota_2\circ \xi_2)})\circ T(\nabla_{\eta_{X_1},\eta_{X_2}})\\=&\alpha \circ T(\nabla_{(f_1, f_2)})\circ T(\nabla_{\iota_1, \iota_2})=\alpha \circ T(\nabla_{(f_1, f_2)})
&\qedhere 
\end{align*}
	     
	        
	\end{itemize}
\end{proof} 

\begin{remark} When $T$ is the monad associated to some algebraic theory (so that $\eim{T}$ is a group, a monoid, a semilattice, etc.\dots), then $E_{(f,g)}$ is just the smallest congruence relation containing the image of $\Delta_{(f,g)}$, i.e. an equivalence relation that is also closed under the operations of the algebraic theory (the sum in $\Ab$, $\Cmon$ and $\Csemi$, multiplication in $\Grp$, $\Mon$ and $\Semi$, finite suprema in $\SLatt$ and $\SLattnobottom$, etc.\dots).
\end{remark}

\begin{remark} As noted in Remark \ref{split_set2}, the existence of $s$ relies to the axiom of choice. However, if we restrict our attention to finitary monads, i.e. those associated to finitary algebraic theory (see Remark \ref{equational}), it is possible to give a proof of the existence of colimits which does not use the axiom of choice (see \cite{linton1}). 
\end{remark}
\marginpar{A me sembra che funzioni, comunque puoi dargli un occhio?}
Now, let $\phi:\xi_1\rightarrow \xi_2$ be a surjective morphism in $\eim{T}$ where $\xi_i:T(X_i)\rightarrow X_i$, and $K$ be
\[\{(x,y)\in X_1\times X_1\mid \phi(x)=\phi(y)\}\]
the kernel pair of $U_T(\phi)$. We know by Proposition \ref{limcolim} that there exists a unique $\kappa:T(K)\rightarrow K$ which is the kernel pair of $\phi$ once endowed with the two projections $k_1, k_2:\kappa \rightarrow \xi_1$. So
\[
\xymatrix@C=30pt{
	K\ar@<-.5ex>[r]_{k_2}\ar@<.5ex>[r]^{k_1}&X_1\ar[r]^{\phi}&X_2 
}
\]
is a coequalizer diagram in $\Set$ which is split by Example \ref{split_set}, therefore, by point (2) of Proposition \ref{limcolim}, $\xi_2$ is the unique algebra structure which makes $\phi$ a morphism of $\eim{T}$ and, moreover it is a coequalizer of $k_1$ and $k_2$. So we have just proven the following.
\begin{corollary}\label{monadic_surjective}
	In any monadic category every surjective morphism is the coequalizer of its kernel pair.
\end{corollary}


\section{A hierarchy of monomorphisms and epimorphisms}\label{MoreMono} \marginpar{Ho cambiato un po' l'introduzione.}

In this subsection we  introduce several classes of monomorphisms and epimorphisms. We have already noted that every isomorphism is both a monomorphism and an epimorphism and that the viceversa does not all (see Example \ref{pos})
\iffalse 
Let us see this for epimorphisms (the statement for monomorphism follows dually): given an isomorphism $\phi\colon X\to Y$ in some category $\X$, let $\alpha,\,\beta\colon Y\rightrightarrows Z$ be two morphisms such that $\alpha\circ\phi=\beta\circ \phi$, then
\[
\alpha=(\alpha\circ\phi)\circ\phi^{-1}=(\beta\circ \phi)\circ\phi^{-1}=\beta.
\] 
\fi 

In what follows we introduce some classes of morphisms that are intermediate between isomorphisms and monomorphisms on the one side and fall in between isomorphisms and epimorphisms on the other side:

\begin{definition}
Let $\X$ be a category. A morphism $\phi\colon X\to Y$ in $\X$ is said to be:
\begin{itemize}
\item a {\em section} if it has a left inverse, that is, there exists $\phi'$ such that $\phi'\circ \phi=\id X$;
\item a {\em regular monomorphism} if it is the equalizer of some pair of parallel morphisms in $\X$;
\item a {\em strong monomorphism} if it is monic %\footnote{\label{coequalizer implies monic}In fact, in categories with coequalizers it is not necessary to assume explicitly that strong and regular monomorphisms are, in particular, monomorphism (see Corollary \ref{ext_mono_implies_mono_coro}).}
and, given any commutative square as follows
\begin{equation}\label{diagonal}
\xymatrix@C=50pt{
X'\ar[d]_-{\phi'}\ar[r]^f&X\ar[d]^{\phi}\\
Y'\ar@{.>}[ur]^{\exists!d}\ar[r]_g&Y}
\end{equation}
with $\phi'\colon X'\to Y'$ an epimorphism, there exists a unique diagonal $d\colon Y'\to X$ making the diagram commutative;
\item an {\em extremal monomorphism} if it is a monic %$^{\text{\ref{coequalizer implies monic}}}$ 
and, whenever ${\phi = \phi' \circ  e}$ holds, with $e$ an epimorphism, $e$ is an isomorphism. 
\end{itemize} 
Dually, a morphisms $\psi\colon X\to Y$ in $\X$ is said to be:
\begin{itemize}
\item a {\em retraction} if it has a right inverse, that is, there exists $\psi'$ such that $\psi\circ \psi'=\id {Y'}$;
\item a {\em regular epimorphism} if it is the coequalizer of some pair of parallel morphisms in $\X$;
\item a {\em strong epimorphism} if it is epic and, given any commutative square as follows
\[
\xymatrix@C=50pt{
X\ar[d]_-{\psi}\ar[r]^f&X'\ar[d]^{\psi'}\\
Y\ar@{.>}[ur]^{\exists!d}\ar[r]_g&Y'}
\]
with $\psi'\colon X'\to Y'$ a monomorphism, there exists a unique diagonal $d\colon Y\to X'$ making the diagram commutative;
\item an {\em extremal epimorphism} if it is an epimorphism and, whenever ${\psi = m\circ \psi'}$ holds, with $m$ a monomorphism,  $m$ is an isomorphism.
\end{itemize} 
\end{definition}

\begin{remark}
In the definition of strong and extremal monomorphisms it is necessary to assume the monicity.

\begin{itemize} 
	\item Take the category $\X$ which has natural numbers has objects, hom-sets are \marginpar{Ho promosso la nota a piè di pagina ad un vero remark inserendo degli esempi espliciti  e spezzato il remark successivo} given by \[	\X(n,m):=\begin{cases}
	\{\id{n}\} & n=m\\
	\{s_n, t_n\} &m=n+1\\
	\{f_{n,m}\} &m>n+1\\
	\emptyset &n<m
	\end{cases}
	\] and composition is defined putting 
	\begin{gather*}	
	f_{n,m}\circ \id{n}=f_{n,m} \quad \id{m}\circ f_{n,m}=f_{n,m} \quad 
	s_{n+1}\circ s_n
	= f_{n, n+2}  \quad	t_{n+1}\circ t_n= f_{n, n+2} \\
	s_{n}\circ \id{n}=s_{n} \quad \id{n+1}\circ s_{n}=s_{n}  \quad  	t_{n+1}\circ s_n= f_{n, n+2}\quad s_{n+1}\circ t_n= f_{n, n+2}\\
	t_{n}\circ \id{n}=t_{n} \quad \id{n+1}\circ t_{n}=t_{n}  \quad 		 f_{m,k}\circ f_{n,m}=f_{n,k}
	\end{gather*}
	This is the category generated by the commutative diagram
	\[
	\xymatrix@C=30pt{
0\ar@/_1.6pc/[rr]^{f_{0,2}}\ar@<-.5ex>[r]_{s_0}\ar@<.5ex>[r]^{t_0}\ar@/^2.5pc/[rrr]^{f_{0,3}} & 1\ar@<-.5ex>[r]_{s_1}\ar@<.5ex>[r]^{t_1}\ar@/^1.6pc/[rr]_{f_{1,3}} & 2\ar@<-.5ex>[r]_{s_2}\ar@<.5ex>[r]^{t_2} &\dots  	
}
	\]
	
	Notice that in $\X$ both the monos and epis coincide with the identities, so for every $\phi:n\rightarrow m$ and any square 
	\[
	\xymatrix@C=30pt{
		k\ar[d]_-{\id{k}}\ar[r]^f&n\ar[d]^{\phi}\\
		k\ar[r]_g&m}
	\]
	there is a unique diagonal filler given by $f$ even if $\phi$ is not monic.
	\item Take the category $\X$ generated by the following diagram.
	\[
	\xymatrix@C=30pt{A\ar@<-.5ex>[r]_g  \ar@<.5ex>[r]^f&B\ar[r]^\phi &C	
}
	\]
	Explicitly $\X$ has three objects $\{A,B,C\}$ and
	\[
	\X(i,j)=\begin{cases}
	\{\id{i}\} &i=j\\
	\{f, g\} &i=A, j=B\\
	\{\phi\} & i=B, j=C\\
	\{h\} &i=A, j=C\\
	\emptyset & \text{otherwise}
	\end{cases} 
	\]
Composition is given by
\begin{gather*}
\phi \circ \id{1}=\phi \quad \id{2}\circ \phi = \phi\quad
f \circ \id{0}=f \quad \id{1}\circ f = f\quad
g \circ \id{0}=g \quad \id{1}\circ g = g \\
h\circ \id{0}=h \quad \id{2}\circ h= h \quad 
\phi \circ f=h \quad \phi \circ g= h 
\end{gather*}
Now, $\phi$ is not a mono but the only way to write it as a composition $\phi'\circ e$ with $e$ epic is to take $\phi'=\phi$ and $e=\id{1}$.
\end{itemize}

In both cases the extra hypothesis of monicity can be dropped if $\X$ has coequalizers (see Corollary \ref{ext_mono_implies_mono_coro}). The dual observation holds for strong and extremal epimorphisms and for equalizers.
\end{remark}

\begin{remark}\label{mono:vs:unique:diagonal_1} 
In principle, for a given monomorphism $\phi$ to be strong, there are two things to check, for every commutative square like \eqref{diagonal}: the existence and the uniqueness of the diagonal $d$. But, in fact, it is enough to just check existence as, if two diagonals, say $d$ and $d'$, both make the diagram commute, then $\phi\circ d=g=\phi\circ d'$ implies $d=d'$, since $\phi$ is monic. Moreover, it is necessary to check only the commutativity of the lower triangle: suppose $\phi \circ d=g$, then we have
\[\phi \circ d \circ\phi'= g\circ \phi'= \phi \circ f\]
and thus $d\circ \phi' = f$ follows again from the hypothesis that $\phi$ is monic.
\end{remark}
\begin{remark} \label{mono:vs:unique:diagonal_2} 
 A morphism $\phi: X \to X'$ which is both epic and a section is an isomorphism. Indeed, assume that $\phi$ has a left inverse $\phi': X' \to X$, that is, $\phi' \circ \phi=\id {X}$. Then  $\phi\circ \phi' \circ \phi=\phi=\id {X'}\circ \phi$ and, since $\phi$ is epic, we deduce that $\phi\circ \phi'=\id {X'}$, that is, $\phi'$ is a two-sided inverse of $\phi$ which is, therefore, an isomorphism. Dually if $\phi$ is monic and a retraction then is an isomorphism.
\end{remark}



In the next lemma we collect all four stronger versions of monomorphism with the connecting implications:  


\begin{proposition}\label{quanti:tipi:mono:ci sono}
The following inclusions hold true in any category $\X$:
\[
\left\{ \begin{matrix}\text{sections}\\ \ \end{matrix}\right\}\ \overset{(1)}\subseteq\ \left\{ \begin{matrix}\text{regular}\\\text{monos}\end{matrix}\right\}\ \overset{(2)}\subseteq\ \left\{ \begin{matrix}\text{strong}\\\text{monos}\end{matrix}\right\}\ \overset{(3)}\subseteq\ \left\{ \begin{matrix}\text{extremal}\\\text{monos}\end{matrix}\right\}\ \overset{(4)}\subseteq\ \left\{ \begin{matrix}\text{monos}\\ \ \end{matrix}\right\}.
\]
Furthermore, analogous inclusions hold for the different types of epimorphisms.
\end{proposition}
\begin{proof}
(1). Let $\phi\colon X\to Y$ be a section, that is, suppose that there exists a map $\phi'\colon Y\to X$ such that $\phi'\circ \phi=\id X$. Then, $\phi$ is the equalizer of the pair $\id Y,\, \phi\circ\phi'\colon Y\rightrightarrows Y$. Indeed, suppose that $\psi\colon Z\to Y$ is a morphism that equalizes $\id Y$ and $\phi\circ\phi'$, that is, $\psi=\id Y\circ\psi=\phi\circ\phi'\circ \psi$. Then, $\phi'\circ \psi\colon Z\to X$ is the unique map such that $\phi\circ(\phi'\circ \psi)=\psi$.

\smallskip\noindent
(2). Suppose that $\phi\colon X\to Y$ is the equalizer of $\alpha$ and $\beta\colon Y\to Z$, by Proposition \ref{reg_mono} $\phi$ is monic. Consider the following commutative diagram in $\X$
\[
\xymatrix@C=50pt{
X'\ar[d]_-{\phi'}\ar[r]^f&X\ar[d]^{\phi}\\
Y'\ar@{.>}[ur]^{\exists!d}\ar[r]_g&Y\ar@<-.5ex>[r]_-{\alpha} \ar@<.5ex>[r]^-{\beta}&Z}
\]
where $\phi'\colon X'\to Y'$ is supposed to be an epimorphism. Since $\phi$ equalizes $\alpha$ and $\beta$, we have that $\beta\circ\phi\circ f=\alpha\circ\phi\circ f$ and so, by the commutativity of the diagram, we deduce that $\beta\circ g\circ \phi'=\alpha\circ g\circ\phi'$. Now, as $\phi'$ is epic, this last equality implies that $\beta\circ g=\alpha\circ g$, that is, $g$ equalizes $\alpha$ and $\beta$. By the universal property of equalizers, there exists a unique morphism $d\colon Y'\to X$ such that $g=\phi\circ d$, now the thesis follows from Remark \ref{mono:vs:unique:diagonal_1}.

\smallskip\noindent
(3). Let $\phi\colon X\to Y$ be a strong monomorphism, by  definition $\phi$ is a monomorphism. Suppose now that $\phi = \phi' \circ  e$, where $e\colon X\to X'$ is an epimorphism, and consider the following commutative square:
\[
\xymatrix@C=50pt{
X\ar@{>}[r]^{\id X}\ar[d]_-{e}&X\ar[d]^-\phi\\
X'\ar[r]_{\phi'}\ar@{.>}[ur]^{\exists!d}&Y.}
\] 
Since $\phi$ is strong, there exists a unique morphism $d\colon X'\to X$, making the above diagram commutative so, in particular, $e$ has a left inverse, that is, $d\circ e=\id {X}$, i.e., $e$ is section.  According to  Remark \ref{mono:vs:unique:diagonal_2}, 
a morphism which is both epic and a section is an isomorphism, so $e$ is  an isomorphism.


\smallskip\noindent
(4) By definition.
\end{proof} 


\begin{proposition}\marginpar{Ho spezzato il corollario in due}
Let $\X$ be a category with coequalizers and $\gamma\colon Z\to T$ a morphism in $\X$. If the following property holds: 
\begin{itemize}
\item[(*)] for each factorization ${\gamma = \gamma' \circ  e}$, with $e$ an epimorphism, $e$ is also monic;
\end{itemize}
then $\gamma$ is a monomorphism.  Dual observations hold for epimorphisms
\end{proposition}
\begin{proof}
Consider a pair of parallel arrows $f_1,\,f_2\colon Z'\rightrightarrows Z$, such that $\gamma\circ f_1=\gamma\circ f_2$. Denote by $e\colon X\to C$ the coequalizer of $f_1$ and $f_2$, so to get the following commutative diagram
\[
\xymatrix@C=40pt{
	Z'\ar@<-.5ex>[r]_-{f_2} \ar@<.5ex>[r]^-{f_1}&X\ar[dr]_{e}\ar[rr]^-\gamma&&T\\
	&&C\ar@{.>}[ur]_-{\exists!\, \gamma'}
}
\]
where $\gamma'$ is the unique morphism such that $\gamma=\gamma'\circ e$. Since $e$ is an epimorphism by Proposition~\ref{quanti:tipi:mono:ci sono}, it is also a monomorphism by condition (*). Since $e$ is the coequalizer of $f_1$ and $f_2$, it is clear that $e\circ f_1=e\circ f_2$, which implies $f_1=f_2$ since $e$ is monic.
\end{proof}

\begin{corollary}\label{ext_mono_implies_mono_coro}
	Let $\X$ be a category with coequalizers, then  in the definitions of strong and extremal monomorphisms there is no need to explicitly assume that the morphism is monic. The dual holds for epimorphisms.
\end{corollary}
\begin{proof}
Suppose that $\phi\colon X\to Y$ is a morphism for which each commutative square like \eqref{diagonal} has a unique diagonal. If we can write a factorization $\phi=\phi'\circ e$, with $e\colon X\to Y'$ an epimorphism, then one can consider the following commutative square:
\[
\xymatrix@C=50pt{
	X\ar[d]_-{e}\ar@{>}^{\id{X}}[r]&X\ar[d]^{\phi}\\
	Y'\ar@{.>}[ur]^{\exists!d}\ar[r]_{\phi'}&Y}
\]
The existence of the diagonal $d$ such that $d\circ e=\id X$, shows that $e$ is a section so, in particular, it is monic. Hence, $\phi$ satisfies (*) and, therefore, it is monic, showing that, in this setting, there is no need to assume that $\phi$ is monic in the definition of strong monomorphism.

Next, suppose that $\phi$ is such that, for each factorization ${\phi = \phi' \circ  e}$, with $e$ epic, $e$ is an isomorphism, then $\phi$ trivially satisfies (*), so $\phi$ is monic. Hence, in this setting, also in the definition of extremal monomorphism it is redundant to assume that $\phi$ is monic.
\end{proof}
    
    The next two lemmas collect some properties of extremal and strong monomorphisms (and, dually, epimorphisms), respectively.

\begin{lemma}\label{extremal_prop_lemma}
Let $\X$ be a category. Then, the following statements hold true:
\begin{enumerate}[\rm (1)]
\item a morphism which is both epic and an extremal monomorphism is an isomorphism;
\item if a composition $\psi\circ \phi$ is an extremal monomorphism, then so is $\phi$;
\item the following conditions are equivalent:
\begin{enumerate}[\rm ({3.}1)]
\item every monomorphism in $\X$ is extremal;
\item every morphism in $\X$ which is both epic and monic is an isomorphism.
\end{enumerate}
\end{enumerate}
Dual statements hold for the extremal epimorphisms. 
\end{lemma}
\begin{proof}
(1). Let $\phi\colon X\to Y$ be an epimorphism in $\X$ which is also an extremal monomorphism. In particular, we can write $\phi=\id Y\circ \phi$ with $\phi$ epic, and deduce by the definition of extremal monomorphism that $\phi$ has to be an isomorphism.

\smallskip\noindent
(2). Consider a factorization $\phi=\phi'\circ e$ with $e$ an epimorphism. This induces a factorization $\psi\circ \phi=(\psi\circ\phi')\circ e$ and so, since $\psi\circ \phi$ is supposed to be an extremal monomorphism, $e$ is an isomorphism.

\smallskip\noindent
(3). Suppose first that every monomorphism in $\X$ is extremal, and consider a morphism $\phi\colon X\to Y$ which is both epic and monic. Then, $\phi$ is both an extremal monomorphism and an epimorphisms so, by part (1), it is an isomorphism. 

On the other hand, suppose that all the morphisms in $\X$ that happen to be both epic and monic are, in fact, isomorphisms, and consider a monomorphism $\phi\colon X\to Y$. Suppose now that $\phi=\phi'\circ e$, with $e$ an epimorphism, by Proposition \ref{composition} $e$ is also a monomorphism, and thus, under our hypotheses, an isomorphism, showing that $\phi$ is extremal.
\end{proof}

\begin{example}\label{composition:of:extremal:monomorphisms} \marginpar{Ho inserito l'esempio esplicito}
	The  composition of two extremal monomorphism need not be an extremal monomorphism as shown by this example due to Arduini (see \cite[Appendix]{Arduini}). Take the category generated  by the following commutative diagram
	\[\xymatrix@C=30pt{
	A\ar@{>}[r]^f\ar@{>}[d]_e &B\ar@<-.5ex>[r]_b\ar@<.5ex>[r]^a \ar@{>}[d]_g &F\\
	D\ar@{>}[r]_m &C\ar@<-.5ex>[r]_c\ar@<.5ex>[r]^h  &E
}\]
	In this category $f$ and $g$ are extremal monomorphisms and $e$ is an epimorphism, but not an isomorphism.
\end{example}


\begin{lemma}\label{strong_prop_lemma}
Let $\X$ be a category, and $\phi\colon X\to Y$ and $\psi\colon Y\to Z$ two morphisms in $\X$. Then, 
\begin{enumerate}[\rm (1)]
\item if $\phi$ and  $\psi$ are both strong monomorphisms, then $\psi\circ \phi$  is a strong monomorphism;
\item if $\psi\circ \phi$ is a strong monomorphism, then $\phi$ is a strong monomorphism.
\end{enumerate}
Dual statements hold for the strong epimorphisms. 
\end{lemma}
\begin{proof}
(1). Suppose  that $\phi$ and  $\psi$ are both strong monomorphisms, and take a commutative diagram like the following:
\[
\xymatrix@C=60pt{
X'\ar[dd]_{e}\ar[r]^x&X\ar[d]^\phi\\
&Y\ar[d]^\psi\\
Z'\ar@{.>}[ruu]^d\ar[r]_z&Z,
}
\]
where $e$ is an epimorphism; we have to prove that there is a unique diagonal $d$ that makes the diagram commute. Indeed, consider the following commutative diagrams:
\[
\xymatrix@C=40pt@R=50pt{
X'\ar[d]_{e}\ar[r]^x&X\ar[r]^\phi&Y\ar[d]^\psi\\
Z'\ar@{.>}[rru]^{\exists!d'}\ar[rr]_z&&Z,
}
\qquad\text{and}\qquad
\xymatrix@R=50pt@C=60pt{
X'\ar[d]_{e}\ar[r]^x&X\ar[d]^\phi\\
Z'\ar@{.>}[ru]^{\exists!d}\ar[r]_{d'}&Y,
}
\]
where, in the left-most square, we use that $\psi$ is a strong monomorphism (and that $e$ is epic) to find the unique diagonal $d'\colon Z'\to Y$ such that  $d'\circ e=\phi\circ x$ and $\psi\circ d'=z$. Similarly, in the square on the right-hand-side we can find a unique diagonal $d\colon Z'\to X$ such that $d\circ e=x$ and $\phi\circ d=d'$. It is now easy to check that this $d$ is also the unique diagonal for our original square, concluding the proof.

\smallskip\noindent
(2). Suppose  $\psi\circ\phi$ is an extremal monomorphism, take the following commutative square
\begin{equation}\label{Com:diag1}
\xymatrix@C=50pt{
X'\ar[r]^x\ar[d]_e&X\ar[d]^\phi\\
Y'\ar[r]_y\ar@{.>}[ur]^d&Y,
}
\end{equation}
where $e$ is epic, and let us prove that it admits a unique diagonal $d\colon Y'\to X$. First we note that  $\phi$ is a monomorphism (by Proposition \ref{composition}) and, therefore, we only need to check the existence of the diagonal $d$, in view of Remark~\ref{mono:vs:unique:diagonal_1}.  Consider the following commutative diagram:
\[
\xymatrix@C=70pt{
X'\ar[r]^x\ar[d]_e&X\ar[d]^\phi\\
Y'\ar@{>}[d]_{\id{Y}}&Y\ar[d]^\psi\\
Y'\ar@{.>}[uur]^{\exists!d}\ar[r]_{\psi\circ y}&Z,
}
\]
and use that $\psi\circ \phi$ is an extremal epimorphism to show that there exists a unique diagonal $d\colon Y'\to X$ such that $d\circ e=x$ and $\psi\circ\phi\circ d=\psi\circ y$. Composing the first equality with $\phi$ and exploiting the commutativity of (\ref{Com:diag1}) we get  \[\phi \circ d\circ e=\phi \circ x =y \circ e\]. Since $e$ is epic this entails $\phi \circ d=y$, i.e., $d$ is the desired diagonal.
\end{proof}


\subsection{Examples}

In this subsection we describe some explicit examples to illustrate some of the results and constructions introduced until now. We start with the category $\Top$ of topological spaces and continuous maps:


\begin{example}\label{exa:monos1} 
 In $\Top$ the extremal monomorphisms are exactly the  embeddings (i.e., the open, continuous and injective maps) while the monomorphisms are the continuous injections. This shows that in $\Top$ the inclusion (4) in Proposition~\ref{quanti:tipi:mono:ci sono} may be proper. Moreover, still in $\Top$, also the  inclusion (1) in the same proposition may be proper (witnessed by the classical example of a subspace embedding that is not a section, e.g. $\{0,1\} \hookrightarrow [0,1]$). 
 \end{example}

To produce an example showing that the inclusion (2) in Proposition~\ref{quanti:tipi:mono:ci sono}  may be strict one can use a category $\X$, where two composable regular monos %$m\colon A \to B$ and $n\colon B \to C$ 
have a non-regular composition:% $k = n \circ m \colon A \to C$
since regular monos are strong and the class of strong monomorphism is closed under composition by Lemma~\ref{strong_prop_lemma}, the resulting morphism is a strong monomorphisms that is not regular. An example of such an $\X$ can be found in \cite[Exercise 7J(a)]{cats}: namely the full subcategory $\FHaus$ of $\Top$ of all {\em functionally Hausdorff topological spaces}. These are the topological spaces $X$, where every pair of distinct points $x,\, y \in X$ can be separated by a continuous real-valued function $X\to \R$, i.e. there exists $f:X\to \R$ continuous such that $f(x)\neq f(y)$. For example, all Tichonov spaces are functionally Hausdorff.  Let us give some details on the specific example:  

\begin{example} \label{ex_FHAus}
Let $(X,\tau)$ be an object of $\FHaus$,  the {\em weak topology} $\tau_w$ on $X$ is the initial topology with respect to the continuous functions $f\colon (X,\tau) \to \R$. Then, $(X,\tau_w)$ is a Tichonov space, whereas the original space $(X,\tau)$ need not be Tichonov. It can be proved (see \cite[Exercise 6.B]{DT}), that the 
regular monomorphisms in $\FHaus$ are (up to isomorphism) the subspace embeddings $M \hookrightarrow (X,\tau) \in \FHaus$ such that $M$ is closed in $(X, \tau_w)$.

Consider now the following three functionally Hausdorff spaces: 
\[
A:=\left\{\frac{1}{n}: n\in \N_+\right\},\qquad B:=\{0\} \cup A\qquad \text{and}\qquad C:=(\R,\tau),
\] 
where $\tau$ is not the usual topology of $\R$, but a finer topology obtained by adding the closed set $A$ to the usual closed sets in $\R$. Notice that in this case $\tau_w$ is the usual euclidean topology on $\R$. The subspace topologies of $A$ and $B$ are discrete, so $A \hookrightarrow B$ is regular. Similarly, $B \hookrightarrow C$ is regular, since $B$ is closed in both $\tau$ and $\tau_w$. Nevertheless, $A \to C$ is not regular since and $A$ is not closed in the euclidean topology.
\end{example}

 %\ \ \maltese\ \footnote{This problem  still remains semi-open as far as we do not provide an an explicit example of a category $\X$ as above (my feeling is that in the example $\X$ of (c) (functionally Hausdorff spaces), extremal monos are also strong [to be checked], so we need here a separate example.):  \\
%\noindent{\bf Question/Problem.} Find an explicit example of a (necessarily not finitely complete) category where the inclusion in Proposition~\ref{quanti:tipi:mono:ci sono}(3) is proper).}

To provide an example of a category where the inclusion (3) in Proposition~\ref{quanti:tipi:mono:ci sono} is proper, one can use the idea (exploiting compositions) used in Example \ref{ex_FHAus}. Indeed,
suppose that $\X$ is a category where the class of extremal monomorphisms is not stable under composition. Since  the class of strong monomorphisms in $\X$ is closed under composition by Lemma~\ref{strong_prop_lemma} and it is contained in the class of  extremal monomorphisms in $\X$, then, for a pair of extremal monomorphisms $\phi$ and $\psi$,  whose composition $\psi\circ\phi$ is not an extremal monomorphism, either $\phi$ or $\psi$ (or both) is not a strong monomorphism.
\begin{example} Let $\X$ be the category of Example \marginpar{Aggiunto esempio esplicito} \ref{composition:of:extremal:monomorphisms}, in it $g$ is an extremal monomorphism and $e$ is an epimorphism but there is no diagonal for the square
	\[
	\xymatrix@C=30pt{
		A\ar@{>}[d]_{e}\ar@{>}[r]^f&B\ar@{>}[d]^{g}\\
		D\ar@{>}[r]_m&C}\]	
\end{example} 

 
%An example of a purely algebraic flavour with the same property **can be found** in \cite{cats} (see also \cite[Exercise 6.C(c)]{DT}); namely
% the category {\bf Sgr} of semigroups, where $S\in {\bf Sgr}$ has two subsemigroups $A \subseteq B$, such that the 
%inclusions $m: A \to B$ and $n: B \to S$ are regular monos, while their composition $k: A \to S$ is not regular. \\
%**{\tt A me quell'esempio non mi torna, ma forse voi riuscite a fare i conti meglio di me :-) \\
%Nel \cite{cats} non lo trovo, siccome la pagine riferita \cite[Exercise 6.C(c)]{DT} è all'edizione del \cite{cats} del 1990.}
%%\item Let $(P,\leq)$ be a poset viewed as a category. Then, any morphism in $P$ is both an epimorphism and a monomorphism (this is trivially true as in a posetal category any two parallel arrows are equal) but the unique isomorphisms in $P$ are the identities (because of the antisymmetry of the partial order relation). Now, consider the following Hasse diagram:
%\[
%\xymatrix{
%&\bullet\\
%\bullet\ar[ur]^{b}&&\bullet\ar[ul]_{m}\\
%&\bullet\ar[ur]_{a}\ar[ul]^{e}}
%\]
%In the associated posetal category, $e$ is an epimorphism which is not strong (as $m$ is a monomorphism but the above square cannot be completed with a diagonal, as there is no morphism going from the target of $e$ to the origin of $m$). Finally, note that $e$ is extremal, because the unique factorizations $e=f\circ g$...





\begin{example}
In any Abelian category, as well as in the  category of groups $\Grp$, all the inclusions (2--4) in Proposition~\ref{quanti:tipi:mono:ci sono} are equalities: see for instance \cite[Exercise 7H(a)]{cats} (or, \cite[Exercise 6.B]{DT}) for a direct proof of the fact that all monomorphisms in $\Grp$ are regular.
	This follows also from Schreier's Amalgamation Theorem \cite{Schr} (which, roughly speaking, says that for any subgroup $H$ of a group $G$ the equalizer of 
	the pair of canonical embeddings $\iota_1,\iota_2\colon G \rightrightarrows G*_HG$ in the amalgamated free product $G*_HG$ coincides with $H$). Even in these cases the inclusion (1) is, in general, proper. For example, the inclusion $\Z/2\Z\to \Z/4\Z$ in the category $\Ab$ is regular but not a section.
 \end{example}











\iffalse 

\section{Concrete categories}

\begin{definition}
A {\em concrete category} is a pair $(\X,|-|)$, where $|-|\colon \X\to \Set$ is a faithful functor, usually called the \emph{forgetful} (or the \emph{underlying set}) functor. 

In a concrete category $(\X,|-|)$, a morphism $\phi\colon X\to Y$ in $\X$ is said to be:
\begin{itemize}
\item {\em injective} if the underlying function $|\phi|\colon |X|\to |Y|$ is injective;
\item {\em surjective} if the underlying function $|\phi|\colon |X|\to |Y|$ is surjective.
\end{itemize}
\end{definition}

For a concrete category  $(\X,|-|\colon \X\to \Set)$, the faithfulness of the forgetful functor $|-|$ means that, given a pair of parallel morphisms $\phi,\, \psi\colon X\rightrightarrows Y$ in $\X$, the equality $|\phi|=|\psi|$ implies that $\phi=\psi$. In other words, in a concrete category two morphism coincide whenever they have the same underlying function.

\begin{lemma}\label{inj->mono_lemma}
Let $(\X,|-|\colon \X\to \Set)$ be a concrete category. Then, injective morphisms are monic and surjective morphisms are epic.
\end{lemma}
\begin{proof}
Let $\phi\colon X\to Y$ be injective and consider $\alpha,\,\beta\colon Y\rightrightarrows Z$ such that $\alpha\circ\phi=\beta\circ\phi$. Then, applying $|-|$, we obtain the following relation in $\Set$: $|\alpha|\circ|\phi|=|\beta|\circ|\phi|$. Since $|\phi|$ is injective, we deduce that $|\alpha|=|\beta|$ and, as forgetful functors are faithful, we see that $\alpha=\beta$.

The statement regarding epimorphisms can be proved similarly.
\end{proof}

The above lemma is a particular case of a more general statement: every faithful functor  $F\colon \X \to \Y$ reflects 
epimorphisms and monomorphisms (i.e., if $F\phi$ is epic in $\Y$, then $\phi$ is epic in $\X$, and similarly for monos).

\begin{corollary}\label{mono_is_injective2}
Let $(\X,|-|\colon \X\to \Set)$ be a concrete category and suppose that:
\begin{enumerate}[\rm (1)]
\item the category $\X$ is finitely bicomplete; 
\item the forgetful functor $|-|\colon \X\to \Set$ is left-exact (that is, it commutes with finite limits or, equivalently, with finite products and equalizers).
\end{enumerate}
Then, for every monomorphism $\phi\colon X\to Y$ in $\X$, the map $|\phi|\colon |X|\to |Y|$ is injective.
\end{corollary}
\begin{proof}
By Proposition \ref{mono=trivial_ker_prop}, $\phi$ is a monomorphism if, and only if, its kernel pair is the diagonal $\Delta_X\colon X\to X\times X$. Since the forgetful functor is supposed to commute with finite products, it is not difficult to verify that $|\Delta_X|=\Delta_{|X|}\colon |X|\to |X|\times |X|$. Furthermore, since the forgetful functor also commutes with equalizers, it has to send kernel pairs to kernel pairs, and so the kernel pair of $|\phi|$ is exactly $\Delta_{|X|}$, showing that $|\phi|$ is a monomorphism in $\Set$, that is, $|\phi|$ is injective.
\end{proof}

\begin{remark}\label{Div:vs:left:adjoint2} 
The relevance of item (2) in the above corollary is clear from the following example. Let $\mathfrak D iv$ denote the complete subcategory of $\Ab$ 
with objects all divisible groups. Then $\mathfrak D iv$ satisfies (1) (actually it is bicomplete), but the monomorphisms in $\mathfrak D iv$ 
need not be injective (e.g., $m: \Q \to \Q/\Z$ is a non-injective monomorphism), this means the (2) fails (indeed, the equalizer of $m$ and the zero morphism $0: \Q \to \Q/\Z$, 
i.e., the kernel of $m$, is the 0 subgroup of $\Q$, which differs from the equalizer formed in $\Set$ of $|m|$ and $|0|$). 
 \end{remark}


Note that, whenever the forgetful functor $|-|\colon \X\to \Set$ has a left adjoint, it automatically commutes with all limits and, in particular, it satisfies the hypotheses of the above corollary. Left adjoints to forgetful functors are usually called ``free functors''; the following is a list of examples of such adjunctions:
\begin{itemize}
\item the left adjoint to the forgetful functor $\Grp\to \Set$ associates to a set $S$ the free group of words in the alphabet $S\sqcup S^{-1}$, where $S^{-1}$ is the set of formal inverses $\{s^{-1}:s\in S\}$ of the elements in $S$;
\item the left adjoint to the forgetful functor $\Ab\to \Set$ associates to a set $S$ the free Abelian group $\Z^{(S)}$;
\item  the left adjoint to the forgetful functor $\SLatt\to \Set$ associates to a set $S$ the $0$-join-semilattice of finite subsets of $S$, partially ordered by inclusion.
\end{itemize}
As a consequence, in all the above categories monomorphisms are injective. A particularly nice concrete category is the category $\Top$ of all topological spaces and continuous maps. In fact, the forgetful functor $\Top\to \Set$ has both a left adjoint (that associates to any set $S$ the topological space $(S,\delta_S)$, where $\delta_S$ is the discrete topology) and a right adjoint (that associates to any set $S$ the topological space $(S,\tau_S)$, where $\tau_S$ is the indiscrete(=trivial) topology). Hence, the forgetful functor $\Top\to \Set$ is exact (=both left- and right-exact). This shows that, in $\Top$, monomorphisms are exactly the injective continuous maps, while epimorphisms are exactly the surjective continuous maps.

\begin{remark}
The reader may be tempted to consider, for a set $S$, the divisible group $\Q^{(S)}$ as a free object in $\mathfrak D iv$ associated to $S$, but easy examples show that the assignment $S \mapsto \Q^{(S)}$ fails to produce a functor, which witnesses again the failure of the proprety (2) in Corollary~\ref{mono_is_injective}, (see also Remark \ref{Div:vs:left:adjoint}).
\end{remark}


%\subsection{Lattices of subobjects and quotients}
%
%Let $\X$ be a category and $X$ an object in $\X$. Consider a pre-order relation on the class of all regular monomorphisms in $\X$ with target $X$: for $\phi_1\colon M_1\to X$ and $\phi_2\colon M_2\to X$ two regular monomorphisms, we let $\phi_1\leq \phi_2$ if, and only if, there exists a morphism $\psi\colon M_1\to M_2$ such that $\phi_1=\phi_2\circ \psi$. As for any pre-order, we can consider the following equivalence relation:
%\[
%(\phi_1\sim \phi_2)\qquad\text{if and only if}\qquad (\phi_1\leq \phi_2\quad\text{and}\quad \phi_2\leq \phi_1).
%\]
%
%\begin{definition}
%Let $\X$ be a category and $X$ an object in $\X$. The {\em partially ordered class $\L(X)$ of regular subobjects of $X$} is the partially ordered class obtained by taking the quotient of the preordered class of regular monomorphism with target $X$ in $\X$. \\
%The category $\X$ is said to be {\em regular well-powered} if $\L(X)$ is a set (as opposed to a proper class) for each $X$ in $\X$.
%\end{definition}
%
%A consequence of Corollary \ref{mono_is_injective} is the following:
%
%\begin{proposition}
%Let $(\X,|-|\colon \X\to \Set)$ be a concrete category satisfying the hypotheses of Corollary \ref{mono_is_injective}. Then $\X$ is regular well-powered.
%\end{proposition}
%\begin{proof}
%Consider an object $X$ in $\X$ and let us verify that $\L(X)$ is a set. Note first that $\L(|X|)$ (the class of regular subobjects of $|X|$ in $\Set$) is exactly the family of subsets of $|X|$ and, therefore, $\L(|X|)=2^{|X|}$ is a set. Furthermore, there is a map
%\[
%\L(X)\to 2^{|X|}\qquad\text{such that}\qquad [\, f\colon M\to X\, ]\mapsto [\, \Im(|f|)\subseteq |X|\, ].
%\] 
%It is enough to show that the above map is injective. For that, let  $\phi_1\colon M_1\to X$ and $\phi_2\colon M_2\to X$ be two regular monomorphisms such that $\Im(|\phi_1|)=\Im(|\phi_2|)$. \NB WE SHOULD INTRODUCE INTERSECTIONS BEFORE... SO WHAT FOLLOWS IS JUST A SKETCH, NOT TO FORGET THE IDEA.\NB
%
%Consider the equalizer $e\colon M_1\cap M_2\to M_1\times M_2$, of the two maps $\phi_1\circ\pi_1,\, \phi_2\circ\pi_2\colon M_1\times M_2\to X$. As the underlying functor is left exact, it sends intersections to intersection, $|\pi_1\circ e|$ and $|\pi_2\circ e|$ are isomorphisms. In particular, these 
%\end{proof}
\fi 
\section{Images, coimages and related factorizations}

\begin{definition}
Let $\X$ be a finitely bicomplete category, take   a morphism $\phi\colon X\to Y$ in $\X$, and let $\kappa\colon K\to X\times X$ the kernel pair and $\gamma\colon Y\sqcup Y\to C$ the cokernel pair of $\phi$ in $\X$:
\[
\xymatrix@C=35pt{
K\ar[r]^-\kappa& X\times X \ar@<-.5ex>[r]_-{\pi_1} \ar@<.5ex>[r]^-{\pi_2} &X\ar[r]^{\phi}&Y
}
\qquad \text{and}\qquad
\xymatrix@C=35pt{
X\ar[r]^-\phi& Y \ar@<-.5ex>[r]_-{\iota_1} \ar@<.5ex>[r]^-{\iota_2} &Y\sqcup Y\ar[r]^{\gamma}&C
}
\]
Let $\kappa_i:=\pi_i\circ \kappa$ and $\gamma_i:=\gamma\circ \iota_i$ (for $i=1,\,2$). Then, the {\em image} of $\phi$ is the equalizer $\Im(\phi):=\eq(\gamma_1,\gamma_2)$, while the {\em coimage} of $\phi$ is the coequalizer $\Coim(\phi):=\coeq(\kappa_1,\kappa_2)$. 
\end{definition}

All the different parts in the above definition can be visualized in the following picture:
\begin{equation}\label{BIG:diagram}
\xymatrix@C=40pt@R=15pt{
 && \Im(\phi) \ar[dr]^{m'}  & & \\ 
K \ar@<-.5ex>[r]_-{\kappa_1} \ar@<.5ex>[r]^-{\kappa_2} &X\ar@{.>}[ur]^{e'} \ar[rr]^(0.5){\phi}\ar[dr]_{e}&  & Y \ar@<-.5ex>[r]_-{\gamma_1} \ar@<.5ex>[r]^-{\gamma_2} & C\\
& & \Coim(\phi)\ar@{.>}[ru]_m  &  &
}
\end{equation}
where $m'\colon\Im(\phi)\to Y$ is the canonical regular monomorphism and $e\colon X\to \Coim(\phi)$ is the canonical regular epimorphism. In the following proposition we show that the above commutative diagram can be completed uniquely with the dotted arrows $m\colon \Coim(\phi)\to Y$ and $e'\colon X\to \Im(\phi)$:

\begin{proposition}
Let $\X$ be a finitely bicomplete category and $\phi\colon X\to Y$ in $\X$. Then there are two factorizations
\[
\xymatrix@C=40pt@R=15pt{
X \ar[rr]^\phi\ar[dr]_{e}&  & Y &&X \ar[rr]^\phi\ar[dr]_{e'}&  & Y \\ 
& \Coim(\phi) \ar[ur]_{m} &&&& \Im(\phi) \ar[ur]_{m'} & 
}
\]
where $e$ is the canonical regular epimorphism to the coimage, and $m'$ is the canonical regular monomorphism from the image. Furthermore, the morphisms $m$ and $m'$ are the unique ones with these properties.
\end{proposition}
\begin{proof}
We consider just the factorization through the coimage, as the other factorization follows by duality. By Proposition $\ref{kp_equ}$ we have a commutative diagram
\[
\xymatrix@C=35pt{
K\ar[r]^-\kappa& X\times X \ar@<-.5ex>[r]_-{\pi_1} \ar@<.5ex>[r]^-{\pi_2} &X\ar[r]^{\phi}&Y,
}
\]
thus $\phi\circ\kappa_1=\phi \circ \kappa_1$, where  $\kappa_i=\pi_i\circ\kappa$.
The coimage is the coequalizer of $\kappa_1$ and $\kappa_2$ so, by the universal property of coequalizers, there exists a unique morphism $m\colon \Coim(\phi)\to Y$ such that $\phi=m\circ e$.
\end{proof}

\begin{corollary}\label{epiepi}
Let $\X$ be a finitely bicomplete category, $\phi\colon X\to Y$ a morphism, and consider the factorization through the coimage:
\[
\xymatrix@R=20pt@C=50pt{
X\ar[rr]^{\phi}\ar[dr]_e&&Y\\
&\Coim(\phi)\ar[ur]_m
}
\] 
Then, $\phi$ is an epimorphism if and only if $m$ is an epimorphism. 
\end{corollary}
\begin{proof} 
This follows from Proposition \ref{composition}.
\end{proof}


Let  $\phi\colon X\to Y$ be a morphism and consider again the diagram \eqref{BIG:diagram}. We have described above two canonical ways to factorize $\phi$ as a composition of two maps: one through its image and one through its coimage. We can now construct a morphism $d\colon \Coim(\phi)\to \Im(\phi)$ that can be thought of as a comparison map between these two factorizations of $\phi$.
\begin{lemma}\label{coim_im}
Let $\X$ be a finitely bicomplete category and $\phi\colon X\to Y$ a morphism in $\X$. Then, there is a unique  $d\colon \Coim(\phi)\to \Im(\phi)$ that makes the following diagram commute:
\[
\xymatrix@C=40pt@R=15pt{
 && \Im(\phi) \ar[dr]^{m'}  & & \\ 
K \ar@<-.5ex>[r]_-{\kappa_1} \ar@<.5ex>[r]^-{\kappa_2} &X \ar[ru]^{e'} \ar[rr]^(0.6){\phi}\ar[dr]_{e}&  & Y \ar@<-.5ex>[r]_-{\gamma_1} \ar@<.5ex>[r]^-{\gamma_2} & C.\\
& & \Coim(\phi) \ar[ur]_{m} \ar@{.>}[uu]^(0.3){d}  &  &
}
\]
\end{lemma}
\begin{proof}
By construction,
\[(\gamma_1\circ m)\circ e=\gamma_1\circ \phi =\gamma \circ (\iota_1\circ \phi) =\gamma \circ (\iota_2\circ \phi)=\gamma_2\circ \phi =(\gamma_2\circ m)\circ e\]
and $e$ is (regular) epi, so we can conclude that $\gamma_1\circ m=\gamma_2\circ m$.  $m'\colon \Im(\phi)\to Y$ is the equalizer of $\gamma_1$ and $\gamma_2$, thus there exists a unique morphism $d$ such that $m=m'\circ d$. On the other hand 
\[m'\circ e'=\phi=m\circ e=m'\circ d\circ e\]
and $m'$ is (regular) mono, therefore $e'=d\circ e$.
\end{proof}

As a concluding general result on these factorizations through the image and coimage, we prove the following lemma, that characterizes them via a universal property:

\begin{proposition}\label{univ_prop_facts_lemma}\marginpar{Cambiati i diagrammi}
Let $\X$ be a finitely bicomplete category, $\phi\colon X\to Y$ a morphism in $\X$ and suppose that $\phi=\beta\circ \alpha$ and $\phi=\beta'\circ \alpha'$ with $\alpha$ is a regular epimorphism and $\beta'$ a regular monomorphism. 
Then there exist unique morphisms $u\colon I\to \Coim(\phi)$ and $u'\colon \Im(\phi)\to I'$, such that $e=u\circ \alpha$ and $m'=\beta'\circ u'$.
\[
\xymatrix@C=40pt@R=20pt{
X \ar[rr]^\phi\ar[dr]_{\alpha} \ar@/_10pt/[ddr]_{e}&  & Y &&X\ar@/_10pt/[ddr]_{e} \ar[rr]^\phi\ar[dr]_{\alpha'}&  & Y \\ 
& I \ar[ur]_{\beta} \ar@{.>}[d]^-{\exists!u} &&&& I' \ar[ur]_{\beta'} & \\&\Coim(\phi) \ar@/_10pt/[uur]_{m}&&&& \Im(\phi)\ar@/_10pt/[uur]_{m'}\ar@{.>}[u]^-{\exists!u'}
}
\]
\end{proposition}
\begin{remark}\label{univ_prop_facts_lemma_2} Notice that, given $u$ and $u'$ as in the thesis, we also have $m\circ u=\beta$ and $u'\circ e=\alpha'$. This follows since
	\begin{align*}
		(m\circ u)\circ \alpha&=m\circ e=\phi =\beta \circ \alpha \\
		\beta'\circ (u'\circ e)&=\beta'\circ \alpha'=\phi =\beta'\circ \alpha'
	\end{align*}
	and by the hypothesis that $\alpha$ is a (regular) epi $\beta'$ a (regular) mono.
\end{remark}
\begin{proof}
We consider the second factorization only, as the proof for the first one is dual. By assumption, we know that $\beta'$ is a regular monomorphism, that is, it is the equalizer of a pair of maps $d_1,\, d_2\colon Y\rightrightarrows D$. In particular, $d_1\circ\beta'=d_2\circ\beta'$, and so 
\[
d_1\circ\phi=d_1\circ\beta'\circ\alpha'=d_2\circ\beta'\circ\alpha'=d_2\circ\phi.
\] 
Let $\gamma\colon Y\sqcup Y\to C$ be the coequalizer of $\iota_1\circ\phi,\, \iota_2\circ\phi\colon X\to Y\sqcup Y$, where $\iota_1,\,\iota_2\colon Y\to Y\sqcup Y$ are the canonical morphisms of the coproduct. The above equalities imply that
\[\nabla_{(d_1,d_2)}\circ (\iota_1\circ \phi)=\nabla_{(d_1,d_2)}\circ (\iota_2\circ \phi)\] so there exists a unique $v\colon C\to D$ such that $v\circ\gamma=\nabla_{(d_1,d_2)}$:
\[
\xymatrix@R=20pt@C=40pt{
  &Y \ar@<-.5ex>[r]_-{\iota_1} \ar@<.5ex>[r]^-{\iota_2}  & Y\sqcup Y\ar[dd]_{\nabla_{(d_1,d_2)}} \ar[r]^{\gamma}&C.\ar@{.>}[ddl]^-{\exists!v}\\ 
  X  \ar[ur]^{\phi} \ar[dr]_\phi\\
&Y\ar@<-.5ex>[r]_{d_1} \ar@<.5ex>[r]^{d_2} & D  
}
\]
In particular $v\circ \gamma \circ \iota_1=d_1$ and $v\circ \gamma \circ \iota_1=d_2$, so we have the following series of equalities:
\[
d_1\circ m'=(v\circ  \gamma\circ   \iota_1)\circ  m'=v\circ  (\gamma\circ   \iota_1\circ  m')=v\circ  (\gamma\circ   \iota_2\circ  m')=(v\circ  \gamma\circ   \iota_2)\circ  m'=d_2\circ  m'.
\]
Hence, there is a unique morphism $u'\colon \Im(\phi)\to \eq(d_1,d_2)=I'$, such that $m'=\beta' \circ u'$.
\end{proof}





\subsection{(Epi, regular mono) and (regular epi, mono)-factorizations}

Given a morphism $\phi$ in a finitely bicomplete category $\X$, we have studied how $\phi$ can be factored through its image and coimage. We now give some conditions that are equivalent to the fact that the factorization through $\Im(\phi)$ is an (epi, regular mono)-factorization, and to the fact that the factorization through $\Coim(\phi)$ is a (regular epi,  mono)-factorization, respectively. In other words, referring to the diagram \eqref{BIG:diagram}, we give necessary and sufficient conditions for $e'$ to be an epimorphism, and for $m$ to be a monomorphism.

\begin{theorem}\label{thm_main}
Let $\X$ be a finitely bicomplete category. The following are equivalent:
\begin{enumerate}[\rm (1)]
\item every strong monomorphism is regular;
\item the class of regular monomorphisms is closed under composition;
\item for each $\phi$, consider the unique factorization $\phi=m'\circ e'$, where $m'\colon \Im(\phi)\to Y$ is the canonical regular mono from the image. Then, $e'$ is an epimorphism.
\end{enumerate}
Dually, the following statements are equivalent:
\begin{enumerate}[\rm (1$'$)]
\item every strong epimorphism is regular;
\item the class of regular epimorphisms is closed under composition;
\item for each $\phi$, consider the unique factorization $\phi=m\circ e$, where $e\colon X\to\Coim(\phi)$ is the canonical regular epi to the coimage. Then, $m$ is a monomorphism.
\end{enumerate}
\end{theorem}
\begin{proof}
(1)$\Rightarrow$(2). Regular monomorphisms are always strong, so (2) means exactly that the class of regular monomorphisms coincides with that of strong monomorphisms. Now use that the composition of two strong monomorphisms is still strong by Lemma \ref{strong_prop_lemma}. 

\smallskip\noindent
(2)$\Rightarrow$(3). Let us verify that $e'\colon X\to \Im(\phi)$ is an epimorphism. Indeed, choose two parallel arrows $\alpha,\,\beta\colon \Im(\phi)\rightrightarrows Z$ such that $\alpha\circ e'=\beta\circ e'$, and consider the following diagram
\[
\xymatrix@R=20pt@C=40pt{
	X\ar@{.>}[d]_-{\exists!v}\ar[r]^-{e'}&\Im(\phi)  \ar@<-.5ex>[r]_-{\alpha} \ar@<.5ex>[r]^-{\beta}   & Z.\\ 
	\eq(\alpha,\beta)\ar[ur]_-q&& 
}
\]
Since $q\colon \eq(\alpha,\beta)\to \Im(\phi)$ is the equalizer of $\alpha$ and $\beta$, there exists a unique morphism $v\colon X\to \eq(\alpha,\beta)$ such that $e'=q\circ v$. Now, $\phi=m'\circ e'=(m'\circ q)\circ v$ and, by hypothesis, $m'\circ q$ is a regular monomorphism. Hence, by Proposition \ref{univ_prop_facts_lemma}, there exists a unique morphism $u'\colon \Im(\phi)\to \eq(\alpha,\beta)$ such that $m'=(m'\circ q)\circ u'$, but $m'$ is a monic so $q\circ u'=\id{\Im(\phi)}$ and thus $q$ is a retraction. By hypothesis $q$ is a regular monomorphism, thus it is an isomorphism by Remark \ref{mono:vs:unique:diagonal_2} but then Corollary ~\ref{equalizer_of_equals} entails $\alpha=\beta$, as desired.

\smallskip\noindent
(3)$\Rightarrow$(1). Let $\phi$ be a strong monomorphism and write it as  $\phi=m'\circ e'$, where $e'$ is an epimorphism and $m'$ is a regular monomorphism. Since $\phi=m'\circ e'$ is a strong monomorphism, also $e'$ is a strong monomorphism (see Lemma~\ref{strong_prop_lemma}) and, as we already knew it was epic, we get that $e'$ is an isomorphism (for this, combine Proposition~\ref{quanti:tipi:mono:ci sono} with Lemma~\ref{extremal_prop_lemma}). Hence, $\phi$ is equal to the composition of an isomorphism with a regular monomorphism, and it is therefore a regular monomorphism. \qedhere 
\end{proof}

One can prove that, in any bicomplete category $\X$, the classes of extremal and strong monomorphisms (resp., epimorphisms) do coincide but the proof uses some properties of pull-back/push-out squares that we do not want to recall here. We conclude with the following corollary that shows that, if we suppose that all regular monomorphisms are strong, then, using the above theorem, it becomes very easy to prove that all the extremal monomorphisms are strong as well.

\begin{corollary}\label{coro_many_classes_usually_coincide}
Let $\X$ be a finitely bicomplete category that satisfies the equivalent conditions (1--3) of Theorem \ref{thm_main}. Then, we have the following equalities of classes of morphisms in $\X$:
\[
\left\{ \begin{matrix}\text{regular}\\\text{monomorphisms}\end{matrix}\right\}\ \overset{}=\ \left\{ \begin{matrix}\text{strong}\\\text{monomorphisms}\end{matrix}\right\}\ \overset{}=\ \left\{ \begin{matrix}\text{extremal}\\\text{monomorphisms}\end{matrix}\right\}.
\]
Dually, if $\X$ satisfies the equivalent conditions (1\,$'$--3\,$'$) of Theorem \ref{thm_main}, then we have the following equalities of classes of morphisms in $\X$:
\[
\left\{ \begin{matrix}\text{regular}\\\text{epimorphisms}\end{matrix}\right\}\ \overset{}=\ \left\{ \begin{matrix}\text{strong}\\\text{epimorphisms}\end{matrix}\right\}\ \overset{}=\ \left\{ \begin{matrix}\text{extremal}\\\text{epimorphisms}\end{matrix}\right\}.
\]
\end{corollary}
\begin{proof}
Let $\phi\colon X\to Y$ be an extremal epimorphism and consider its factorization through the image: 
\[
\xymatrix@R=20pt@C=50pt{
X\ar[rr]^{\phi}\ar[dr]_e&&Y\\
&\Im(\phi)\ar[ur]_m
}
\] 
where $m$ is a regular monomorphism by construction, and $e$ is an epimorphism by hypothesis. By Lemma \ref{extremal_prop_lemma} (2), $e$ is an extremal monomorphism and thus an isomorphism by Lemma \ref{extremal_prop_lemma}, therefore $\phi$ is a regular monomorphism, just like $m$.
\end{proof}

Another consequence of Theorem \ref{thm_main} is that, whenever strong and regular \marginpar{Non capisco, dove si usa qui che strong = regolare? A me sembra che che se $\beta$ è mono allora l'unico morfismo $u$ risulta un isomorfismo anche senza chiedere strong= regolare. Peraltro questa ipotesi appare solo nell'introduzione e non nel corollario vero e proprio, quindi mi pare fuorviante.} monomorphism (resp., epimorphisms) coincide, then the factorization of a morphism through its (co)image can be nicely characterized as follows:

\begin{corollary}\label{uniqueness_fact_coro} \reversemarginpar \marginpar{Semmai il fatto è che se non chiediamo strong = regolare allora la coimmagine non è più la fattorizzazione epi regolare-mono finale} \normalmarginpar 
Let $\X$ be a finitely bicomplete category and $\phi\colon X\to Y$ a morphism in $\X$. Then, the following statements hold true:
\begin{enumerate}[\rm (1)]
\item if $\alpha\colon X\to I$ is a regular epimorphism and $\beta\colon I\to Y$ is a monomorphism such that $\phi=\beta\circ\alpha$, then there is a unique isomorphism $u\colon I\tilde\longrightarrow \Coim(\phi)$ such that $u\circ \alpha=e$ and $\beta=m\circ u$. In particular, $\alpha\colon X\to I$ is a coimage for $\phi$;
\item if $\alpha'\colon X\to I'$ is an epimorphism and $\beta'\colon I'\to Y$ is a regular monomorphism such that $\phi=\beta'\circ\alpha'$, then there is a unique isomorphism $u'\colon \Im(\phi)\tilde\longrightarrow I'$ such that $\alpha'=u'\circ e'$ and $u'\circ \beta'=m'$. In particular, $\beta'\colon I'\to Y$ is an image for $\phi$.
\end{enumerate}
\end{corollary}
\begin{proof}
We just verify (1), as (2) follows dually. Take the following commutative diagram:
\[\xymatrix@R=30pt@C=30pt{
X\ar@{>}[r]^{\alpha}\ar@{>}[d]_{e}&I\ar@{.>}[dl]_{\exists!u}\ar[d]^{\beta}\\
\Coim(\phi)\ar@{>}[r]_{m}&Y
}\]
\iffalse 
\[
\xymatrix@R=10pt@C=70pt{
X\ar[rr]^{\phi}\ar[dr]_{\alpha}\ar@{>}[dddd]_{\id{X}}&&Y\ar@{>}[dddd]^{\id{Y}}\\
&I\ar@{.>}[dd]_{\exists!u}\ar[ur]_{\beta}\\
\\
&\Coim(\phi)\ar[dr]^{m}\\
X\ar[rr]_{\phi}\ar[ur]^{e}&&Y
}
\]\fi 
By Proposition \ref{univ_prop_facts_lemma} and Remark \ref{univ_prop_facts_lemma_2} , there exists a unique morphism $u\colon I\to \Coim(\phi)$ such that $u\circ \alpha=e$ and $m\circ u=\beta$. We should verify that $u$ is an isomorphism. Note that $u$ is a strong epimorphism (by Lemma \ref{strong_prop_lemma}), so in particular also an extremal epimorphism. Now, $\beta$ is monic so, by Proposition \ref{composition}, also $u$ is and we can conclude by Lemma \ref{extremal_prop_lemma}.
\end{proof}

\begin{corollary}\label{reg2}\marginpar{Questo prima era un lemma nella sezione su reg che secondo me sta meglio qua. Vanno controllate le ipotesi! (vedi osservazione nella pagina precedente)}
An arrow $m\colon M\rightarrow X$ in a finitely bicomplete $\X$ is a regular monomorphism if and only if the canonical morphism $M\to \Im(m)$ is an isomorphism.
\end{corollary}
\begin{proof}
	$(\Leftarrow)$ Since $m$ is a regular monomorphism, the trivial factorization $m=\id M\circ m$ is an (epi, regular mono)-factorization. The thesis now follows by point (2) of the previous corollary. 
	
	\smallskip\noindent
	$(\Rightarrow)$  This follows at once since $\Im(m)\to X$ is a regular monomorphism. 
	%then fact $3$ above gives 
	%\begin{center}
	%\begin{tikzpicture}
	%			\node(E)at(-2.5,0){$M$};
	%			\node(A)at(0,0){$\im{M}$};
	%			\node(C)at(2.5,0){$M$};
	%			\node(D)at(0,-1.3){$M$};
	%			\draw[->](A)--(C)node[pos=0.5, above]{$\im{m}$};
	%			\draw[->](E)..controls(-2,1.3)and(2,1.3)..(C)node[pos=0.5, above]{$n$};
	%			\draw[->](E)--(A)node[pos=0.5, above]{$e_m$};
	%			\draw[<-](D)--(A)node[pos=0.5, right]{$u$};		
	%			\draw[->](E)--(D)node[pos=0.5, below, xshift=-0.1cm]{$1_{M}$};
	%			\draw[->](D)--(C)node[pos=0.5, below, xshift=0.1cm]{$m$};			
	%			\end{tikzpicture}
	%		\end{center}
	%		$u$ is split epi since $u\circ e_m=1_M$ and it is mono since $m\circ u=\im{M}$, thus it is an isomorphism.
\end{proof}



\subsection{Examples}

In this subsection we give some examples to illustrate how images, coimages, and the related factorizations of morphisms look like in some well-known categories. 


\begin{example}
In the category $\Set$ both images and coimages do coincide with what one expects to call the image of a function. More explicitly, given a function $\phi\colon X\to Y$ between two sets, let $\phi(X):=\{\phi(x)\in Y:x\in X\}\subseteq Y$, and call $e\colon X\to \phi(X)$ the obvious corestriction of $\phi$, and $m'\colon \phi(X)\to Y$ the inclusion. Then, $\Im(\phi)=\phi(X)=\Coim(\phi)$, and both the factorization through the image and the coimage coincide with $\phi=m'\circ e$.
\end{example}

\begin{example}
In $\Top$ the image and the coimage of a morphism $\phi\colon X \to Y$ do have the same underlying set, and this is the set-theoretic image $\phi(X)$. On the other hand, image and coimage carry different topologies: while the topology of $\Coim(\phi)$ is the quotient topology of $e\colon X \to \phi(X)$, $\Im(\phi)$ carries the subspace topology of $m' \colon \phi(X) \hookrightarrow Y$. In this case, the canonical map $d\colon \Coim(\phi)\to \Im(\phi)$ is just the identity, which is continuous but, in general, not open (due to the different choice of topologies on $\phi(X)$).
\end{example}

\begin{example}
Let $\Haus$ be the category of Hausdorff topological spaces, and continuous maps between them. Given a morphism $\phi\colon X\to Y$ in $\Haus$, the coimage $\Coim(\phi)$ can be identified with the set-theoretic image $\phi(X)$ and, letting $e\colon X \to \phi(X)$ be the corestriction of $\phi$, the topology $\Coim(\phi)$ is the quotient topology of $e\colon X \to \phi(X)$. 
Let us now describe the image $m'\colon \Im(\phi)\to Y$. Indeed, being $m'$ the equalizer of two parallel maps in $\Haus$, it is the inclusion of a closed subspace of $Y$. In fact, it is not difficult to verify that $\Im(\phi)$ is, as a set, precisely the closure $\overline{\phi(X)}$ of the set-theoretic image $\phi(X)$ in $Y$, and it carries the subspace topology with respect to the inclusion in $Y$. With this explicit description at hand, it is easy to find explicit examples of morphisms $\phi$ for which $\Coim(\phi)$ is properly contained in $\Im(\phi)$ so, contrarily to what happens in $\Top$, here image and coimage do not have in general the same underlying set (e.g., take $X=\Q$, $Y=\R$, with their usual topologies, and let $\phi\colon \Q\to \R$ be the obvious inclusion, then $\Coim(\phi)=\Q$ and $\Im(\phi)=\R$).
\end{example}

After the above examples of a topological flavor, we now turn our attention to more algebraic examples.

\begin{example}
In the category of Abelian groups $\Ab$ the situation is similar to what happens in $\Set$: both images and coimages coincide, up to isomorphism, with the set-theoretic image. More precisely, given a homomorphism $\phi\colon X\to Y$ of Abelian groups, we let as usual $\ker(\phi):=\{x\in X:\phi(x)=0\}$. Then, the coimage of $\phi$ is constructed by taking the quotient $e\colon X\to X/\ker(\phi)$, while the image is the inclusion $m'\colon\phi(X)\to Y$. The comparison map $d\colon X/\ker(\phi)\to \phi(X)$ takes a coset $x+\Ker(\phi)$ to $\phi(x)$. The fact that this $d$ is an isomorphism is usually referred to as the ``First Isomorphism Theorem'' (see also Proposition~\ref{algebraic_factorization_prop} below).
\end{example}

\marginpar{Non mi è chiaro che cosa intendo con purificazione, a me sembra che l'immagine sia data da $\phi(X)_*$, l'insieme dei $y\in Y$ tali che $y^n\in \phi(X)$ per qualche $n$.}

The situation of the above example is not unique to Abelian groups: the fact that the canonical morphism $d\colon \Coim(\phi)\to\Im(\phi)$ is an isomorphism (that is, the fact that ``the First Isomorphism Theorem holds'') is one of the traditional axioms defining an Abelian category. Hence, in any Abelian category (e.g., the category of left $R$-modules $R\text{-}\mathrm{Mod}$ over a ring $R$ or even the categories of presheaves, or that of quasi-coherent sheaves, over a scheme) image and coimage always coincide.


\begin{example}
Consider the full subcategory $\TFA\subseteq \Ab$ of the torsion-free Abelian groups. The obvious inclusion $\iota\colon \Z \to \Q$ has (up to isomorphism) \[\Z = \Coim(\iota) < \Im(\iota) =\Q\]
 More generally, for a morphism $\phi\colon X\to Y$ in $\TFA$, the morphism $e\colon X \to \Coim(\phi)$ is the usual corestriction $X \to \phi(X)$ of $\phi$, while $m\colon \Coim(\phi)\to Y$ is simply the inclusion $\phi(X) \hookrightarrow Y$. On the other hand, $m' \colon \Im(\phi) \to Y$ is, by definition, the equalizer of two parallel morphisms, so it is the inclusion of a pure subgroup of $Y$. In fact, $\Im(\phi)$ can be identified with the purification $\phi(X)_*$ of $\phi(X)$ in $Y$. Then, $e'\colon X \to \phi(X)_* = \Im(\phi)$
is the corestriction of $\phi$ and it is a, not necessarily surjective, epimorphism in $\TFA$, as the quotient $\phi(G)_*/\phi(G)$ is a torsion group. 
\end{example}

\begin{example}\cite[Proposition~1.8]{MS}
Let $\Cat$ be the category of all small categories, with functors as morphisms. It is known that monomorphisms in $\Cat$ are those functors that are faithful and injective on objects. Consider then the following small categories:
\[
X:=\{a_0,a_1,a_2,a_3,a_4,a_5\}\times\{0\to 1\}\qquad\text{and}\qquad Y:=\{0\to 1\to 2\},
\]
that is, $X$ is the disjoint union of six copies of the ordinal $2$, while $Y$ is the ordinal $3$. In particular, $Y$ has exactly $6$ distinct morphisms (three identities and three non-identities). Let us give names to the morphisms in $Y$: we let $f_i:=\id i$ (for $i=0,1,2$), $f_3\colon 0\to 1$, $f_4\colon 1\to 2$ and $f_5\colon 0\to 2$. Take now the unique functor $\phi\colon X\to Y$ such that $\phi[(a_i,0)\to (a_i,1)]:=f_i$, for $i=0,\dots,5$. The coimage $\Coim(\phi)$ is then the following category:
\[
\Coim(\phi):=\left\{\xymatrix{0\ar[r]\ar@/_-0.5cm/[rr]&1\ar[r]&2}\right\}
\]
which has $7$ distinct morphisms (as there are two distinct morphisms $0\rightrightarrows 2$). Hence, the canonical morphism $m\colon \Coim(\phi)\to Y$, which identifies the two morphisms $0\rightrightarrows 2$ is not monic in $\Cat$.
\end{example}


Let us return for just a moment to the abstract setting, where $\X$ is a finitely bicomplete category, $\phi\colon X\to Y$ a morphism in $\X$, and denote as usual by $m\colon \Coim(\phi)\to Y$ and $m'\colon \Im(\phi)\to Y$, the morphisms appearing in the factorizations of $\phi$ through the coimage and image, respectively. Clearly, when $\phi$ is epic, both $m$ and $m'$ are epic. On the other hand, we have seen in Corollary \ref{epiepi} that, if $m$ is epic, then so is $\phi$. The above example shows why the fact that $m'$ is epic may not be enough, in general, to infer that $\phi$ is epic. In fact, it may happen that $e'\colon X\to \Im(\phi)$ is not an epimorphism (take, as an example, the opposite category $\Cat^{\mathrm{op}}$, reversing arrows in the above example).
\medskip 

We conclude the subsection with the following result, about monadic categories (which are finitely bicomplete by Theorem \ref{bicomp}).
\begin{proposition}\label{algebraic_factorization_prop}\marginpar{Ora questo e i due corollari discendono dalla sottosezione sulle monadi}
	Let $(T, \eta, \mu)$ be a monad on $\Set$ and $\phi:\xi_1\rightarrow \xi_2$ be a morphism in $\eim{T}$ and suppose $\phi=m\circ e$  where $e:\xi_1\rightarrow \Coim(\phi)$ and $m:\Coim(\phi)\rightarrow \xi_2$ are given by the factorization through the coimage. Then $|m|\circ |e|$ is the injective-surjective factorization of $|\phi|$ in $\Set$. In particular $e$ is surjective and $m$ injective.    
\end{proposition}
\begin{proof}
	Take $X=U_T(\xi_1)=X$, $Y=U_T(\xi_2)$ and $K$ to be
	\[\{(x,y)\in X\times X\mid \phi(x)=\phi(y) \}\]
		$e$ is the coequalizer of $k_1, k_2:\kappa\rightarrow \xi_1$, where $\kappa:T(K)\rightarrow K$ is the kernel pair of $\phi$ in $\eim{T}$. Notice 
	 that in such case $E_{(k_1,k_2)}=K$, thus, by the construction of coequalizers in point (1) of Theorem \ref{limcolim}, $U_T(e):X\rightarrow U_T(\Coim(\phi))$ is the projection on the quotient $X/K$. The thesis now follows at once.
\end{proof}

\begin{corollary}
In any concrete category $(\X, |-|)$ monadic over $\Set$ if $m\circ e$ is the factorization of a morphism through its coimage, then $e$ is surjective and $m$ injective.
\end{corollary}

\begin{corollary}
	In any concrete category $(\X, |-|)$ monadic over $\Set$ the following equalities hold\[
	\left\{ \begin{matrix}\text{surjective}\\\text{morphisms}\end{matrix}\right\}\ \overset{}=\ \left\{ \begin{matrix}\text{regular}\\\text{epimorphisms}\end{matrix}\right\}\ \overset{}=\ \left\{ \begin{matrix}\text{strong}\\\text{epimorphisms}\end{matrix}\right\}\ \overset{}=\ \left\{ \begin{matrix}\text{extremal}\\\text{epimorphisms}\end{matrix}\right\}.
	\]
\end{corollary}
\begin{proof}By Corollaries \ref{monadic_surjective} and \ref{coro_many_classes_usually_coincide} it is enough to show that every regular epi is surjective. Indeed, take a regular epimorphism $\phi\colon X\to Y$ and consider its factorization through the coimage:
	\[
	\xymatrix@R=20pt@C=50pt{
		X\ar[rr]^{\phi}\ar[dr]_e&&Y.\\
		&\Coim(\phi)\ar[ur]_m
	}
	\] 
	By the previous corollary $m$ is a monomorphism and an extremal epimorphism (since it is strong, as $\phi=m\circ e$ is strong), and so $m$ is an isomorphism. Since $|\phi|=|m|\circ |e|$ with $|e|$ surjective and $|m|$ an isomorphism (as the forgetful functor, just like any other functor, sends isomorphisms to isomorphisms) we conclude that $\phi$ is surjective.
\end{proof}

\iffalse 
\begin{proposition}\label{algebraic_factorization_prop} 
	
	Let $(\X, |-|)$ be a monadic category  over $\Set$ and consider the usual diagram depicting the two factorizations we are studying: 
	\[
	\xymatrix@C=40pt@R=15pt{
		&& \Im(\phi) \ar[dr]^{m'}  & & \\ 
		K \ar@<-.5ex>[r]_-{\kappa_1} \ar@<.5ex>[r]^-{\kappa_2} &X \ar[ru]^{e'} \ar[rr]^(0.6){\phi}\ar[dr]_{e}&  & Y \ar@<-.5ex>[r]_-{\gamma_1} \ar@<.5ex>[r]^-{\gamma_2} & C.\\
		& & \Coim(\phi) \ar[ur]_{m} \ar@{.>}[uu]^(0.3){d}  &  &
	}
	\] 
	
	
	$|\Coim(\phi)|=|\phi|(|X|)$, that is, the underlying set of the coimage is the set-theoretic image, while $|e|\colon |X|\to |\Coim(\phi)|$ and $|m|\colon |\Coim(\phi)|\to |Y|$ can be naturally identified with the usual surjection $|X|\to |\phi|(|X|)$ (obtained by corestriction from $|\phi|$) and the embedding $|\phi|(|X|)\to |Y|$, respectively. In particular, $e$ is always a regular epimorphism, while $m$ is always a monomorphism. Finally, the following classes of morphisms in $\X$ all coincide:
	\[
	\left\{ \begin{matrix}\text{surjective}\\\text{morphisms}\end{matrix}\right\}\ \overset{}=\ \left\{ \begin{matrix}\text{regular}\\\text{epimorphisms}\end{matrix}\right\}\ \overset{}=\ \left\{ \begin{matrix}\text{strong}\\\text{epimorphisms}\end{matrix}\right\}\ \overset{}=\ \left\{ \begin{matrix}\text{extremal}\\\text{epimorphisms}\end{matrix}\right\}.
	\]
\end{proposition} 
\begin{proof}
	contenuto...
\end{proof}


\begin{proposition}\label{algebraic_factorization_prop}
Let $\X$ be a category of ``algebraic origin'' like $\Ab$, $\Grp$, $\Cmon$ (commutative monoids), $\Mon$ (monoids), $\Csemi$ (commutative semigroups), $\Semi$ (semigroups), $\SLatt$, $\SLattnobottom$, and let $\phi\colon X\to Y$ be a morphism in $\X$. Consider then the usual diagram depicting the two factorizations we are studying: 
\[
\xymatrix@C=40pt@R=15pt{
 && \Im(\phi) \ar[dr]^{m'}  & & \\ 
K \ar@<-.5ex>[r]_-{\kappa_1} \ar@<.5ex>[r]^-{\kappa_2} &X \ar[ru]^{e'} \ar[rr]^(0.6){\phi}\ar[dr]_{e}&  & Y \ar@<-.5ex>[r]_-{\gamma_1} \ar@<.5ex>[r]^-{\gamma_2} & C.\\
& & \Coim(\phi) \ar[ur]_{m} \ar@{.>}[uu]|(0.3){d}  &  &
}
\]
Then, $|\Coim(\phi)|=|\phi|(|X|)$, that is, the underlying set of the coimage is the set-theoretic image, while $|e|\colon |X|\to |\Coim(\phi)|$ and $|m|\colon |\Coim(\phi)|\to |Y|$ can be naturally identified with the usual surjection $|X|\to |\phi|(|X|)$ (obtained by corestriction from $|\phi|$) and the embedding $|\phi|(|X|)\to |Y|$, respectively. In particular, $e$ is always a regular epimorphism, while $m$ is always a monomorphism. Finally, the following classes of morphisms in $\X$ all coincide:
\[
\left\{ \begin{matrix}\text{surjective}\\\text{morphisms}\end{matrix}\right\}\ \overset{}=\ \left\{ \begin{matrix}\text{regular}\\\text{epimorphisms}\end{matrix}\right\}\ \overset{}=\ \left\{ \begin{matrix}\text{strong}\\\text{epimorphisms}\end{matrix}\right\}\ \overset{}=\ \left\{ \begin{matrix}\text{extremal}\\\text{epimorphisms}\end{matrix}\right\}.
\]
\end{proposition} 
\begin{proof}
In all the mentioned categories, the forgetful functor $|-|\colon \X\to \Set$ has a left adjoint and so it commutes with products and equalizers. Hence, these constructions can be performed just like in $\Set$. More precisely, the product of two objects $A$ and $B$ is just the cartesian product of the two underlying sets $|A|\times |B|$ with pointwise operation, while the equalizer of two  maps $f,\,g\colon A\rightrightarrows B$ is  the subset $\eq(|f|,|g|)=\{a\in |A|:|f|(a)=|g|(a)\}\subseteq |A|$, with the operation inherited from $A$ (by restriction). Consider the following diagram, where $\kappa\colon K\to X\times X$ is the obvious inclusion:
\[
\xymatrix@C=35pt{
K=\eq(\phi\circ\pi_1,\phi\circ\pi_2)\ar[r]^-\kappa& X\times X \ar@<-.5ex>[r]_-{\pi_1} \ar@<.5ex>[r]^-{\pi_2} &X\ar[r]^{\phi}&Y.
}
\]
We now make the following claim:
\begin{claim}\label{claim_inside_prop_fact_in_alg}
When $\X$ is one of the categories mentioned in the statement and $\phi\colon X\to Y$ is surjective, it is the coequalizer of the pair $\pi_1\circ\kappa,\, \pi_2\circ \kappa\colon K\rightrightarrows X$. 
\end{claim}
To prove this, one needs an explicit description of coequalizers in $\X$, so let us describe them in general. Indeed, given $f,\, g\colon A\to B$ in $\X$, the coequalizer $c\colon B\to C$ is constructed as follows: one first takes the following set 
\[
I_{(f,g)}:=\{(f(a),g(a)):a\in A\}\subseteq |B|\times |B|.
\]
Then, one takes the smallest congruence relation $\mathcal R_{(f,g)}\subseteq  |B|\times |B|$ containing $I_{(f,g)}$, where by congruence relation we mean an equivalence relation that is also closed under the needed operation (the sum in $\Ab$, $\Cmon$ and $\Csemi$, multiplication in $\Grp$, $\Mon$ and $\Semi$, finite suprema in $\SLatt$ and $\SLattnobottom$, etc.). Finally one takes the set of equivalence classes $|B|/\mathcal R_{(f,g)}$ and notices that, whatever the operation we have in $B$, it induces a well-defined operation among the equivalence classes (precisely because we have used a congruence and not just an equivalence relation). Hence, $|B|/\mathcal R_{(f,g)}$ is the underlying set of an object $C$ in $\X$ (i.e., $|C|=|B|/\mathcal R_{(f,g)}$) and the map $|X|\to |C|$ sending $x$ to its equivalence class $[x]$ respects the operation (by construction) and it therefore underlies a morphism $c\colon X\to C$. This morphism is the desired coequalizer. 

Now that we have this explicit description of coequalizers we can easily conclude: as sets we have that
\[
K=\eq(\phi\circ\pi_1,\phi\circ\pi_2)=\{(x,y)\in |X|\times |X|: |\phi|(x)=|\phi|(y)\},
\]
so $K$ is already a congruence (it is clearly an equivalence relation, and it is closed under the desired operation, as $\phi$ is supposed to respect such operation) and the equivalence class of a given element $x\in |X|$ is just 
\[
[x]=\{y\in X:|\phi|(y)=|\phi|(x)\}=|\phi|^{-1}(|\phi|(x)).
\] 
Hence, the coequalizer of the pair $\pi_1\circ\kappa,\, \pi_2\circ \kappa\colon K\rightrightarrows X$, which is the set of these equivalence classes with the operation induced by $X$, can be clearly identified with $Y$. We have just proved that all surjective morphisms are regular epimorphisms in $\X$.

With this in mind, we can now prove Claim \ref{claim_inside_prop_fact_in_alg}: given a morphism $\phi\colon X\to Y$ in $\X$, where $\X$ is any of the categories mentioned in the statement, the subset $f(X)\subseteq Y$ is easily seen to be a subobject, and the obvious projection $p\colon X\to \phi(X)$ and inclusion $i\colon \phi(X)\to Y$ are morphisms in $\X$. Furthermore, as $p$ is surjective, it is a regular epimorphism by the first part of the proof, while $i$ is injective and, therefore, a monomorphism. By Corollary \ref{uniqueness_fact_coro}, the morphism $p\colon X\to \phi(X)$ is the coimage of $\phi$ and $\phi=i\circ \phi$ is the unique factorization of $\phi$ through the coimage. Note also that this proves condition (3$'$) of Theorem~\ref{thm_main}  (as all the factorizations through the coimage have clearly monic, even injective, second map) and therefore, by Corollary~\ref{coro_many_classes_usually_coincide}, regular, strong and extremal epimorphisms are all the same. The last thing to check is that regular epimorphisms are surjective. Indeed, take a regular epimorphism $\phi\colon X\to Y$ and consider its factorization through the coimage that, we now know, looks like depicted in the following diagram:
\[
\xymatrix@R=20pt@C=50pt{
X\ar[rr]^{\phi}\ar[dr]_p&&Y.\\
&\phi(X)\ar[ur]_i
}
\] 
Then, $i$ is both a monomorphism and an extremal epimorphism (since it is strong, as $\phi=i\circ p$ is strong), and so $i$ is an isomorphism. Since isomorphisms are always bijective (as the forgetful functor, just like any other functor, sends isomorphisms to isomorphisms, and isomorphisms in $\Set$ are the bijective maps), $\phi$ is the composition of a surjective map with a bijective one. Hence, it is surjective.
\end{proof}
\fi 


Notice that these last results applies, in particular, to all categories of ``algebraic origin'' like $\Ab$, $\Grp$, $\Cmon$ (commutative monoids), $\Mon$ (monoids), $\Csemi$ (commutative semigroups), $\Semi$ (semigroups), $\SLatt$, $\SLattnobottom$, \dots

\section{Convenient hypotheses for the epimorphism problem}\label{convenient_setting_sec}

In this subsection we want to briefly summarize the results proved until now and, based on that, to obtain a list of hypotheses that are optimal for the study of the epimorphism problem in an abstract categorical setting. As we will see, all these hypotheses are satisfied in categories monadic over $\Set$, so, in particular, by  categories of models of (finitary) algebraic structures as $\Ab$, $\Grp$, $\Cmon$ (commutative monoids), $\Mon$ (monoids), $\Csemi$ (commutative semigroups), $\Semi$ (semigroups), $\SLatt$, $\SLattnobottom$, etc\dots, showing that they are quite natural to assume.

First of all, we have seen that, to formulate the epimorphism problem we need to start with a concrete category $(\X,|-|)$ (just to make sense of expressions like ``injective morphism'' or ``surjective morphism''). Furthermore, whenever $\X$ is finitely bicomplete and $|-|$ commutes with finite products and equalizers then all monomorphisms are injective. Therefore, it seems harmless to assume:
\begin{enumerate}[\rm ({Hyp.}1)]
\item $(\X,|-|)$ is a concrete category;
\item $\X$ is finitely bicomplete;
\item there is a ``free functor'' $F\colon \Set \to \X$ which is left adjoint to $|-|\colon \X\to \Set$. 
\end{enumerate}

Once we have assumed (Hyp.1--3), there is an additional simplification that is often useful. Indeed, consider an epimorphism $\phi\colon X\to Y$ in $\X$ and consider the canonical factorization through the coimage:
\[
\xymatrix@R=20pt@C=50pt{
X\ar[rr]^{\phi}\ar[dr]_{e_\phi}&&Y\\
&\Coim(\phi)\ar[ur]_{m_\phi}
}
\] 
where $e_\phi$ is a regular epimorphism. We have seen that, whenever the class of regular epimorphisms coincides with that of strong epimorphisms, then $m_\phi$ is a monomorphism and, as $\phi$ is epic, also an epimorphism. 

Restricting again our attention to categories monadic over $\Set$, we have seen in Proposition~\ref{algebraic_factorization_prop} that, in all such categories, the following holds:
\begin{enumerate}[\rm ({Hyp.}4)]
\item in the concrete category $(\X,|-|)$ the following equalities hold true:
\[
\left\{ \begin{matrix}\text{surjective}\\\text{morphisms}\end{matrix}\right\}\ \overset{}=\ \left\{ \begin{matrix}\text{regular}\\\text{epimorphisms}\end{matrix}\right\}\ \overset{}=\ \left\{ \begin{matrix}\text{strong}\\\text{epimorphisms}\end{matrix}\right\}\ \overset{}=\ \left\{ \begin{matrix}\text{extremal}\\\text{epimorphisms}\end{matrix}\right\}.
\]
\end{enumerate}

Note that, whenever the hypotheses (Hyp.1--4) are satisfied, the morphism $e_\phi$ in the diagram above is automatically surjective and, to check whether or not $\phi$ is surjective it is enough to check whether the injective epimorphism $m_\phi\colon \Im(\phi)\to Y$ is surjective. As a result of our discussion we have obtained that, in all categories satisfying (Hyp.1--4) the epimorphism problem becomes equivalent to the following:

\begin{question}
Let $m\colon M\to X$ be the inclusion of a subobject and suppose that $m$ is epic. Under which hypotheses is $m$ surjective?
\end{question}

In the following two sections we will always assume hypotheses (Hyp. 1--4).

\section{A connection with closure operators}
%Here we say a few words about Theorem \ref{AbsNon} and the use of the term ``regular closure". 
%
%The fact that the maps $f : X \to Y$ with dense image are epimorphisms in the category $\X = \Top_2$ of
%Hausdorff spaces belongs to the folklore and probably gave rise the well rooted intuition that empimorphisms
%are the morphisms with a ``dense" image, whatever this ``dense" may mean. A formal ground for discussing ``density"
%was gradually created in seventies and the eighties of the last century (with the pioneering papers of John Isbell  and S\' ergio de Ornelas Salbany, where the dominion and the Kuratowski closure played a prominent role). 
%
In this section we explore the connection between the epimorphism problem and the theory of closure operators. For this, let us start by fixing a category $\X$ that fits in the convenient setting introduced in Section \ref{convenient_setting_sec}:
\subsection{Some remarks on subobjects}\label{subobjects}\marginpar{Aggiunta questa sezione che incorpora materiale anche da quelle successive}

\begin{definition} Let $m:M\rightarrow X$ and $n:N\rightarrow X$ be two monomorphism with the same codomain, we say that $m\leq n$ if and only if there exists a $\phi:M\rightarrow N$ such that $n\circ \phi=m$ 
	\[
	\xymatrix {M \ar@{.>}[rr]^{\phi} \ar[dr]_{m}&& N\ar[dl]^{n}\\
	& X}
	\]
We define $m\equiv n$ if and only if $n\leq m$ and $m\leq n$. This is an equivalence relation on the (possibly proper) class
\[\mathcal{M}/X=\{m:A\rightarrow X \mid m \text{ is monic }\}\]

A \emph{subobject of $X$} is an equivalence class $[m]$ with respect to the relation $\equiv$, we will denote by $\sub{\X}(X)$ the class of these subobjects.
\end{definition}

\begin{remark}
	Notice that $m\equiv n$ is equivalent to the existence of an isomorphism $\phi$ such that $n\circ \phi = m$. 
\end{remark}

\begin{remark}
	$\sub{\X}(X)$ is naturally equipped with the order $[m]\leq[n]$ if and only if $m\leq n$ and, with respect to this order, it has a maximum $[\id{X}]$.
\end{remark}

\begin{example}
	Let $\X$ be $\Set$ and $m:M\rightarrow X$ $n:N\rightarrow X$ two monomorpihisms, then $m\leq n$ if and only if the inclusion of sets $m(M)\subseteq n(N)$ holds. In particular this means that $\sub{\Set}(S)$ is an ordered set (not a class) which is isomorphic to $\mathcal{P}(S)$ ordered by the inclusion.
\end{example}

The previous example implies that $\sub{X}(X)$ is a set for every concrete category $(\X, |-|)$ satisfying our hypotheses (Hyp. 1--4): since $|-|$ preserves monomorphism, we have a monotone function
\[\sub{X}(X)\rightarrow\sub{Set}(|X|)\quad
[m]\mapsto [|m|]
\] 
which is injective since $|-|$ is faithful. Categories which enjoy this property are often called \emph{well-powered}.

\medskip 

Let now $\phi:X\rightarrow Y$ be a morphism in $\X$ and $m:M\rightarrow Y$ a monic, take the pullback square
\[\xymatrix{
	\phi^*(M)\ar@{}[dr]|-{\text{P.B.}}\ar[d]_{\phi^*(m)}\ar[r]^{f}&M\ar[d]^m\\
	X\ar[r]_{\phi}&Y
}
\]
by Lemma $\ref{mono_are_pullback_stable}$ we know that $\phi_*(m)$ is a mono, so $[\phi_*(m)]\in \sub{X}(X)$. This yields a functor, as summarized by the following proposition.

\begin{proposition}\label{prop_monos_lemma}
There exists a functor $\sub{X}:\X^{op}\rightarrow \catname{Pos}$ which sends $X$ to $\sub{\X}(X)$ and $\phi:X\rightarrow Y$ to
\[
\phi^*:\sub{\X}(Y)\rightarrow \sub{\X}(X)\qquad 
[m]\mapsto [\phi^*(m)]
\]
Moreover $\phi^*$ has a left adjoint $\phi_!:\sub{\X}(X)\rightarrow \sub{\X}(Y)$.
\end{proposition}
\begin{proof} 
The condition on the identity is obvious since the following is a pullback square.
\[
\xymatrix{X\ar[r]^{\phi} \ar[d]_{\id{X}}& X\ar[d]^{\id{X}}\\
	X \ar[r]_{\phi}& X}
\]
For composition, take two arrows $\phi_1:X_1\rightarrow X_2$ and $\phi_2:X_2\rightarrow X_3$ and a monomorphism $m:M\rightarrow M$. We have a commutative diagram
\[
\xymatrix{\phi^*_1(\phi_2^*(M)) \ar[d]_{\phi_1^*(\phi_2^*(m))}\ar[r]^{f_1} &\phi_2^*(M) \ar[d]_{\phi_2^*(m)}\ar[r]^{f_2}& M\ar[d]^{m}\\
	X_1\ar[r]_{\phi_1} & X_2\ar[r]_{\phi_2} & X_3
}
\]
in which the two squares are pullbacks, if we show that the external rectangle is a pullback too then we are done. If $p:Q\rightarrow M$ and $q:Q\rightarrow X_1$ are such that $(\phi_2\circ \phi_1)\circ q=m\circ p$ then there exists a unique $f':Q\rightarrow P_2$ such that $\phi_2^*(m)\circ f'=\phi_1\circ q$ and $f_2\circ f'=p$, by the pullback property of $P_1$ we have a unique $f:Q\rightarrow P_1$ such that $\phi_1^*(\phi_2^*(m))\circ f=q$ and $f_1\circ f=f'$ and this is our witness that the outer rectangle is a pullback.

We are left with $\phi_!$. Now, (Hyp.4) and Theorem~\ref{thm_main} imply that in the coimage factorization of $\psi\colon Z\to T$ in $\X$
	\[
	\xymatrix@C=40pt@R=10pt{
		Z\ar[rd]_{e_\psi}\ar[rr]^\psi&&T,\\
		&\Coim(\psi)\ar[ur]_{m_\psi}
	}
	\]
 $m_\psi$ is monic. So, given $m\colon M\rightarrow X$, we can define
	\[
	\phi_!:\sub{\X}(X)\rightarrow\sub{\X}(Y)\qquad  \phi_!([m]):=[m_{\phi\circ m}]
	\] 
For simplicity we will denote $m_{\phi\circ m}$
 as $\phi_!(m):\phi_!(M)\rightarrow Y$.	 Notice that, as in the case of $\phi^*$, its domain is define donly up to isomorphism. Let also $\pi_m:X\rightarrow \phi_!(M)$ be the regular epi $e_{\phi\circ m}$. Now, if $m\leq n$, then $m=n\circ \psi$ for some $\psi$ and we have a commutative diagram
 
  \[
  \xymatrix{M \ar[d]_{\pi_m} \ar@/^0.75cm/[rr]^{\psi} \ar[r]^{m}& X\ar[d]_{\phi} & N\ar[l]_{n} \ar[d]_{\pi_n}\\
  	\phi_!(M)\ar@{.>}@/_0.75cm/[rr]_{\theta} \ar[r]^{\phi_!(m)} & Y & \phi_!(N) \ar[l]_{\phi_!(n)}
  } \]
Let $k_1, k_2:K\rightrightarrows M$ be the kernel pair of $\phi\circ m$, then, by definition
\[(\phi \circ m) \circ k_1=(\phi \circ m) \circ k_2\]
but, by the previous diagram
\[\phi \circ m= \phi_!(n)\circ \pi_n\circ \psi\]
therefore we get the dotted $\theta:\phi_!(M)\to \phi_!(N)$ by the coequalizer property og $\phi_!(M)$.

  To conclude, we should verify that the following conditions are equivalent, for all monics $m\colon M\to X$ and $n\colon N\to Y$:
	\begin{enumerate}[\rm (a)]
		\item $[m]\leq [\phi^*(n)]$, that is, there is $k\colon M\rightarrow \phi^*(N)$ such that $\phi^*(n)\circ k=m$;
		\item $[\phi_!(m)]\leq [n]$, that is, there is $h\colon \phi_!(M)\rightarrow N$ such that $\phi_!(m)=n\circ h$.
	\end{enumerate}
	\noindent
	(a)$\Rightarrow$(b). Consider the following commutative diagram
	\[
	\xymatrix@R=30pt@C=30pt{
		M\ar[rd]_m\ar[r]^k&\phi^*(N)\ar@{}[dr]|{\text{P.B.}}\ar[d]_{\phi^*(n)}\ar[r]^{p_n}&N\ar[d]^n\\
		&X\ar[r]_\phi&Y
	}
	\]
	In particular  $n\circ (p_{n}\circ k)=\phi\circ m$, so
	\[
	\xymatrix@R=30pt@C=30pt{
	M\ar[r]^k\ar[d]_{\pi_m}&\phi^*(N)\ar[r]^{p_n}&N\ar[d]^n\\
	\phi_!(M)\ar[rr]_{\phi_!(m)}\ar@{.>}[urr]^{k'}&&Y.
	}
	\]
	commutes too. Furthermore, by (Hyp.4), $\pi_m$ is strong, therefore the second diagram has a unique diagonal $k'\colon \phi_!(M)\to N$, witnessing $[\phi_!(m)]\leq [n]$.
	
	\smallskip\noindent
	(b)$\Rightarrow$(a). Since $[\phi_!(m)]\leq[n]$, we have that $[\phi^*(\phi_!(m))]\leq [\phi^*(n)]$, so it is enough to verify that $[m]\leq [\phi^*(\phi_!(m))]$. This follows immediately from the universal property of pullbacks applied to the following diagram:
	\belowdisplayskip=-14.5pt
	\[
	\xymatrix@C=50pt@R=17pt{
		M\ar@{.>}[dr]_{\exists! \theta}\ar@/_-10pt/[rrd]^{\pi_m}\ar@/_+10pt/[rddd]_{m}\\
		&\phi^*(\phi_!(M))\ar@{}[ddr]|{\text{P.B.}}\ar[r]^-{p_{\phi_!(m)}}\ar[dd]_{\phi^*(\phi_!(m))}&\phi_!(M)\ar[dd]^{\phi_!(m)}\\
		\\
		&X\ar[r]_\phi&Y
	}
	\]\qedhere
	\end{proof}

\begin{remark}\label{comp}
	Note that, if $\phi\colon X\to Y$ is itself a mono, then $\phi_!([m])$ is simply $[\phi\circ m]$ for every $[m]\in \sub{X}(X)$.
\end{remark}

\begin{proposition}\label{lim_inf}
	Let $\{[m_i]\}_{i\in I}$ be a subset of $\sub{X}(C)$ and suppose that there exists a limit for the diagram given by the arrows $\{m_i\}_{i\in I}$, then $\{[m_i]\}_{i\in I}$ has an infimum.
\end{proposition}
\begin{proof}
By definition of limit, for every $i\in I$ we have a triangle
\[\xymatrix{M\ar[r]^{l_i}\ar[dr]_{m} & M_i\ar[d]_{m_i}\\ & X}\]	
where $M$ is the vertex of the limiting cone and $l_i$ and $m$ are its structural arrows. $m$ is monic: if $m\circ f=m\circ g$, then, for every $i\in I$, $m_i\circ l_i\circ f=m_i\circ l_i \circ g$ and thus, since every $m_i$ is a mono, $l_i\circ f=l_i\circ g$ and thus, $f=g$. Clearly $[m]\leq [m_i]$ for every $i$. Let $[n]$ be another lower bound, with $n:N\rightarrow X$, then there must be $k_i:N\rightarrow M_i$ such that, for every $i\in I$, $m_i\circ k_i=n$ and thus there exists $\phi:N\rightarrow M$ such that $l_i\circ \phi = k_i$, composing with any $m_i$ we get $m\circ \phi =n$, i.e. $[n]\leq [m]$.
\end{proof}

In particular if $\X$ satisfies hypotheses (Hyp. 1--4) then $\sub{X}(X)$ has binary infima computed by pullbacks (take $I$ to be ${1,2}$). Furthermore, proposition $\ref{limcolim}$ implies that  $\sub{X}(X)$ is a complete lattice whenever $(\X, |-|)$ is monadic over $\Set$.

Suprema of subobjects, when they exist, are not computed as colimits, nevertheless they enjoy a similar property.

\begin{lemma}\label{jointly_epic}\marginpar{spostato qui un lemma dalla sottosezione epi} Let $\X$ be a category with equalizers and $\left\{[m_i]\right\}_{i\in I}$ be a subset of $\sub{\X}(X)$ with a supremum $[m]$ represented by some $m\colon M\rightarrow X$. For each $i\in I$, let $k_i\colon M_i\rightarrow M$ be a witness of $[m_i]\leq [m]$ ($M_i$ is the domain of $m_i$). Then, for any $\phi,\psi:M\rightrightarrows Y$,  if $\phi\circ k_i=\psi\circ k_i$ for every $i\in I$ then $\phi=\psi$.
\end{lemma}
\begin{proof} Let $e\colon \eq(\phi,\psi)\rightarrow M$ be the equalizer of $\phi$ and $\psi$, clearly $[m\circ e]\leq [m]$. By the universal property of the equalizer, for each $i\in I$ there is a morphism $e_i\colon M_i\to \eq(\phi,\psi)$ such that $e\circ e_i=k_i$, and, composing with $m$, this implies $[m_i]\leq [m\circ e]$. Since $[m]=\sup\left\{[m_i]\right\}_{i\in I}$ then $[m]\leq [m\circ e]$ and thus $[m]=[m\circ e]$. We conclude that $e$ is an isomorphism and so $\phi=\psi$ by Corollary \ref{equalizer_of_equals}.
	%
	%By hypothesis for any $i\in I$ there exists $k_i:M_i\rightarrow E$ such that $k\circ k_i=j_i$, so, since $k$ is a (regular) mono,  $[m\circ k]\geq [m_i]$ in $\sub{\X}(X)$, this implies $[m\circ k]\geq [m]$. On the other hand $[m]\geq [m\circ k]$ by construction, so $[m]=[m\circ k]$ and $k$ is an isomorphism, so $u=v$.
	%			\begin{center}
	%				\begin{tikzpicture}
	%				\node(A)at(0,0){$X$};
	%				\node(B)at(2.5,0){$M$};
	%				\node(C)at(0,-2){$M_i$};
	%				\node(D)at(2.5,-2){$E$};
	%				\node(E)at(5,0){$Y$};
	%				\draw[<-](A)--(C)node[pos=0.5, left]{$m_i$};
	%				\draw[<-](B)--(D)node[pos=0.5, right]{$k$};	
	%				\draw[<-](A)--(B)node[pos=0.5, above]{$m$};
	%				\draw[->](C)--(D)node[pos=0.5, below]{$k_i$};	
	%				\draw[->](C)--(B)node[pos=0.5, above, xshift=-0.1cm, yshift=-0.1cm]{$j_i$};
	%				\draw[->](B.20)--(E.157)node[pos=0.5, above]{$u$};
	%				\draw[->](B.340)--(E.203)node[pos=0.5, below]{$v$};
	%				\end{tikzpicture}
	%			\end{center}	
\end{proof}

\begin{notaz}We will use the same convention used for $\phi^*$ and $\phi_!$: given two subobjects $[m]$ and $[n]$ represented by $m:M\rightarrow X$ and $n:N\rightarrow X$ we will denote by
	\[
	m\wedge n: M\wedge N\rightarrow X \hspace{5pt}(\text{resp. } m\vee n:M\vee N\rightarrow X)
	\]
any representative of their infimum (resp. supremum) in $\sub{\X}(X)$.
\end{notaz}

\subsection{Closure operators}\marginpar{Uniformato la notazione}
Let's introduce the notion of closure operator as defined in \cite{DT}.

\begin{definition}
Let $\X$ be a category, a {\em closure operator} $c$ is a collection $\{c_X:\sub{\X}(X)\rightarrow \sub{\X}(X)\}_{X \in {\X}}$ such that: 
\begin{enumerate}[\rm {(CO.}1)]
\item $[m] \leq c_X([m])$, for all $[m]\in \sub{\X}(X)$\quad [{\em expansion}]; 
\item if $[m] \leq [n]$ in $\sub{\X}(X)$, then $c_X([m]) \leq c_X([n])$\quad [{\em monotonicity}]; 
\item $\phi_!([c_X(m)]) \leq c_Y([\phi_!(m)])$, for all $\phi\colon X  \to  Y$	 and $[n]\in \sub{\X}(X)$\quad [{\em continuity}].
\end{enumerate} 
A subobject $[m]\in \sub{X}(X)$ in $\X$, is said to be {\em $c$-closed} (resp., {\em $c$-dense}) in $X$ if $c_X([m])=[m]$ (resp., $c_X([m])=[\id{X}]$).
\end{definition}
\begin{notaz}We will adopt the same rule used for $\phi^*$ and $\phi_!$ denoting by $c_X(m):c_X(M)\rightarrow X$ any representative of $c_X([m])$. In particular $[c_X(m)]=c_X([m])$. We will also denote by $j_m$ a monic $M\rightarrow c_X(M)$ witnessing $[m]\leq c_X([m])$.
\end{notaz}
Before starting a brief discussion of these three axioms defining the notion of a closure operator, let us
mention just three immediate examples. 


\begin{example}\label{ex_CO_easy}
The first two examples below make sense in any finitely bicomplete category $\X$ where the class of monomorphisms satisfies the properties (1--4) of Lemma~\ref{prop_monos_lemma}:
\begin{enumerate}[\rm (1)]
\item the {\em trivial} closure operator $t$ is defined by $t_X([m]):=[\id{X}]$, for all monics $m:M\rightarrow X$ in $\X$. In other words, $t$ is the unique closure operator with respect to which all subobjects are dense;
\item  the {\em discrete}  closure operator $d$ is defined by $d_X([m]) := [m]$, for all monics $m:M\rightarrow X$ in $\X$. In other words, $d$ is the unique closure operator with respect to which all subobjects are closed;
\item the prototypical example, that gives the name to closure operators, can be found in the category of topological spaces $\Top$. This is the {\em Kuratowski closure operator} $k$, such that $k_X([m]):=[\iota_m]$, where, given a monic $m:M\rightarrow X$, $\iota_m$ is the inclusion of the topological closure $\overline{m(M)}$ of the image of $m$. In this case, $k$-closed ($k$-dense) means exactly that the image is closed (dense) in the topological sense.
\end{enumerate}
\end{example}
\begin{remark}
	Consider now the axioms (CO.1--3) that define a closure operator: while expansion and monotonicity are rather mild and natural conditions, the continuity axiom may look somewhat strange, especially as far as the name ``continuity" is concerned. To better illustrate this axiom, let us analyze it in the particular case of the Kuratowski closure operator $k$ in $\Top$. Indeed, consider two topological spaces $X$ and $Y$, and a function $\phi\colon X\to Y$. The usual definition of continuity for $\phi$ asks that, for each open subset $A\subseteq Y$, we have that $\phi^{-1}(A)$ is open in $X$. On the other hand, it is not difficult to verify that this is equivalent to ask that $\phi(\overline{M}) \subseteq  \overline{\phi(M)}$, for each $M \subseteq X$, so justifying the name of the axiom.
\end{remark}
%%, i.e., $f$ is {\em $K$-continuous}
%in the sense of the above continuity axiom. Moreover, one can easily prove that every $K$-continuous map $f : X \to Y$ between the topological spaces $X, Y$ is 
%continuous. This phenomenon in $\Top$ motivated the use of the term ($C$-)continuity of the axiom. 

In the following corollary we give an alternative formulation of the continuity axiom:

\begin{proposition}\label{eq}
Consider a collection $\{c_X\colon \sub{\X}(X)\to \sub{\X}(X)\}_{X\in\X}$ that satisfies (CO.1) and (CO.2). Then, the following are equivalent for a morphism $\phi\colon X\to Y$ in $\X$:
\begin{enumerate}[\rm (1)]
\item $\phi_!(c_X([m]))\leq c_Y(\phi_!([m]))$, for all $[m]\in \sub{X}(X)$;
\item $c_X(\phi^*([n]))\leq \phi^*(c_Y([n]))$, for all $[n]\in \sub{\X}(Y)$.
\end{enumerate}
\end{proposition}
\begin{proof}
(1)$\Rightarrow$(2). Take a monic $n\colon N\to Y$, then (1) yields
\[\phi_!(c_X(\phi^*([n])))\leq c_Y(\phi_!(\phi^*([n])))\]
 By the adjunction $\phi_!\dashv \phi^*$ we also have  $\phi_!(\phi^*([n]))\leq [n]$, hence, by monotonicity
\begin{align*}
\phi_!(c_X(\phi^*([n])))&\leq c_Y(\phi_!(\phi^*([n])))\leq c_Y([n])
\end{align*}
The thesis now follows exploting again the adjunction $\phi_!\dashv \phi^*$.

\smallskip\noindent
(2)$\Rightarrow$(1). Start with a monic $m\colon M\to X$ then we can use the hypothesis and the inequality $[m]\leq \phi^*(\phi_!([m]))$ to get
\begin{align*}
c_X([m])&\leq c_X(\phi^*(\phi_!([m])))\leq \phi^*(c_Y(\phi_!([m]))) 
\end{align*} 
and again we can conclude since  $\phi_!\dashv \phi^*$.
\end{proof}





%Roughly speaking, a closure operator $C$ can be defined in a category $\X$ such that to every object $X\in \X$
%is assigned a family $\mathcal M_X$ of monomorphisms $m:M \to X$, called  $\mathcal M$-subobjects
%(or, briefly, subobjects) of $X$, ordered by $m\leq m'$ if $m = m'\cdot f$ ($f$ is necessarily a monomorphism as $\mathcal M$ consists of monos).  
%It is convenient to ask the following 
%
%(i) each $\mathcal M_X$  is a complete lattice (with top element $id_X$); 
%
%(ii) each $f\colon X \to Y$ in $\X$ (uniquely) factorizes as $f = m_f \circ e_f$, where $e_f : X \to f(X)$ is an epimorphism and
%$m_f: f(X)\to X$ is an $\mathcal M$-subobject of $X$. 
%
%This allows for the definition of an ``$\mathcal M$-image" of a subobject $\alpha: M \to X$ in $\mathcal M_X$ under a morphism $f: X \to Y$,
%(denoted by $f(M)$) as the $m$-part $m_{f\circ \alpha}\colon M\to (f\circ \alpha) (M)$ of the factorization of $f\circ \alpha\colon M \to Y$ (see \cite[Chapter 1]{DT} for a more rigorous outline of the setting). 


\subsubsection{Idempotency and weakly hereditariness}\marginpar{Aggiunta la parte sull'ereditarietà per semplificare la dimostrazione di epi = reg}
A part from the defining axioms (CO.1--3), there are two important propertirs satisfied by many important closure operators: idempotency and weakly hereditariness. For example, all the three closure operators of Example~\ref{ex_CO_easy}. 

\begin{definition}
	Let $c$ be a closure operator on a category $\X$. We say that $c$ is
	\begin{itemize}
		\item \emph{idempotent} if, for each $[m]\in \sub{\X}(X)$,
		\begin{equation*}
		c_X(c_X([m]))= c_X([m])
		\end{equation*} 
		\item \emph{weakly hereditary} if, for each $[m]\in \sub{\X}(X)$,
		\begin{equation*}
			c_{c_X(M)}([j_m])=[\id{c_X(M)}]
		\end{equation*}		
	\end{itemize} 
In other words, a closure operator is idempotent if the closure of any subobject is closed and weakly hereditary if every subobject is dense in its closure.
\end{definition}

In the following lemma we show a different, but equivalent, characterization of idempotency for a given closure operator. This different formulation is often useful because it suggests a strategy to define new closure operators: for each $M\leq X$ in $\X$ one can try to define the closure of $M$ in $X$ as the smallest subobject of $X$, containing $M$, and having a specified property. If this work, the resulting closure operator will be idempotent.

\begin{lemma}\label{char_idem} 
Let $c$ be a closure operator on $\X$ and, for each $[m]\in \sub{\X}(X)$ define:
\begin{align*}
\mathcal{F}([m])&:=\left\{[n]\in  \sub{\X}(X) \mid [m]\leq [n] \text{ and } c_X([n])=[n]\right\}
\end{align*}
Then, $c$ is idempotent if and only if, for every $[m]$, $\inf(\mathcal{F}([m]))$ exists and it is equal to $c_X([m]).$
\end{lemma} 
\begin{proof} $(\Rightarrow)$ If $c$ is idempotent then, $c_X([m])\in \mathcal{F}([m])$ for every $[m]\in \sub{\X}(X)$. Moreover, if $[n]\in \mathcal{F}([m])$,by (CO.2),  one has $c_X([m])\leq c_X([n])$, but $c_X([n])=[n]$ and we can conclude.

\noindent $(\Leftarrow)$ We have to verify that $c_X([m])$ is $c$-closed for any $[m]\in \sub{\X}(X)$. If we show that $\mathcal{F}([m])= \mathcal{F}(c_X([m]))$ we are done: in such case our hypotheses imply
\[c_X(c_X([m]))=\inf(\mathcal{F}(c_X([m])))=\inf(\mathcal{F}([m]))=c_X([m]) \]
Let $[n]\in \mathcal{F}(M)$, then $[n]$ is $c$-closed and (CO.2) entails \[c_X([m])\leq c_X([n])=[n]\]
 thus, $[n]\in \mathcal{F}(c_X([m]))$. On the other hand if $[n]\in \mathcal{F}(c_X([m]))$, we can use (CO.1) to get  
 \[[m]\leq c_X([m])\leq c_X([n])=[n]\]
 proving that $[n]\in \mathcal{F}(M)$.
\end{proof}

\begin{lemma}\label{char_epi}
	Let $c$ be a closure operator on $\X$ and, for each $[m]\in \sub{\X}(X)$ define:
	\begin{align*}
	\mathcal{G}([m])&:=\left\{[n]\in  \sub{\X}(X) \mid m=n\circ k \text{ for some } k \text{ such that } [k] \text{ is }  c\text{-dense}\right\}
	\end{align*}
	Then, $c$ is weakly hereditary if and only if, for every $[m]$, $\sup(\mathcal{G}([m]))$ exists and it is equal to $c_X([m]).$
\end{lemma}
\begin{proof}
$(\Rightarrow)$ Let, $m:M\rightarrow X$ be a mono, by (CO.1) $[m]\leq c_X([m])$ and so there exists $j_m:M\rightarrow c_X(M)$ such that $m=c_X(m)\circ j_m$. Since $c$ is weakly hereditary, $[j_m]$ is $c$-dense, hence $c_X([m])\in \mathcal{G}([m])$. On the other hand, if $[n]\in \mathcal{G}([m])$ we can use (CO.3) and Remark \ref{comp} to get
\[[n]=[n\circ \id{N}]=n!([\id{N}])=n_!([c_N(k)])=n_!(c_N[k])\leq c_{X}(n_!([k]))=c_X([m])\]
\noindent
$(\Leftarrow)$ If we show that $c_X([m])\leq[c_X(m)\circ c_{c_X(M)}(j_m)]$ in $\sub{\X}(X)$ we are done. Notice that, since $c_{X}(m)_!$ is a left adjoint, the subobject on the right is the supremum of the family
\[\mathcal{A}:=\left \{[c_X(m)\circ n] \mid [n] \in \mathcal{G}([j_m]) \right \}\subseteq \sub{\X}(X)\]
Take now $n:N\rightarrow X$ such that $[n]\in \mathcal{G}([m])$, by definition $m=n\circ k$ for some $k$ such that $c_N([k])=[\id{N}]$, but our hypotheses on $c$ imply that $[n]\leq c_X([m])$, so there exists also $h:N\rightarrow C_X(M)$ such that $c_X(m)\circ h=n$, thus
\[c_X(m)\circ j_m=m=n\circ k=c_X(m)\circ h\circ k\]
and thus $j_m=h\circ k$, therefore $h\in \mathcal{G}([j_m])$ and we can conclude since this entails $[n]\in \mathcal{A}$. 
\end{proof}

\subsection{Two closure operators} \marginpar{Ho riorganizzato questa sezione cambiando l'introduzione e mettendo prima epi}
In this section we introduce two closure operators, $\epi$ and $\reg$, related to the epimorphism problem. The first one arises ``enlarging'' a subobject $[m]$ in such a way that the inclusion $m\rightarrow \epi_{X}(m)$ remains an epimorphism. The second one is given ``eneveloping'' a subobject with all the regular subobjects (i.e. those with regular representatives) that contain it. At the end of this section we will show when these two approaches coincide.



\subsubsection{The $\mathsf{epi}$ closure operator}

 In the following lemma we collect a useful property regarding the interaction of epimorphisms with infima an suprema in $\sub{\X}(X)$.


\begin{lemma}\label{lem_tech_before_epi}
	Let $\X$ be a category satisfying (Hyp. 1--4), $m\colon M\rightarrow X$ and $n\colon N\rightarrow Y$ be two monomorphisms and $\phi\colon X\rightarrow Y$ another arrow. Let also $k:M\wedge \phi^*(N)\to M$ and $j:N\to \phi_!(M)\vee N$ be arrows witnessing, 
	respectively, $[m]\wedge \phi_*([n])\leq [m]$ and  $[n]\leq [n]\vee \phi_{!}([m])$.
	Then if $k$ is epi, so is $j$. 
\end{lemma} 
\begin{proof}We have the following commutative diagram:
	\[
	\xymatrix@C=30pt@R=20pt{
		M\wedge\phi^*(N)\ar[d]_{q}\ar[r]^{k}&M\ar[d]^{\pi_m}\\
		\phi^*(N)\ar[d]_{p}&\phi_!(M)\ar[d]^{l}\\
		N\ar[r]_{j}&N\vee\phi_!(M)
	}
	\]
	Let $u,\, v\colon \phi_!(M)\vee N\rightrightarrows Y$ be a pair of parallel morphisms   such that $u\circ j =v \circ j$. By the commutativity of the above square we deduce that
	\[
	u\circ l\circ \pi_m\circ k=u\circ l\circ p\circ q =u\circ l\circ p\circ q=v\circ l\circ \pi_m\circ k
	\]
	and, since both $k$ and $\pi_m$ are epimorphisms, we get that $u\circ l=v\circ l$. Now we can conclude invoking Lemma~\ref{jointly_epic}.
	%
	%, by the claim it is enough to show that $u\circ t=v\circ t$. Computing we have:
	%\begin{align*}
	%u\circ t\circ e_m \circ i &=u\circ k \circ p_N\circ j\\&=v\circ k \circ p_N\circ j\\&v\circ t\circ e_m \circ i
	%\end{align*}
	%with $i$ epi, so 
	%\begin{equation*}
	%u\circ t\circ e_m=v\circ t\circ e_m
	%\end{equation*}
	%and the thesis now follow since $e_m$ is an epimorphism.
\end{proof}

\begin{definition}\label{epi_def}
	Let $\X$ be a category, $X\in \X$ and $[m]$ in $\sub\X(X)$. Consider the following family:
	\[
	\G([m]):=\left\{[k]\in\sub\X(X) \mid k\circ s=m\ \text{ for some epi } s\right\}.
	\]
	We say that $\epi$ is {\em well-defined} if the supremum $\epi_X([m]):=\sup (\G([m]))$ exists in $\sub\X(X)$, for all $X\in \X$ and $[m]\in \sub\X(X)$.
\end{definition}

\begin{remark}
	The $\epi$-closure $\epi_X([m])$ is defined as a supremum of a possibly infinite family in $\sub\X(X)$ so, to ensure that $\epi$ is well-defined, one a priori needs the completeness of $\sub{\X}{X}$, but by Proposition \ref{lim_inf} this additional hypothesis is satisfied whenever $\X$ has arbitrary limits. Furthermore, Proposition \ref{limcolim} guarantees that this is true in categories monadic over $\Set$, in particular, $\sub{\X}(X)$ is complete in any category of ``of algebraic origin''.
\end{remark}

Let us show now that, if $\X$ satisfies our hypotheses (Hyp. 1--4) and epi is well-defined then we get a closure operator.

\begin{theorem}\label{main_thm_epi}
	Let $\X$ be a category that satisfies (Hyp.1--4) and suppose that $\epi$ is {well-defined} in $\X$. 
	Then, $\mathsf{epi}$ is a closure operator. 
\end{theorem}
\begin{proof}
	Let us verify the three axioms in the definition of a closure operator:
	\begin{enumerate}[\rm {(CO.}1)]
		\item	Let $X\in \X$ and take a monomorphism $m\colon M\rightarrow X$. Then $m=m\circ \id{M}$ and $\id M$ is an epimorphism, so $[m]\in \mathcal{G}([m])$ and $[m]\leq \epi_{X}(M)$. 
		\item Consider two monomorphisms $m\colon M\to X$ and $n\colon N\to X$, and a morphism $r\colon M\to N$ such that $m=n\circ r$ witnessing $[m]\leq [n]$ in $\sub\X(X)$. Suppose now that $k\colon K\rightarrow X$ is a monomorphism such that $m=k\circ s$, where $s$ is epic, so that $[k]\in \mathcal{G}([m])$. Computing $[k]\wedge [n]$ we get the following commutative solid diagram from Proposition \ref{lim_inf}:
		\[
		\xymatrix@C=60pt@R=15pt{
			&K\wedge N\ar@{}[ddr]|{\text{P.B.}}\ar[dd]_{l_K}\ar[r]^{l_N}&N\ar[dd]^n\\
			M\ar[rd]|s\ar@{.>}[ru]|{\exists!\, t\ }\ar@/_30pt/[rrd]_m\ar@/_-30pt/[rru]^r\\
			&K\ar[r]_{k}&X
		}
		\]
		where, by the universal property of the pullback, there exists a unique morphism $t\colon M\to K\wedge N$ making the whole diagram commute. Now, $l_k\circ t=s$ is epic, so $l_k$ is epic too by Proposition \ref{composition} and Lemma~\ref{lem_tech_before_epi} (applied with $\phi=\id X$), gives us that $j_N\colon N\to K\vee N$, witnessing $[n]\leq [k]\vee [n]$, is an epimorphism. Now, $(k\vee n)\circ j_N=n$, so $[k\vee n]\in \mathcal{G}([n])$ and we can conclude that
		\[[k]\leq [k\vee n] \leq \epi_{X}([n])\]
		Since $[k]$ is an arbitrary element of $\mathcal{G}([m])$, this shows that $\epi_X([m])\leq \epi_X([n])$. 
		\item Let $m\colon M\to X$ be monic and $\phi\colon X\rightarrow Y$ be a morphism; we have to verify that
		\[\phi_!(\epi_{X}(M))\leq\epi_{Y}({\phi_!(M)})\]
		 Indeed, take a factorization $m=k\circ s$, with $k\colon K\to X$ monic and $s$ epic. Applying $\phi_!$, we obtain the following picture:
		\[
		\xymatrix{
			M\ar[rrrr]^-{\pi_{m}}\ar[dr]^s\ar[dd]_m&&&& \phi_!(M)\ar[dd]^{\phi_!(m)}\ar[dl]_{\phi_!(s)}\\
			&K\ar[dl]^k\ar[rr]^{\pi_{k}}&&\phi_!(K)\ar[dr]_{\phi_!(k)}\\
			X\ar[rrrr]_-\phi&&&&Y
		}
		\]
		Hence,
		\[
		\phi_!(k)\circ (\phi_!(s)\circ \pi_{m})=(\phi_!(k)\circ \phi_!(s))\circ \pi_{m}=\phi_!(m)\circ \pi_{m}=\phi_!(k)\circ (\pi_{k}\circ s)
		\]
		and, since $\phi_!(k)$ is monic, $\phi_!(s)\circ \pi_{m}=\pi_{k}\circ s$. As both $s$ and $\pi_{k}$ are epic,
		$\phi_!(s)\circ \pi_{m}$ is epic too (it being equal to $\pi_{k}\circ s$), therefore, by Proposition \ref{composition}, $\phi_!(s)$ is an epimorphism. Hence, $\phi_!([k])\in \mathcal{G}([\phi_!([m])])$ and thus
		\[\phi_!([k])\leq \epi_Y(\phi_!([m]))\]
		Since this is true for every $[k]\in \mathcal{G}([m])$ we have that
		\[
		\sup\left\{\phi_!([k]) \mid [k]\in \mathcal{G}([m]) \right\}\leq \epi_Y(\phi_!([m])).
		\]
		but $\phi_!$ is a left adjoint, so it commutes with suprema, and thus that the term on the left of the above inequality is just $\phi_!(\epi_\X([m]))$.\qedhere 
		\end{enumerate}
\end{proof}

The closure operator $\epi$ is defined as a supremum, so it is natural to ask if we can apply Lemma \ref{char_epi} tho show that it is weakly hereditary. This is possible as shown by the next propositions.
\begin{proposition}\label{coro_map_to_epi}
	Let $\X$ be a category which satisfies (Hyp. 1--4) and such that $\epi$ is {well-defined}. Then, for each $[m]\in \sub{\X}(X)$ in $\X$, the canonical morphism $j_m\colon M\to \epi_X(m)$ is an epimorphism.
\end{proposition}
\begin{proof}
	Consider a pair of parallel arrows $a,\,b\colon \epi_X(m)\rightrightarrows Y$ such that $a\circ\varphi=b\circ\varphi$. For each $[k]\in \G([m])$, with $k\colon K\to X$,  we have a factorization $m=s_k\circ k$, with $s_k$ epic. Moreover $[k]\leq \epi_{X}([m])$, so we get the following commutative diagram:
	\[
	\xymatrix@C=40pt@R=10pt{
		M\ar[rd]_{s_k}\ar[rr]^-{j_m}&&\epi_X(m)\ar@<-.5ex>[r]_-{b} \ar@<.5ex>[r]^-{a}&Y\\
		&K\ar@{>}[ur]_-{t_k}
	}
	\]
	where $t_k$ exists by definition of $\epi_X(M)$. We obtain that \[
	a\circ t_k\circ s_k=a\circ j_m=b\circ j_m=b\circ t_k\circ s_k\]
	and, since $s_k$ is epic this implies that $a\circ t_k=b\circ t_k$. Now we conclude that $a=b$ from Lemma~\ref{jointly_epic}.
\end{proof}

\begin{remark}Since $j_m:M\rightarrow \epi_{X}(m)$ always exists, the previous proposition holds under the weaker hypotheses that $\X$ has equalizers and $\epi$ is well defined.
\end{remark}

\begin{corollary}
	The class of a monomorphism $m:M\rightarrow X$ is $\epi$-dense if and only if $m$ is epic.
\end{corollary}
\begin{proof}
	($\Rightarrow$) This follows at once from Proposition \ref{coro_map_to_epi}.
	
	\smallskip \noindent
	($\Leftarrow$) If $m$ is epic then $[\id{X}]$ belongs to $\mathcal{G}([m])$ and we are done.
\end{proof}
Now Lemma \ref{char_epi} gives us the following corollary at once.
\begin{corollary}\label{wh}
	In a category $\X$ which satisfies (Hyp. 1--4), if $\epi$ is well defined, then it is is weakly hereditary.
\end{corollary}


\subsubsection{The $\mathsf{reg}$ closure operator}

The $\epi$ closure operator is defined as a supremum and we do not have an explicit formula to compute suprema in $\sub{\X}(X)$, while, thanks to Proposition \ref{lim_inf} we have one for computing infima. So we introduce another closure operator which can be more easily computed.
\begin{definition}\label{def_reg_locale}
Given a monomorphism $m\colon M\rightarrow X$ in a category $\X$ satisfying (Hyp. 1--4), we define
$\reg_{X}([m]):=[\im{m}]$.
\end{definition}

The rest of the subsection will be devoted to prove that, under our usual hypotheses (Hyp.1--4), the above definition gives a well-defined idempotent closure operator on $\X$.

\begin{remark}
One could be tempted, in analogy with Definition~\ref{def_reg_locale}, to use the coimage to similarly define a closure operator. On the other hand, this would just produce the discrete closure operator, in fact, for each monomorphism $m\colon M\to X$ in $\X$,  the canonical morphism $M\to \Coim(m)$ is both a monomorphism and a regular epimorphism and, therefore, it is an isomorphism, showing that $[m]=[\Coim(m)]$.
\end{remark}


%\begin{remark} Notice that from $m\circ u=\im{f}$ and $\Coim{f}=u\circ e$ we get 
%	\begin{equation*}
%		\begin{split}
%		m\circ e=f=m\circ u\circ e_f
%		\end{split} \qquad 
%		\begin{split}
%		m\circ e=f=m_f\circ u \circ e
%		\end{split}
%	\end{equation*} Since $m$ is mono and $e$ is epi we can also conclude that $u\circ e_f=e$ and $m_f\circ u =m$.
%\end{remark}

\begin{theorem}\label{thm_reg} 
Let $\X$ be a category satisfying (Hyp.1--4). Then, $\mathsf{reg}$ is an idempotent closure operator.
\end{theorem}
\begin{proof}
Let us prove first that $\reg$ is a closure operator, verifying the three defining axioms:
\begin{enumerate}[\rm {(CO.}1)]
\item Given a monomorphism $m\colon M\to X$ in $\X$ and consider the factorization $m=m'_m\circ e'_m$ of $m$ through its image. This shows that $[m]\leq [\Im(m)]=\reg_X([m])$.
\item Consider two monomorphisms $m\colon M\to X$ and $n\colon N\to X$, and a morphism $h\colon M\to N$ such that $n\circ h=m$, that is, $[m]\leq [n]$. Take now the factorization $n=m'_n\circ e'_n$ of $n$ through $\Im(n)$, and note that we get a factorization \[m=(m'_n\circ e'_n)\circ h=m'_n\circ (e'_n\circ h)\] with $m'_n$ a regular monomorphism. By Proposition~\ref{univ_prop_facts_lemma}, there exists a (unique) morphism $u'\colon \Im(m)\to \Im(n)$ such that $m'_m=m'_n\circ u'$. Therefore, $\reg_X(M)\leq \reg_X(N)$.
\item Given a morphism $\phi\colon X\rightarrow Y$ and a monomorphism $n\colon N\rightarrow Y$, the factorization $n=m'_n\circ e'_n$ of $n$ through $\Im(n)$, induces a factorization $\phi^*(n)=\phi^*(m'_n)\circ \phi^*(e_n')$.
% shows that $N\leq \Im(n)$. Hence, $\phi^*(N)\leq \phi^*(\Im(n))$, so there exists a morphism $h\colon \phi^*(N)\to \phi^*(\Im(n))$ such that $\phi^*(n)=\phi^*(m'_n)\circ h$. 
 By Lemma~\ref{reg}, $\phi^*(m'_n)$ is a regular monomorphism and so, by Proposition~\ref{univ_prop_facts_lemma}, there exists a (unique) morphism $u'\colon \Im(\phi^*(n))\to \phi^*(m'_n)$ such that $m'_{\phi^*(n)}=\phi^*(m'_n)\circ u'$. Therefore, $\reg_X(\phi^*([n]))\leq \phi^*(\reg_Y([n]))$.
%
%we have a diagram
%			\begin{center}
%				\begin{tikzpicture}
%				\node(A)at(0,4){$\phi^*(N)$};
%				\node(B)at(3,2){$\im{N}$};
%				\node(C)at(0,0){$X$};
%				\node(D)at(3,0){$Y$};
%				\node(E)at(3,4){$N$};
%				\node(F)at(0,2){$\phi^*(\im{N})$};
%				\node(G)at(-3,2){$\im{\phi^*(N)}$};
%				\draw[->](A)--(F)node[pos=0.5, left]{$j$};
%				\draw[->](B)--(D)node[pos=0.5, left]{$\im{n}$};	
%				\draw[->](C)--(D)node[pos=0.5, below]{$f$};	
%				\draw[<-](E)--(A)node[pos=0.5, above]{$p_N$};
%				\draw[->](E)--(B)node[pos=0.5, right]{$e_n$};			
%				\draw[<-](C)--(F)node[pos=0.5, left]{$\phi^*(\im{n}) $};
%				\draw[->](F)--(B)node[pos=0.5, above]{$p_{\im{N}} $};
%				\draw[->](E)..controls(4.5,3.5)and(4.5,0.5)..(D)node[pos=0.5, right]{$n$};
%				\draw[->](A)--(G)node[pos=0.5, above, xshift=-0.25cm]{$e_{\phi^*(n)}$};
%				\draw[dashed,<-](F)--(G)node[pos=0.5, above]{$u$};
%				\draw[->](G)..controls(-3,1)and(-2,0)..(C)node[pos=0.5, left, xshift=-0.1cm ]{$\im{\phi^*(n)}$};
%				\end{tikzpicture}
%			\end{center} 
%in which the outer and the below squares are pullbacks. By Lemma~\ref{reg} $\phi^*(\im{n})$ is a regular mono, so the dashed $u$ exists by fact number $3$. 
\end{enumerate}
Idempotency of $\reg$ follows directly from Corollary~\ref{reg2}.
\end{proof}

From Lemma \ref{char_idem} and Corollary $\ref{reg2}$ we can deduce the following corollary (explaining the name $\mathsf{reg}$):

\begin{corollary}\label{supinf} 
Given a monomorphism $m\colon M\to X$ in $\X$,
\begin{equation*} 
\rg{X}{m}=\inf\left\{[n]\in \sub{\X}(X) \mid [m]\leq [n] \text{ and } n \text{ is regular}\right\}.
\end{equation*}
\end{corollary}

\begin{remark}\label{clos_coim}Let $\phi:X\rightarrow Y$ be a morphism in a category $\X$ satisfying (Hyp. 1--4), the previous corollary with Lemma \ref{coim_im} and Proposition \ref{univ_prop_facts_lemma} gives us that 
	\[\reg_Y([m_\phi])=m'_\phi\]
\end{remark}

\begin{example}\marginpar{Nel terzo punto c'è il solito problema di nomenclatura: quella che definisco è davvero la purificazione di un sottogruppo? È di sicuro puro, ma soddisfa una proprietà molto più forte (cioè che se $nx$ gli appartiene allora anche $x$ gli appartiene)}
	We can use this last characterization to provide some concrete example.
	
	\begin{enumerate}[\rm (1)] 
		\item Take $\X$ to be the category $\Haus$ of Hausdorff topological space. Then a subspace of $X$ is an equalizer if and only if it is closed, so in this case $\reg$ is just the Kuratowski closure operator of Example \ref{ex_CO_easy}.
		
		\item It is possible to compute $\reg$ even for the category $\Ring$ of rings (with unit). Let $m:R\rightarrow S$ be a monomorphism, then $S$ has a structure of $R$-algebra and we can take the tensor product $S\otimes_R S$. The \emph{epicenter of $m$} is the subring $\ec{S}{m}$ given by the equalizer of the two inclusions $S\rightrightarrows S\otimes_R S$. It holds that $\reg_{S}([m])$ coincides with the class of the inclusion $\ec{S}{m}\rightarrow S$ (see \cite{epiring1,epiring2})
		
		\item Let $\X$ be the category $\TFA$ of torsion free abelian groups. Then a subgroup $H$ of a group $G$ is an equalizer if and only if  $g\in H$ whenever $g^{n}\in H$. Thus $\reg_{G}([m])$ is the class of the inclusion in $G$ of the subgroup
		\[
	m(M)_*=\{g\in G \mid g^{n}\in m(M) \text{ for some } n \in \mathbb{Z}\}	
		\]
		i.e. the inclusion of the purification of $m(M)$ into $G$.
	\end{enumerate}
 \end{example}

Notice that, \emph{a posteriori}, examples $(2)$ and $(3)$ above, show that the construction of the epicenter of a subring and the purification of a subgroup can be seen as closure operators over $\Ring$ and $\TFA$ respectively.

\normalmarginpar 

\subsection{An application to the epimorphism problem } \marginpar{Ho cambiato il titolo di queste ultime due sezioni}

Arrived at this point of the section, we have studied two closure operators on a suitable category $\X$, with the hope that they will be useful for the epimorphism problem. Before proceeding further we can show when, under our working hypotheses,  $\epi=\reg$. 

\begin{proposition}\label{whe}Let $\X$ be a category which satisfies (Hyp.1--4) and in which every strong mono is regular, then the following hold true
	\begin{enumerate}[\rm (1)]
		\item $\reg$ is weakly hereditary; 
		\item a subobject $[m]$ is $\reg$-dense if and only if $m$ is epic. 
	\end{enumerate}
 \end{proposition}
\begin{proof} \begin{enumerate}[\rm (1)]
		\item Let $[m]\in \sub{\X}(X)$ where $m:M\rightarrow X$. We can factorize it through its image and further factorize $M\rightarrow \im{m}$ to get the square
		\[
		\xymatrix{\im{e'_m}\ar[r]^-{n} & \im{m}\ar[d]^-{m'_m}\\
			M \ar[r]_{m} \ar[u]^{k} \ar[ur]^{e'_m}& X}
		\]
		where $n$ and $m'_m$ are regular mono and, by Theorem \ref{thm_main}, $e'_m$ and $k$ are epis. By Proposition \ref{composition} $n$ is epi too and thus an isomorphism by point (1) of Lemma \ref{extremal_prop_lemma}.
		\item  $(\Rightarrow)$ Suppose that $[m]$ is dense and factorize it through $\im{m}$ as $m'_m\circ e'_m$, by hypothesis $m'_m$ is an isomorphism and thus $m$ is epic because $e'_m$ is such (again because of Theorem \ref{thm_main}).
		
		\smallskip
		\noindent $(\Leftarrow)$  Let $m:M\rightarrow X$ be epic, than we can apply Corollary \ref{uniqueness_fact_coro} to the factorization $m=\id{X}\circ m$ to get that $[\id{X}]=\reg_X( [m])$.	\qedhere
	\end{enumerate}
\end{proof}

Now we are ready to deduce the following result.
\begin{theorem}\label{thm_epi=reg} 
	\normalmarginpar
	\marginpar{
		C'è una classe generale di esempi?($\Ring$ e $\Haus$ dovrebbero funzionare perché so già per altre vie che epi=reg, si può generalizzare?)}
	
	The following are equivalent for a category $\X$ that satisfies (Hyp.1--4):
	\begin{enumerate}[\rm (1)]
		\item every strong mono is regular;
		\item $\epi$ is well-defined and $\epi=\reg$.
	\end{enumerate} 
\end{theorem}

\begin{proof} 
	(1)$\Rightarrow$(2) By Lemma \ref{char_epi} and point (1) of Proposition \ref{whe} we know that $\reg_X([m])$ is the supremum of
	\[
	\mathcal{G}([m])=\left\{[n]\in  \sub{\X}(X) \mid m=n\circ k \text{ for some } k \text{ such that } [k] \text{ is epi}  \right\}
	\]
	and this is exactly the definition of $\epi_{X}([m])$.
	
	\smallskip \noindent 
	(2)$\Rightarrow$(1). Let $\phi\colon X\to Y$ be a morphism in $\X$, and consider its factorizations through $\Coim(\phi)$ and $\Im(\phi)$, as illustrated in the usual diagram:
	\[
	\xymatrix@C=40pt@R=15pt{
		& \Im(\phi) \ar[dr]^{m'_\phi}  &  \\ 
		X \ar[ru]^{e'_\phi} \ar[rr]|-{\ \ \ }^(0.6){\phi}\ar[dr]_{e_\phi}&  & Y,\\
		& \Coim(\phi) \ar[ur]_{m_\phi} \ar[uu]^(0.35){d}&
	}
	\]
	where $m'_\phi$ is a regular monomorphism, $e_\phi$ is a regular epimorphism, and $m_\phi$ is a monomorphism by (Hyp.4). We have to verify that $e'_\phi$ is an epimorphism and, for this, it is enough to show that $d$ is epic. 
	Now, from Remark \ref{clos_coim} and the hypothesis
	\[[m'_\phi]=\reg_Y([m_\phi])=\epi_Y([m_\phi])\]
	 therefore, $d$ is the morphism witnessing from $[m_\phi] \leq \epi_Y([m_\phi ])$ which is an epimorphism by Corollary~\ref{coro_map_to_epi}.
\end{proof} 

\subsection{When epi implies surjective?} 
\begin{theorem}
Let $\X$ be a category satisfying (Hyp.1--4). The following are equivalent:
	\begin{enumerate}[\rm (1)]
		\item every epimorphism in $\X$ is surjective;
		\item $\X$ is \emph{balanced}, i.e. an arrow which is both epic and monic is an isomorphism;
		\item $\epi$ is well-defined and it coincides with the discrete closure operator.
	\end{enumerate}
Furthermore, if every strong mono is regular, then the above conditions are also equivalent to the following:
	\begin{enumerate}[\rm (4)]
		\item for every morphism $\phi\colon X\rightarrow Y$ in $\X$ the arrow $d$ in the following diagram is an isomorphism
			\[
		\xymatrix@C=40pt@R=15pt{
			& \Im(\phi) \ar[dr]^{m'_\phi}  &  \\ 
			X \ar[ru]^{e'_\phi} \ar[rr]|-{\ \ \ }^(0.6){\phi}\ar[dr]_{e_\phi}&  & Y,\\
			& \Coim(\phi) \ar[ur]_{m_\phi} \ar[uu]^(0.35){d}&
		}
		\]
\end{enumerate}
\noindent 
\begin{enumerate}[\rm(5)]
	\item  $\mathsf{reg}$ is the discrete closure operator.
\end{enumerate}
\end{theorem} 
\begin{proof}
(1)$\Rightarrow$(2). Let $s$ be both epic and monic and note that, by (1), $s$ is surjective and so, by (Hyp.4), it is a regular epimorphism. But regular epimorphisms that are also monic are isomorphisms.

\smallskip\noindent
(2)$\Rightarrow$(3). Let $X\in \X$ and consider two monomorphisms $m\colon M\to X$ and $k\colon K\to X$, such that $m=k\circ s$, with $s$ and epi. Since $s$ is also monic by Proposition \ref{composition}, (2) entails that $s$ is an isomorphisms, showing that the family $\G([m])$ of Definition \ref{epi_def} is just the singleton $\left \{[m]\right \}$. 

\smallskip\noindent
(3)$\Rightarrow$(2). Suppose that $s\colon S\to X$ is both epic and monic, then $[\id{X}]\in \G([s])$, showing that $\epi_X([s])=[\id{X}]$. By (3), $\epi_X([s])=[s]$, so $[s]=[\id{X}]$ in $\sub{\X}(X)$, that is, $s$ is an isomorphism.

\smallskip\noindent
(2)$\Rightarrow$(1). Let $\phi\colon X\to Y$ be an epimorphism and consider its factorization $\phi=m_\phi\circ e_\phi$ through $\Coim(\phi)$. Then, $m_\phi=\phi_{!}(\id{X})$ is monic, but also epic (from Proposition \ref{composition}). Thus, by (2), $m_\phi$ is an isomorphism and thus $\phi$ is a regular epimorphism. But, by (Hyp.4), this just means that $\phi$ is surjective. 

\smallskip\noindent
From now on, suppose that every strong mono is regular.

\smallskip\noindent
(2)$\Leftrightarrow$(5) follows by Theorem~\ref{thm_epi=reg}.

\smallskip\noindent
(1)$\Rightarrow$(4). Let $\phi\colon X\to Y$ be a morphism and consider its factorization $\phi=m'_\phi\circ e'_\phi$ through $\Im(\phi)$. Condition (3) of Theorem~\ref{thm_main}, tells us that $e'_\phi$ is an epimorphism which, by (1), implies that it is surjective and so, by (Hyp.4), we get that $e'_\phi$ is a regular epimorphism. Hence, $\phi=m'_\phi\circ e'_\phi$ is  a (regular epi, mono)-factorization. One can now conclude by Corollary~\ref{uniqueness_fact_coro}.

\smallskip\noindent
(4)$\Rightarrow$(5). Let $\phi\colon N\to X$ be a monomorphism in $\X$. Then, as subobjects of $X$, we have that $[m]=[m'_\phi]$ and $\reg_X([m])=[m_\phi]$. But (4) implies that $[m_\phi]=[m'_\phi]$ and thus $\reg$ is the discrete closure operator.
\end{proof}


%Point $1$ of theorem \ref{thm_reg} now gives the following.
% \begin{corollary}
% 	Every epi is surjective if and only if every mono is regular.
% \end{corollary}
%
%
%
%\section{A failed attempt}
%\begin{definition} A closure operator $c$ is said to be \emph{weakly hereditary} if
%	\begin{equation*}
%	c_{c_X(M)}(j_m)\simeq \id{c_X(M)}
%	\end{equation*} for any $m:M\rightarrow X$ in $\mathcal{M}$, $c_X(m):c_X(M)\rightarrow X$ and $j_m$ is the witness to $m\leq c_X(m)$.
%\end{definition}
%
%\begin{lemma}\label{char2} Let $c$ be a closure operator on $\X$, for any $m:M\rightarrow X$ in $\mathcal{M}$.
%	\begin{equation*}
%	\mathcal{G}(m):=\{k\in \subm{X} \mid m=k\circ d \text{ for some } d \ c\text{-dense} \}
%	\end{equation*}
%	Then $c$ is weakly hereditary if and only if for all $m$ $\sup(\mathcal{G}(m))$ exists and it is equal to $c_X(m)$.
%\end{lemma} 
%\begin{proof}Suppose that $c$ is weakly hereditary, then $j_m$ is $c$-dense, so $c_X(m)\in \mathcal{G}(m)$, moreover, if $k:K\rightarrow X$, 
%	\begin{align*}c_X(m)&=c_X(k\circ d)\\&\simeq c_X(\exists_{k}(d))\\&\geq \exists_k(c_K(d))\\&\simeq\exists_k(\id{K})\\&\simeq k\circ \id{K}\\&=k
%	\end{align*}
%	hence $c_X(m)$ is the maximum of $\mathcal{G}(m)$. For the other implication we have to compute $c_{c_X(M)}(j_m)$. By \ref{comp} composition with $c_X(m)$ induces a function $\exists_{c_X(m)}:\subm{c_X(M)}\rightarrow\subm{X}$, since $c_X(m)$ a mono we have that, for all $k, h\in \subm{c_X(M)}$, $k= h\circ j$ if and only if $c_X(m)\circ k=c_X(m)\circ (h\circ k)$, so $\exists_{c_X(m)}$ is an order-embedding. Now, $k:K\rightarrow c_X(M)\in \mathcal{G}(j_m)$ if and only if $k\circ d= j_m$ for some $d$ such that $c_K(d)=\id{K}$, so $c_X(m)\circ k\in \mathcal{G}(m)$ while, by hypothesis all $l\in \mathcal{G}(m)$ are less or equal than $c_X(m)$. We conclude that $\exists_{c_X(m)}(\mathcal{G}(j_m))=\mathcal{G}(m)$ and:
%	\begin{align*}
%	c_X(m)\circ c_{c_{c_X(M)}(j_m)}&=\exists_{c_X(m)}(c_{c_X(M)}(j_m))\\&=\exists_{c_X(m)}(\sup(\mathcal{G}(j_m)))\\&=\sup(\exists_{c_X(m)}(\mathcal{G}(j_m)))\\&=\sup(\mathcal{G}(m))\\&= c_X(m)
%	\end{align*}
%	and we are done.
%\end{proof}
%\begin{definition}
%	Let $m:M\rightarrow X$ be in $\mathcal{M}$, we define:
%	\begin{equation*} \ep{X}{m}=[\Coim{m}]
%	\end{equation*}
%\end{definition}
%
%
%\begin{theorem} The following are true:
%	\begin{enumerate}
%		\item  Given $m:M\rightarrow X$ in $\subm{X}$, $\Coim{m}\simeq \id{X}$ if and only if $m$ is an epi;
%		\item $\mathsf{repi}$ is a weakly hereditary closure operator.
%	\end{enumerate}
%\end{theorem}
%\begin{proof}\hspace{1pt}
%	\begin{enumerate}
%		
%		\item Since the arrow $M\rightarrow \Coim{M}$ is a regular epi, if $\Coim{m}\simeq\id{X}$ we have immediately that $m$ is a regular epi too. For the other implication fact $4$ above gives:
%		\begin{center}
%			\begin{tikzpicture}
%			\node(E)at(-3,0){$M$};
%			\node(A)at(0,0){$\Coim{M}$};
%			\node(C)at(3,0){$X$};
%			\node(D)at(0,-1.3){$X$};
%			\draw[->](A)--(C)node[pos=0.5, above]{$\Coim{m}$};
%			\draw[->](E)..controls(-2.1,1.5)and(2.1,1.5)..(C)node[pos=0.5, above]{$m$};
%			\draw[->](E)--(A)node[pos=0.5, above]{$e_m$};
%			\draw[->](D)--(A)node[pos=0.5, right]{$u$};		
%			\draw[->](E)--(D)node[pos=0.5, below, xshift=-0.1cm]{$m$};
%			\draw[->](D)--(C)node[pos=0.5, below, xshift=0.1cm]{$\id{X}$};			
%			\end{tikzpicture}
%		\end{center}
%		$u$ is split mono since $\Coim{m}\circ u=\id{X}$ and epi since $u\circ m=e_m$, thus $u$ is an isomorphism and $m$ is a regular epi. 
%		\item Notice that $\mathsf{repi}_X=\exists_{\id{X}}:\sub{\X}(X)\rightarrow \sub{\X}(X)$, so $\mathsf{ep}$ is surely monotone, while extensionality is given by the unit of $\exists_{\id{X}}\dashv\id{X}^*$:
%		\begin{align*}
%		m&\leq {\id{X}^*}(\exists_{\id{X}}(m))\\&\simeq \exists_{\id{X}}(m)
%		\end{align*}
%		We are left with continuity, but this follows since adjoints compose:
%		\begin{align*}
%		\phi_!(\exists_{\id{X}}(m))&=\exists_{f\circ \id{X}}(m)\\&=\exists_{\id{Y}\circ f}(m)\\&=\exists_{\id{Y}}(\phi_!(m))
%		\end{align*}
%		Weakly hereditarity now follows from the previous point since the arrow $M\rightarrow \Coim{M}$ is a regular epi for all $m:M\rightarrow X$.
%	\end{enumerate}
%\end{proof}
%\begin{corollary}\label{supinf2} For any $m\in \subm{X}$:
%	\begin{equation*} 
%	\ep{X}{m}=\bigvee\{[k]\in \sub{\X}(X) \mid k\circ d=m \text{ for some regular epi } d\}
%	\end{equation*}
%\end{corollary}
%\fi
%
%
%\section{Using coequalizers to describe the regular closure}\label{coeq:vs:regclop}
%
%The following lemma (paired with the explicit description of coequalizers given in Section \ref{coeq_subs}) will be our main criterion to study regular closures and, consequently, to characterize epimorphisms in $\Lqm$.
%
%\begin{lemma}\label{claim:co-eq}
%Let $X \in \mathfrak X$ and let $\mu\colon M\to X$ be the inclusion of subobject $M$ in $X\in\Lqm$. Let also $\gamma\colon X\sqcup X\to X\sqcup_M X$ be the cokernel pair of $\mu$. Then
%\[\re_X(M)= \eq(\gamma\circ \iota_1, \gamma\circ \iota_2),\] 
%where $\iota_k\colon X\to X\sqcup X$ are the canonical maps to the coproduct, for $k=1,2$. 
%\end{lemma}
%\begin{proof} 
%Given $m\in M$, clearly $\mu(m)\in \eq(\gamma\circ \iota_1, \gamma\circ \iota_2)$, in fact, $\gamma (\iota_1(\mu(m)))=\gamma(\mu_1(m))=\gamma(\mu_2(m))=\gamma (\iota_2(\mu(m)))$ (as $\gamma$ coequalizes $\mu_1:=\iota_1\circ\mu$ and $\mu_2:=\iota_2\circ\mu$ by construction). Hence, $M \subseteq \eq(\gamma\circ \iota_1, \gamma\circ \iota_2)$. Now, as $\re_X(M)$ is, by definition, the smallest equalizer containing $M$, we deduce that  $\re_X(M)\subseteq \eq(\gamma\circ \iota_1, \gamma\circ \iota_2)$. 
%
%\smallskip
%To prove the opposite inclusion, assume that $M \subseteq  \eq(f,g)$ for some pair of morphisms $f,\, g\colon X \rightrightarrows Y$. Let $\xi\colon X \sqcup X \to Y$ be the unique morphism such that $\xi \circ \iota_1 = f$ and $\xi \circ \iota_2 = g$ 
%\[
%\xymatrix@R=20pt@C=50pt{
%X\ar@/^18pt/[rr]^{f}\ar@/_18pt/[rr]_{g}\ar@<-.5ex>[r]_-{\iota_2} \ar@<.5ex>[r]^-{\iota_1}&X \sqcup X\ar[r]^-\xi&Y
%}
%\]
%(existence and uniqueness are ensured by the universal property of the coproduct). The inclusion $M \subseteq \eq(f,g) = \eq(\xi\circ \iota_1, \xi\circ \iota_2)$ implies that $M \subseteq \eq(\xi\circ \iota_1\circ \mu, \xi\circ \iota_2 \circ \mu)=\eq(\xi\circ \mu_1, \xi\circ \mu_2)$, so the universal property of the coequalizer $\gamma\colon X \sqcup X \to X \sqcup_M X$ gives a unique morphism $\eta\colon X \sqcup_M X \to Y$ such that $\xi = \eta \circ \gamma$:
%\[
%\xymatrix@R=15pt@C=60pt{
%%M\ar@/^15pt/[rr]^{\mu_1}\ar@/_15pt/[rr]_{\mu_2}\ar[r]^-{\mu}&X\ar@<-.5ex>[r]_-{\iota_2} \ar@<.5ex>[r]^-{\iota_1}&
%X \sqcup X\ar[dr]_{\xi} \ar[r]^-{\gamma}& X \sqcup_M X\ar@{.>}[d]^-{\exists!\,\eta}\\
%&Y
%}
%\] 
%%Since $m$ is a monomorphism, 
%Then, $\eq(f,g) =  \eq(\xi\circ \iota_1, \xi\circ \iota_2) = \eq(\eta \circ  \gamma\circ \iota_1 , \eta \circ \gamma\circ \iota_2) \supseteq \eq(\gamma\circ \iota_1, \gamma\circ \iota_2)$. This shows that $\eq(\gamma\circ \iota_1, \gamma\circ \iota_2)$ is contained in any equalizer that contains $M$, hence, it is the smallest equalizer containing $M$, that is, $\eq(\gamma\circ \iota_1, \gamma\circ \iota_2)=\re_X(M)$.
%\end{proof}
%

%\section{Surjectivity and the $\mathsf{epi}$, $\mathsf{reg}$ operators}
%
%
%\begin{corollary}
%	In $\X$ all the epis are surjective if and only if $\mathsf{epi}$ is the discrete closure operator.
%\end{corollary}
%\begin{proof}
%	Let $e:X\rightarrow Y$ be an epi, by hypotheses $1$-$4$ can factor an epi $e:X\rightarrow Y$ as $m\circ e'$ with $e'$ surjective and $m$ mono, so $m$ is mono and epi. 
%	\begin{itemize}
%		\item[$\Rightarrow $] Let $m:M\rightarrow X$ be a mono, if $m=n\circ d$ for some epi $d$ then $d$ is mono and surjective, thus an isomorphism. The definition of $\mathsf{epi}$ now implies that  $\ep{X}{m}=[m]$.
%		\item[$\Leftarrow $] $m$ is mono and epi, and $1_X\circ m=m$, so $[1_X]$ appears in the family of which $\ep{X}{m}$ is the supremum, so $\ep{X}{m}= [1_X]$. Since $\ep{X}{m}$ it is assumed to be discrete this means that $[m]=[1_X]$, i.e. that $m$ is an isomorphism. So $e$ is surjective by hypothesis $4$.
%	\end{itemize}
%\end{proof}
%
%
\begin{thebibliography}{999}
 
\footnotesize{ 

\bibitem{cats} {J. Ad\'amek, H. Herrlich, G. E. Strecker}: {\em Abstract and Concrete Categories. The Joy of Cats}, online edition, {\tt http://katmat.math.uni-bremen.de/acc/acc.pdf} (2005).
\bibitem {pres} {Ad{\'a}mek, J. and Rosick{\`y}, J.}, {\em Locally Presentable and Accessible Categories}, {\em London Mathematical Society Lecture Note Series},  vol. 188, Cambridge University Press 1994, pp. 316+x


\bibitem {algth} {Ad{\'a}mek, J. and Rosick{\`y}, J. and Vitale, E. M.}, {\em Algebraic theories: a categorical introduction to general algebra}, {\em Cambridge Tracts in Mathematics},  vol. 184, Cambridge University Press 2010, pp. 249+xvii
\bibitem{AKM} {R. L. Adler, A. G. Konheim, M. H. McAndrew}, \textit{Topological entropy}, Trans. Amer. Math. Soc. 114 (1965) 309--319.

%\bibitem{AD} {L. Aussenhofer, D. Dikranjan}, \textit{Locally quasi-convex compatible topologies on LCA groups}, submitted.
 
%\bibitem{AZD} {F. Ayatollah Zadeh Shirazi, D. Dikranjan}, \textit{Set-theoretical entropy: A tool to compute topological entropy}, Proceedings ICTA2011 Islamabad Pakistan July 4--10 2011, pp. 11--32, Cambridge Scientific Publishers (2012).

 \bibitem{AU} P. Alexandrov and P. Urysohn, {\em M\' emoire sur les espaces topologiques compactes}, 
Verh. Kon. Akad. Wetensch. Amsterdam. Afd. Natuurk. Sect. 1. 14, no. 1 (1929) [{\em  On compact topological spaces}, (Russian) Trudy Mat. Inst. Steklov. 31, (1950). 95 pp.]. 
 
\bibitem{Arduini} P. Arduini, {\em Monomorphisms and epimorphisms in abstract categories}, Rendiconti del Seminario Matematico dell'Universit\`a di Padova 42 (1969) 135--166. 
 
\bibitem{Ba} S. Baron, \emph{Note on epi in $T_0$}, Canad. Math. Bull. 11 (1968) 503--504.
%\bibitem{BDG} \textsc{F. Berlai, D. Dikranjan, A. Giordano Bruno}: \textit{Scale function vs Topological entropy}, Topology Appl. 160 (2013) 2314--2334.

%\bibitem{B} \textsc{R. Bowen}: \textit{Entropy for group endomorphisms and homogeneous spaces}, Trans. Amer. Math. Soc. 153 (1971) 401--414.

%\bibitem{bow71} \textsc{R. Bowen}: \textit{Erratum to ``Entropy for group endomorphisms and homogeneous spaces''}, Trans. Amer. Math. Soc. 181 (1973) 509--510.

\bibitem{B} {G. Birkhoff}, {\em Lattice Theory}, in: AMS Colloquium Publications, Vol. 25, American Mathematical Society, Providence, RI, 1984.

\bibitem{borceux1} {F. Borceux} {\it Handbook of categorical algebra: volume 1, Basic Category Theory}, {\sl  Encyclopedia of Mathematics and its Applications }, vol. 50, Cambridge University Press 1994, pp. 345+xvi

\bibitem{bor2} {F. Borceux} {\it Handbook of categorical algebra: volume 2, Categories and Structures}, {\sl  Encyclopedia of Mathematics and its Applications }, vol. 51, Cambridge University Press 1994, pp. 443+xvii

%\bibitem{Carraro} \textsc{G. Carraro}: \textit{Invarianti nella categoria dei flussi per spazi vettoriali}, Master Thesis, University of Padova, 2015.

\bibitem{IAA} {I. Castellano,  D. Dikranjan, D. Freni, A. Giordano Bruno, D. Toller}, \textit{Intrinsic entropy for \gen \qm semilattices}, to appear in Journal of Algebra and its Applications. 

\bibitem{CGBalg} {I. Castellano, A. Giordano Bruno}, \textit{Algebraic entropy on locally linearly compact vector spaces}, Rings, Polynomials, and Modules, Springer (2017) 103--127.

\bibitem{CGBtop} {I. Castellano, A. Giordano Bruno}, \textit{Topological entropy on locally linearly compact vector spaces}, Topol. Appl. 252 (2019) 112--144.

\bibitem{CasGBZa} {I. Castellano, A. Giordano Bruno, N. Zava}, {\em Weakly weightable \qm semilattices}, work in progress.

\bibitem{CDT} M. Clementino, D. Dikranjan, W. Tholen, {\em Torsion theories and radicals in normal categories}, J. Algebra 305 (2006), no. 1, 98--129. 

\bibitem{Cas} G. Castellini, {\em Categorical closure operators}, Mathematics: Theory \& Applications. Birkh\" auser Boston, Inc., Boston, MA, 2003. 

\bibitem{CP} A. H. Clifford, G. B. Preston, \emph{The algebraic theory of semigroups}, Vol. 1, Mathematical Surveys of the American Mathematical Society. Providence, Rhode Island, 1961.

\bibitem{D} {D. Dikranjan}, {\it Closure operators related to von Neumann's kernel}, Topology Appl. 153, 11 (2006) 1930--1955. 

%%%%%%%%%%%%%%%%%%%%%%%%%%%%%%%%%%%%%%%%%%%%%%%%
%
%\bibitem{DDR} \textsc{U. Dardano, D. Dikranjan, S. Rinauro}: \textit{Inertial properties in groups}, Int. J. Group Theory 7 (2018) 17--62.
%
%\bibitem{D} \textsc{D. Dikranjan}: \textit{A uniform approach to chaos}, Algebra meets Topology: Advances and Applications, July 19-23, 2010, UPC - Barcelona Tech., Barcelona (Spain), http://atlas-conferences.com/cgi-bin/abstract/cbah-54. 
%
%%\bibitem{DFGB} \textsc{D. Dikranjan, A. Fornasiero, A. Giordano Bruno}: \textit{Algebraic entropy for amenable semigroup actions}, preprint.
%
%%\bibitem{DG0} \textsc{D. Dikranjan, A. Giordano Bruno}: \textit{The Pinsker subgroup of an algebraic flow}, J. Pure Appl. Algebra 216 (2) (2012) 364--376.
%
%%\bibitem{DG-lf} \textsc{D. Dikranjan, A. Giordano Bruno}: \textit{Limit free computation of entropy}, Rend. Istit. Mat. Univ. Trieste 44 (2012) 297--312.
%
%\bibitem{DG-islam} \textsc{D. Dikranjan, A. Giordano Bruno}: \textit{Topological entropy and algebraic entropy for group endomorphisms}, Proceedings ICTA2011 Islamabad Pakistan July 4--10 2011, 133--214, Cambridge Scientific Publishers (2012).
%
%%\bibitem{DG-bridge} \textsc{D. Dikranjan, A. Giordano Bruno}: \textit{The connection between topological and algebraic entropy}, Topology Appl. 159 (13) (2012) 2980--2989.
%
%%\bibitem{DG1} \textsc{D. Dikranjan, A. Giordano Bruno}: \textit{Entropy in a category}, Appl. Categ. Structures 21 (1) (2013) 67--101.
%
%\bibitem{DGB_PC} \textsc{D. Dikranjan, A. Giordano Bruno}: \textit{Discrete dynamical systems in group theory}, Note Mat. 33 (1) (2013) 1--48.
%
%
%%%%%%%%%%%%%%%%%%%%%%%%%%%%%%%%%%%%%%%%%%%%%%%%%%%%%%%%%%%%%%

%\bibitem{DGB-BT} {D. Dikranjan, A. Giordano Bruno}, \textit{The Bridge Theorem for totally disconnected LCA groups}, Topol. Appl. 169 (2014) 21--32.

\bibitem{DG} {D. Dikranjan, A. Giordano Bruno}, \textit{Entropy on abelian groups}, Adv. Math. 298 (2016) 612--653.

\bibitem{DGBuatc} {D. Dikranjan, A. Giordano Bruno}, \textit{Entropy on normed semigroups}, Diss. Math. 542 (2019) 1--90.

%\bibitem{DGS} \textsc{D. Dikranjan, A. Giordano Bruno, L. Salce}: \textit{Adjoint algebraic entropy}, J. Algebra 324 (3) (2010) 442--463.

%\bibitem{GBS2} \textsc{D. Dikranjan, A. Giordano Bruno, L. Salce}: \textit{Intrinsic adjoint algebraic entropy}, work in progress.

%\bibitem{DGSV1} \textsc{D. Dikranjan, A. Giordano Bruno, L. Salce, S. Virili}: \textit{Fully inert subgroups}, J. Group Theory 16 (2013) 915--939.

\bibitem{IAA.2} {D. Dikranjan, A. Giordano Bruno, H.-P. Kunzi, D. Toller, N. Zava}, {\em Generalized \qm semilattices}, submitted. 

%\textit{Intrinsic entropy for \gen \qm semilattices}, to appear in Journal of Algebra and its Applications. 

\bibitem{DGSV} {D. Dikranjan, A. Giordano Bruno, L. Salce, S. Virili}, \textit{Intrinsic algebraic entropy}, J. Pure Appl. Algebra 219 (2015) 2933--2961.

\bibitem{DGiu}  {D. Dikranjan,  E. Giuli}, {\it Closure  operators  I},  Topology Appl. 27  (1987) 129--143.

\bibitem{DGiu1}  {D. Dikranjan,  E. Giuli}, {\it Factorizations, injectivity and compactness in categories of modules.}  Comm. in Algebra {19} (1) (1991) 45--83.

\bibitem{DGT}  {D. Dikranjan,  E. Giuli, W. Tholen}, {\em Closure operators} II. {\sl Categorical topology and its relation to analysis, algebra and combinatorics} (Prague, 1988) 297--335, World Sci. Publ., Teaneck, NJ, 1989. 

%\bibitem{DGSZ} {D. Dikranjan, B. Goldsmith, L. Salce, P. Zanardo}, \textit{Algebraic entropy for abelian groups}, Trans. Amer. Math. Soc. 361 (2009) 3401--3434.

%\bibitem{DGBZ} \textsc{D. Dikranjan, A. Giordano Bruno, N. Zava}: \textit{The category of \gen \qm semilattices with an application to the intrinsic entropy}, work in progress.

%\bibitem{DK} \textsc{D. Dikranjan, H.-P. K\" unzi}: \emph{Uniform  entropy vs topological entropy}, Topol. Algebra Appl. 3 (2015) 104--114. 

\bibitem{DikProZav} {D. Dikranjan, I. Protasov, N. Zava}, {\em Hyperballeans of groups}, Topol. Appl. 263 (2019) 172--198.

\bibitem{DT} {D. Dikranjan, W. Tholen},  {\it Categorical Structure of Closure Operators with Applications to Topology, Algebra and   Discrete Mathematics},  
{\sl Mathematics and its Applications}, vol. 346, Kluwer Academic Publishers,  Dordrecht-Boston-London 1995, pp. 358+xviii.

\bibitem{DT1} {D. Dikranjan, W. Tholen},  {\it  Closure operators and dual closure operators},  J. Algebra 439 (2015) 373--416

\bibitem{DTW}  D. Dikranjan, W. Tholen, S. Watson, {\em Classification of closure operators for categories of topological spaces},
Categorical structures and their applications, 69--98, World Sci. Publ., River Edge, NJ, 2004.

\bibitem{DTo} D. Dikranjan, A. Tonolo, {\it On a characterization of linear compactness}, Rivista di Matematica Pura ed Applicata {17} (1995) 95--106.  

\bibitem{DU} D. Dikranjan, V. Uspenskij, {\it Categorically compact  topological groups}, J. Pure and Appl. Algebra, {126} (1998) 149--168.  

\bibitem{DikZa1} {D. Dikranjan, N. Zava}, {\em Some categorical aspects of coarse spaces and balleans}, Topology Appl., 225 (2017), 164--194.

\bibitem{Fay} T. Fay, {\em Compact modules}, Comm. Algebra 16 (1988), 1209--1219. 

\bibitem{FG} T. Fay, G. Walls, {\em A characterization of categorically compact locally nilpotent groups}, 
Comm. Algebra 22 (1994), no. 9, 3213--3225.

%\bibitem{DSZ} \textsc{D. Dikranjan, L. Salce, P. Zanardo}: \textit{Fully inert subgroups of free Abelian groups}, Period. Math. Hungar. 69 (2014), no. 1, 69--78.

%\bibitem{Dik+Manolo} \textsc{D. Dikranjan, M. Sanchis}: \emph{Dimension and entropy in compact topological groups}, submitted.
%
%\bibitem{DSV} \textsc{D. Dikranjan, M. Sanchis, S. Virili}: \textit{New and old facts about entropy in uniform spaces and topological groups}, Topology Appl. 159 (2012) 1916--1942.

%\bibitem{Fek} \textsc{M. Fekete}: \textit{\"Uber die Verteilung der Wurzeln bei gewisser algebraichen Gleichungen mit ganzzahlingen Koeffizienten}, Math. Zeitschr. {17} (1923) 228--249.

%\bibitem{Fornasiero}  \textsc{A. Fornasiero}, \textit{Entropy of endomorphisms of weighted monoids}, preprint.

%\bibitem{G0} \textsc{A. Giordano Bruno}: \textit{Algebraic entropy of shift endomorphisms on products}, Comm. Algebra 38 (11) (2010) 4155--4174.

%\bibitem{G} \textsc{A. Giordano Bruno}: \textit{Adjoint entropy vs Topological entropy}, Topology Appl. 159 (9) (2012) 2404--2419.

%\bibitem{DG-tdlc} \textsc{A. Giordano Bruno}: \textit{Entropy for automorphisms of totally disconnected locally compact groups}, Topology Proc. 45 (2015) 175--187.

%\bibitem{GBS} {A. Giordano Bruno, L. Salce}, \textit{A soft introduction to algebraic entropy}, Arabian J. Math. 1 (2012) 69--87.

\bibitem{GBST} {A. Giordano Bruno, M. Shlossberg, D. Toller}, \textit{Algebraic entropy in compactly covered locally compact groups}, Topology Appl. 263 (2019) 117--140.

%\bibitem{GBSp} \textsc{A. Giordano Bruno, P. Spiga}: \textit{Some properties of the growth and the algebraic entropy of group endomorphisms}, J. Group Theory 20 (4) (2017) 763--774.

%\bibitem{GV} \textsc{A. Giordano Bruno, S. Virili}: \textit{Algebraic Yuzvinski Formula}, J. Algebra 423 (2015) 114--147.

%\bibitem{GV2} \textsc{A. Giordano Bruno, S. Virili}: \textit{On the Algebraic Yuzvinski Formula}, Topol. Algebra and its Appl. 3 (2015) 86--103.

\bibitem{GBV} {A. Giordano Bruno, S.Virili}, \textit{Topological entropy in totally disconnected locally compact groups}, Ergodic Theory Dynam. Systems 37 (2017) 2163--2186.


%\bibitem{Her}  {H. Herrlich}, {\em Hyperconvex hulls of metric spaces}, Proceedings of the Symposium on General Topology and Applications (Oxford, 1989). Topology Appl. 44 (1992), no. 1-3, 181--187. 

%\bibitem{GS}  \textsc{R. Goebel, L. Salce}: \textit{Endomorphism rings with different rank-entropy supports}, Quarterly J. Math. Oxford (2011) 1--22.
 
%\bibitem{GK} \textsc{B. Goldsmith, K. Gong}: \textit{On adjoint entropy of abelian groups}, Comm. Algebra 40 (2012) no. 3, 972--987.
%
%\bibitem{GSZ} \textsc{B. Goldsmith, L. Salce, P. Zanardo}: \textit{Fully inert subgroups of Abelian $p$-groups}, J. Algebra 419 (2014) 332--349.

%\bibitem{Lef} \textsc{S. Lefschetz}: \textit{Algebraic topology}, Vol. 27. American Mathematical Soc., 1942.

\bibitem{HSS} H. Herrlich, G. Salicrup and G. Strecker, {\em Factorizations, denseness, separation, and relatively compact objects}, 
Topology Appl. 27 (1987), no. 2, 157--169. 

\bibitem{HKi} A. Horn and N. Kimura, {The category of semilattices}, Algebra Universalis 1 (1971), no. 1, 26--38.

\bibitem{HJ} J. M. Howie and J. Isbell, {\em Epimorphisms and dominions II}, J. Algebra 6 (1967) 7--21.

%\bibitem{Is1} \NBD \textsc{J. Isbell}: Some remarks concerning categories and subspaces, Canad. J. Math. 9 (1957) 563-577.

\bibitem{Is1} {J. Isbell}, \emph{Epimorphisms and dominions}, in: Proc. Conf. Categorical Algebra, La Lolla, 1965 (Springer, Berlin 1966) 232--246.

\bibitem{Is2} {J. Isbell}, \emph{Epimorphisms and dominions IV}, Proc. London Math. Soc. (2) 1 (1969) 265--273.

\bibitem{JMT} G. Janelidze, L. Márki and W. Tholen, {\em Semi-abelian categories}, 
Category theory 1999 (Coimbra). J. Pure Appl. Algebra 168 (2002), no. 2-3, 367--386. 

\bibitem{stone} P. T. Johnstone, {\em Stone spaces},{\em Cambridge Studies in Advanced Mathematics}, vol.3, Cambridge University Press, 1982, pp. 370 + xxi.

\bibitem{KOO} A. Klyachko, A. Olshanskii and D. Osin, {\em  On topologizable and non-topologizable groups} Topology Appl. 160 (2013), no. 16, 2104--2120. 
\bibitem{lawvere} Lawvere F. W., {\em Functorial semantics of algebraic theories}, Proceedings of the National Academy of Sciences of the United States of America 50 (1963), no. 5, 869--872.


\bibitem{linton2} F. E. J. Linton, {\em An outline of functorial semantics},
Seminar on triples and categorical homology theory, Springer 1969, 7--52.

\bibitem{linton1} F. E. J. Linton, {\em Coequalizers in categories of algebras},
Seminar on triples and categorical homology theory, Springer 1969, 75--90.

\bibitem{linton3} F. E. J. Linton, {\em Some aspects of equational categories},
{Proceedings of the Conference on Categorical Algebra}, Springer, 1966, 84--94.

\bibitem{MS} J. L. MacDonald, A. Stone, {\em The tower and regular decomposition}, Cahiers de Topologie et G\'eom\'etrie Diff\'erentielle Cat\'egoriques 23 (1982) no. 2, 197--213.


\bibitem{mac} {S. Mac Lane}, {\it Categories for the working mathematician}, {\sl Graduate Texts in Mathematics}, vol. 5, Springer Science \& Business Media 2013, pp. 317 +xii.

\bibitem{MMS} M. Majidi-Zolbanin, N, Miasnikov, L. Szpiro, \emph{Dynamics and entropy in local algebra, submitted}, arxiv:1109.6438. -- aggiornare 

\bibitem{Man} E. G. Manes, {\em Compact Hausdorff objects},   General Topology and Appl. 4 (1974), 341--360. 

\bibitem{ManAlg} E. G. Manes, {\em Algebraic theories}, {\sl Graduate Texts in Mathematics}, vol. 26, Springer Science \& Business Media, 2012, pp. 356+ix.

\bibitem{epiring1} P. Mazet, {\em Caract{\'e}risation des {\'e}pimorphismes par relations et g{\'e}n{\'e}rateurs}, {S{\'e}minaire Samuel. Alg{\`e}bre commutative} (1967), no. 2 , {1--8}.

%\bibitem{Na}  {\sc L. Nachbin}, {\em  Topology and order}, Translated from the Portuguese by Lulu Bechtolsheim. Van Nostrand Mathematical Studies, No. 4 D. Van Nostrand Co., Inc., Princeton, N.J.-Toronto, Ont.-London 1965 vi+122 pp.

\bibitem{Naka} {Y. Nakamura}, {\em Entropy and semivaluations on semilattices}, Kodai Math. Sem. Rep. Volume 22, Number 4 (1970) 443--468.


\bibitem{Rich} {G. Richter}, {\em Axiomatizing the category of compact Hausdorff spaces}, {Category Theory at Work}, Heldermann Berlin, 1991, 199--215.

\bibitem{epiring2} {N. Roby}, {\em Diverses caract{\'e}risations des {\'e}pimorphismes},	{S{\'e}minaire Samuel. Alg{\`e}bre commutative} (1967), no. 2 , {1--12}.


%\bibitem{Li} \textsc{H. Li}: \emph{Compact group automorphisms, addition formulas and Fuglede-Kadison determinants}, Ann. of Math. (2) 176 (2012) 303--347.

%\bibitem{NR} \textsc{D. G. Northcott, M. Reufel}: \textit{A generalization of the concept of length}, Quart. J. of Math. (Oxford) 16 (2) (1965) 297--321. 
%
%\bibitem{P} {J. Peters}, \textit{Entropy on discrete Abelian groups}, Adv. Math. {33} (1979) 1--13.
%
%\bibitem{Pet1}  \textsc{J. Peters}: \textit{Entropy of automorphisms on {L}.{C}.{A}. groups}, Pacific J. Math. 96 (2) (1981) 475--488.
%
%\bibitem{Po} \textsc{L. S. Pontryagin}: \emph{Topological Groups}, Gordon and Breach, New York, 1966.

%\bibitem{Roh} \textsc{V. Rohlin}: {Metric properties of endomorphisms of compact commutative groups}, Izv. Akad. Nauk. S.S.S.R., Ser. Mat. 28 (1964) 867--874 (In Russian).

%\bibitem{Salce} \textsc{L. Salce}: \textit{Some results on the algebraic entropy}, in ``Groups and Model Theory'' Contemp. Math. 576 (2012).

%\bibitem{SVV} \textsc{L. Salce, P. V\' amos, S. Virili}: \textit{Length functions, multiplicities and algebraic entropy}, Forum Math. 25 (2) (2013)  255--282.

%\bibitem{SV1} \textsc{L. Salce, S. Virili}: \textit{The addition theorem for algebraic entropies induced by non-discrete length functions}, Forum Math. 28 (2016) 1143--1157.
%
%\bibitem{SV2} \textsc{L. Salce, S. Virili}: \textit{Two new proofs concerning the intrinsic algebraic entropy}, Comm. Algebra 46 (2018) 3939--3949.
%
%%\bibitem{SZ0}  \textsc{L. Salce, P. Zanardo}: \textit{Commutativity modulo small endomorphisms and endomorphisms of zero algebraic entropy},  in Models, Modules and Abelian Groups, de Gruyter (2008) 487-497.  
% 
%\bibitem{SZ} {L. Salce, P. Zanardo}, \textit{A general notion of algebraic entropy and the rank entropy},  Forum Math. {21} (4) (2009) 579--599.
%
%\bibitem{SZ1} \textsc{L. Salce, P. Zanardo}: \textit{Abelian groups of zero adjoint entropy}, Colloq. Math. 121 (1) (2010) 45--62.

%\bibitem{Sch} \textsc{A. S. Schwarzc}: \textit{A volume invariant of coverings}, Dokl. Ak. Nauk USSR {105} (1955) 32--34.

%\bibitem{Sinai} \textsc{Y. G. Sinai}: \textit{On the concept of entropy of a dynamical system}, Doklady Akad. Nauk. SSSR 124 (1959) 786--781 (in Russian).

%\bibitem{S} \textsc{L.~N. Stoyanov}: \textit{Uniqueness of topological entropy for endomorphisms on compact groups}, Boll. Un. Mat. Ital. B (7) 1 (3) (1987) 829--847. 

%\bibitem{Tits} \textsc{J. Tits}: \textit{Free subgroups in linear groups}, J. Algebra {20} (1972), 250--270.

\bibitem{Sal} S. Salbany, {\it Reflective subcategories and  closure operators,} {Lecture Notes in Mathematics} {\bf 540} (Springer-Verlag,
Berlin-Heidelberg-New  York 1976) 548--565.

\bibitem{Sch_96} {M. P. Schellekens}, {\em On upper weightable spaces}, %Proc. 11th Summer Conference on General Topology and Applications,
 Ann. New York Acad. Sci. 806 (1996) 348--363.

\bibitem{Sch_1} {M. P. Schellekens}, {\em Extendible spaces}, Appl. Gen. Top. 3 (2002) 169--184.

\bibitem{Sch} { M. P. Schellekens}, \textit{The correspondence between partial metrics and semivaluations}, Th. Computer Science 315 (2004) 135--149.

\bibitem{Schr} O. Schreier, {\em Die Untergruppen der freien Gruppen}, Abh. Math. Sem. Univ. Hamburg 5 (1927), no. 1, 161--183.

\bibitem{Usp} {V. Uspenskij}, \emph{The epimorphism problem for Hausdorff topological groups}, Topology Appl. 57 (1994) 287--294.

%\bibitem{SYRV} {\sc M.P. Schellekens, T. Yu, S. Romaguera, O. Valero}: \textit{On semivaluations and semimodularity}, preprint. 
 
%
%\bibitem{V1} \textsc{P. V\'amos}: \textit{Additive Functions and Duality over Noetherian Rings}, Quart. J. of Math. (Oxford) (2) 19 (1968) 43--55.
%
%\bibitem{V2} \textsc{P. V\'amos}: \textit{Length Functions on Modules}, Ph.D. Thesis, University of Sheffield (1968).
%     
%\bibitem{Vi} \textsc{S. Virili}: \textit{Algebraic $i$-entropies}, Master's Thesis, University of Padova (2010).

%\bibitem{V} \textsc{S. Virili}: \textit{Entropy for endomorphisms of LCA groups}, Topology Appl. 159 (9) (2012) 2546--2556.

%\bibitem{Virili1} \textsc{S. Virili}: \textit{Entropy and length functions on commutative Noetherian rings}, to appear.

%\bibitem{V_BT} \textsc{S. Virili}: \textit{Algebraic and topological entropy of group actions}, Topology Appl., to appear.

%\bibitem{V3} \textsc{S. Virili}: \emph{Algebraic entropy of amenable group actions}, Mathematische Zeitschrift, to appear.

%\bibitem{Virili2} \textsc{S. Virili}: \textit{Length functions, multiplicities and intrinsic polyentropy}, work in progress.

%\bibitem{Wa} \textsc{P. Walters}: \textit{An Introduction to Ergodic Theory}, Springer-Verlag, New-York, 1982. 
%
%\bibitem{W} {M.~D. Weiss}, \textit{Algebraic and other entropies of group endomorphisms}, Math. Systems Theory 8 (3) (1974/75) 243--248.
%
%\bibitem{Willis} \textsc{G. A. Willis}: \textit{The structure of totally disconnected locally compact groups}, Math. Ann. 300 (2) (1994) 341--363. 
%
%\bibitem{Willis2} \textsc{G. A. Willis}: \textit{Further properties of the scale function on a totally disconnected group}, J. Algebra 237 (1) (2001) 142--164. 

%\bibitem{Wilson} \textsc{J. S. Wilson}: \textit{On exponential growth and uniformly exponential growth for groups}, Invent. Math. 155 (2) (2004) 287--303.

%\bibitem{Wolf} \textsc{J. Wolf}: \textit{Growth of finitely generated solvable groups and curvature of Riemannian manifolds}, J. Diff. Geom. {2} (1968) 424--446.

%\bibitem{XDGBST} \textsc{M. Shlossberg, D. Toller, W. Xi}: \emph{Algebraic entropy on topologically quasihamiltonian groups}, work in progress.

%\bibitem{Y} \textsc{S. Yuzvinski\u \i}: \textit{Metric properties of endomorphisms of compact groups}, Izv. Acad. Nauk SSSR, Ser. Mat. {29} (1965) 1295--1328 (in Russian); Engl. Transl.: Amer. Math. Soc. Transl. (2) {66} (1968) 63--98. 
%
%\bibitem{Y1} \textsc{S.~A. Yuzvinski\u{\i}}: \textit{Calculation of the entropy of a group endomorphism}, Sibirsk. Mat. \u Z. 8 (1967) 230--239.
%
%\bibitem{Z1}  \textsc{P. Zanardo}: \textit{Algebraic entropy of endomorphisms over local one-dimensional domains}, J. Algebra Appl. 8 (6) (2009) 759--777.
%
%\bibitem{Z2} \textsc{P. Zanardo}: \textit{Multiplicative invariants and length functions over valuation domains}, J. Commut. Algebra 3 (4) (2011) 561--587. 
\bibitem{Za_gen} N. Zava, {\em Generalisations of coarse spaces}, Topology Appl. 263 (2019) 230--256.
\bibitem{Za_cwp} N. Zava, {\em Cowellpoweredness and closure operators in categories of coarse spaces}, Topology Appl. 268 (2019), 106899.




}

\end{thebibliography}   



\end{document}   
