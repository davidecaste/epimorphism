\documentclass[12pt]{article}
\usepackage[english]{babel}
\usepackage{amssymb,amsthm,amscd,amsmath, txfonts,mathrsfs,enumerate}
\usepackage{xcolor}
\usepackage{tikz-cd}
\usepackage{amsmath}
\usepackage{amsfonts}
\usepackage{amssymb}
\usepackage{latexsym}
\usepackage{relsize}
\usepackage{eufrak}
\usepackage{makeidx}
%\usepackage{enumitem}




\input xy
\xyoption{all}


\usepackage{vmargin}
\setmarginsrb{28mm}{20mm}{28mm}{25mm}%
{10mm}{7mm}{10mm}{10mm}

\usepackage[all]{xy}
%\usetikzlibrary{arrows,shapes,positioning,decorations.markings, decorations.pathreplacing,decorations.pathmorphing}

\usepackage{hyperref}
%% hyperref should be the last package loaded

\usepackage[makeroom]{cancel}



\newtheorem{lemma}{Lemma}[section]
\newtheorem{proposition}[lemma]{Proposition}
\newtheorem{theorem}[lemma]{Theorem}
\newtheorem{corollary}[lemma]{Corollary}
\newtheorem{conjecture}[lemma]{Conjecture}
\newtheorem{problem}[lemma]{Problem}
\newtheorem{fact}[lemma]{Fact}
\newtheorem{claim}[lemma]{Claim}
\newtheorem{question}[lemma]{Question}
\newtheorem{fatti}[lemma]{Facts}
\newtheorem*{claim*}{Claim}

\newtheorem{introtheorem}{Theorem}
\renewcommand{\theintrotheorem}{\Alph{introtheorem}} 

\theoremstyle{definition}
\newtheorem{definition}[lemma]{Definition}
\newtheorem{remark}[lemma]{Remark}
\newtheorem{construction}[lemma]{Construction}
\newtheorem{example}[lemma]{Example}
\newtheorem{examples}[lemma]{Examples}
\newtheorem*{notaz}{Notation}


\newcommand{\bvector}[2]{\textstyle{\binom{#1}{#2}}}
\newcommand*{\norm}[1]{\rVert #1 \rVert}
\newcommand*{\ec}[2]{\mathscr{E}_{#1}(#2)}

% NOTES
\def\NB{$\clubsuit$}
\def\NBA{$\blacklozenge$}
\def\NBI{$\spadesuit$}
\def\NBD{$\maltese\ $}

% \mathbb
\def\TT{\mathsf U}
\def\SS{\mathbb S}
\def\N{\mathbb N}
\def\R{\mathbb R}
\def\Z{\mathbb Z}
\def\Q{\mathbb Q}
\def\T{\mathbb T}
\def\K{\mathbb K}
\def\A{\mathcal A}
\def\Coim{\mathrm{Coim}}
\def\coeq{\mathrm{coeq}}

% Operators
\def\Ker{\mathrm{Ker}}
\def\P{\mathcal P}
\def\Pf{\P_{fin}}
\def\Aut{\mathrm{Aut}}
\def\Mod{\mathrm{Mod}}
\def\Se{\mathfrak S}
\def\G{\mathcal{G}}

\def\hti{\widetilde{h}}


%somma e prodotto in xy
\newcommand{\Sum}{\mathlarger{\sum}}
\newcommand{\Prod}{\mathlarger{\prod}}
%\newcommand{\LM}{\mathcal L}
\newcommand{\F}{\mathcal F}
%\newcommand*{\Mon}[1]{\mathcal M_{#1}}
%\newcommand{\LCG}{\textbf{\textup{LC}}}
%\newcommand{\LC}{{}_{\K}\LCG}
%\newcommand{\LLC}{{}_{\K}\textbf{\textup{LLC}}}
%\newcommand{\LCA}{\textbf{\textup{LCA}}}
%\newcommand{\Set}{\textbf{\textup{Set}}}
%\newcommand{\CC}{\LCA_{cc}}
%\newcommand{\Vect}{{}_{\K}{\mathbf Vect}}
%\newcommand{\Flow}{\mathrm{Flow}}
\newcommand{\bcL}{\mathbf{\mathcal{L}}}
%\newcommand{\Lqm}{\bcL_{qm}}
\newcommand{\barLqm}{\overline{\bcL}_{qm}}
\newcommand{\Lpqm}{\mathcal{L}_{pqm}}
%\newcommand{\Ab}{\textbf{\textup{Ab}}}
%\newcommand{\TDLC}{\textbf{\textup{TDLC}}}
\newcommand{\Spec}{\mathrm{Spec}}
\newcommand{\re}{\mathrm{reg}}
% Functors
\newcommand{\LCO}{\mathcal{LCO}}
\newcommand{\CO}{\mathcal{CO}}
\newcommand{\clop}{closure operator}

\def\Triple{\mathbb T}

\def\Inv{\mathrm{Inv}}

%\DeclareMathOperator{\td}{t.d.}
%\DeclareMathOperator{\dist}{dist}
%\DeclareMathOperator{\cardm}{card}
\def\Im{\mathrm{Im}}
%\DeclareMathOperator{\supp}{Supp}

\newcommand*{\card}[1]{\left\vert #1 \right\vert}
%\newcommand*{\abs}[1]{\left\lvert #1 \right\rvert}
%\newcommand*{\Abs}[1]{\bigl\lvert #1 \bigr\rvert}
%\newcommand*{\set}[1]{\left\{ #1 \right\}}
%\newcommand*{\tuple}[1]{\langle #1 \rangle}

% Entropies
\newcommand{\halg}{h_{alg}}
\newcommand{\hset}{\texorpdfstring{\ensuremath{h_{set}}}{h-set}}
\newcommand{\Halg}{H_{alg}}
\def\ent{\mathrm{ent}}
\def\M{\mathcal{M}}
\newcommand\eps{\varepsilon}
% Categories
\newcommand{\LM}{\mathcal L}
\newcommand{\LCG}{\textbf{\textup{LC}}}
\newcommand{\LC}{{}_{\K}\LCG}
\newcommand{\LLC}{{}_{\K}\textbf{\textup{LLC}}}
\newcommand{\LCA}{\textbf{\textup{LCA}}}
\newcommand{\Set}{\textbf{\textup{Set}}}
\newcommand{\CC}{\LCA_{cc}}
\newcommand{\Vect}{{\K}\text{-}\textbf{\textup{Vect}}}
\newcommand{\Flow}{\mathrm{Flow}}
\newcommand{\Lqm}{\mathcal{L}_{qm}}
\newcommand{\LL}{\overline{\mathcal{L}}_{qm}}
\newcommand{\Ab}{\textbf{\textup{Ab}}}
\newcommand{\Grp}{\textbf{\textup{Grp}}}
\newcommand{\Ring}{\textbf{\textup{Ring}}}
\newcommand{\Modu}{R\text{-}\textbf{\textup{Mod}}}
\newcommand{\Mon}{\textbf{\textup{Mon}}}
\newcommand{\TFA}{\textbf{\textup{TFAb}}}
\newcommand{\Cat}{\textbf{\textup{Cat}}}
\renewcommand{\Top}{\textbf{\textup{Top}}}
\newcommand{\Haus}{\textbf{\textup{Haus}}}
\newcommand{\FHaus}{\textbf{\textup{FHaus}}}
\newcommand{\Cmon}{\textbf{\textup{Cmon}}}
\newcommand{\Semi}{\textbf{\textup{Semi}}}
\newcommand{\Csemi}{\textbf{\textup{Csemi}}}
\newcommand{\TDLC}{\textbf{\textup{TDLC}}}
\newcommand{\SL}{\mathcal{SL}}
\newcommand{\VSL}{\mathcal{VSL}}
\newcommand{\TAb}{\mathbf{TAb}}
\newcommand{\QQ}{\mathcal Q}
\newcommand{\SLatt}{\mathbf{SLatt_0}}
\newcommand{\SLattnobottom}{\mathbf{SLatt}}
\newcommand{\eim}[1]{\mathbf{EM}{(#1)}}
\newcommand{\CHaus}{\mathbf{CHaus}}
\newcommand{\Bool}{\mathbf{Bool}}
\newcommand{\CSLatt}{\mathbf{CSLatt}}
\newcommand{\MM}{\mathfrak{M}}
\newcommand{\BL}{\mathbf{BLatt}}

\newcommand{\bi}[3]{\mathsf{BiHom}(#1,#2;#3)}
% Functors
%\newcommand{\LCO}{\mathcal{LCO}}
%\newcommand{\CO}{\mathcal{CO}}

%Miscellanea
\newcommand{\argu}{\hbox to 7truept{\hrulefill}}

\DeclareMathOperator{\Ufor}{U}
\DeclareMathOperator{\Cfun}{C}
\DeclareMathOperator{\Dfun}{D}
\DeclareMathOperator{\Ifun}{I}
\DeclareMathOperator{\Ffun}{F}
\DeclareMathOperator{\eq}{eq}
\DeclareMathOperator{\Gfun}{G}
\DeclareMathOperator{\Mor}{Mor}
\DeclareMathOperator{\wid}{width}
\DeclareMathOperator{\Conv}{Conv}
\DeclareMathOperator{\Dclop}{D}
\DeclareMathOperator{\Tclop}{T}
\DeclareMathOperator{\Iclop}{I}
\DeclareMathOperator{\Cclop}{C}
%\newcommand{\Spdf}{\texorpdfstring{$S$}{S}}
%\newcommand{\Gpdf}{\texorpdfstring{$G$\xspace}{G\xspace}}
%% Pdf-LaTeX lancia un warning se trova una formula dentro il titolo di una
%% sezione. La soluzione per evitarlo e' usare il comando \texorpdfstring
%% usage: instead of writing \section{$S$-sfift}, write \section{\Spdf-shift}
%%%%%%%%%%%%%%%%%%%%%%%%%%%%%%%%%%%%%
\def\rel{\mathfrak{R}}
%% <x,y> tuple
\newcommand*{\Pa}[1]{\bigl(#1\bigr)}
\newcommand*{\PA}[1]{\left(#1\right)}
% \newcommand{\PF}[1]{[#1]^{<\omega}}
\newcommand{\et}{\ \& \ }
\newcommand{\SDiff}{\mathbin{\Delta}}

\newcommand{\rest}{\mathbin\restriction}

\newcommand*{\ns}[1]{\tensor[^*]{#1}{}}
% {{}^*{#1}}
\newcommand{\nsA}{\ns A}
%{{}^*\!A}
\newcommand{\nsV}{\ns{\mathbb V}}

%\newcommand{\av}{{\bar a}}
%\newcommand{\cv}{{\bar c}}
%\newcommand{\kv}{{\bar k}}
%\newcommand{\nv}{{\bar n}}
%\newcommand{\Aver}{\mathcal H}
%\newcommand{\Afam}{{\mathcal A}}
%\newcommand{\FolSeq}{\mathfrak F}
%\DeclareMathOperator{\td}{t.d.}
%\DeclareMathOperator{\dist}{dist}
%\DeclareMathOperator{\cardm}{card}
%\newcommand{\eqeps}{\mathrel{=_{\eps}}}
%\newcommand{\leqeps}{\mathrel{\leq_{\eps}}}
%\newcommand{\geqeps}{\mathrel{\geq_{\eps}}}
%\newcommand{\eps}{\varepsilon}
%\newcommand{\beps}{{\bar\varepsilon}}
%\newcommand{\nb}{{\bar n}}
%
%\newcommand{\ellinf}{\ell_{\infty}}
%\newcommand{\ellone}{\ell_{1}}
%\newcommand*{\pair}[1]{\langle #1 \rangle}


\renewcommand{\theequation}{\thesection.\arabic{equation}}
\numberwithin{equation}{section}

\newlength{\bibitemsep}\setlength{\bibitemsep}{.2\baselineskip plus .05\baselineskip minus .05\baselineskip}
\newlength{\bibparskip}\setlength{\bibparskip}{0pt}
\let\oldthebibliography\thebibliography
\renewcommand\thebibliography[1]{%
	\oldthebibliography{#1}%
	\setlength{\parskip}{\bibitemsep}%
	\setlength{\itemsep}{\bibparskip}%
}


\def\gen{generalized\ }
\def\nbd{neighbourhood\ }
%\def\Gen{Generalized\ }
\def\qm{quasi-metric\ }
\def\qms{quasi-metrics\ }


\DeclareFontFamily{OT1}{pzc}{}
\DeclareFontShape{OT1}{pzc}{m}{it}{<-> s * [1.200] pzcmi7t}{}
\DeclareMathAlphabet{\mathcal}{OT1}{pzc}{m}{it}
\newcommand{\vin}{\rotatebox[origin=c]{-90}{$\dashv$}}
\newcommand{\catname}[1]{\mathbf{#1}}
\newcommand{\homodd}[0]{\catname{HoMod}}
\newcommand{\until}[0]{\mathsf{until}}
\newcommand{\slice}[2]{(\catname{#1}\downarrow{#2})}
\newcommand{\reach}[1]{\mathsf{reach}_{#1}}
\newcommand{\abs}[1]{\left\lvert#1\right\rvert}
\newcommand{\nat}{\xrightarrow{\bullet}}
\newcommand{\im}[1]{\mathrm{Im}(#1)}
\newcommand{\chain}[1]{\catname{Ch}(\catname{#1})}
\newcommand{\homo}[1]{\mathcal{K}(\catname{#1})}
\newcommand{\colim}[0]{\mathrm{colim}}
\newcommand{\rela}[2]{\mathcal{Rel}_\mathcal{#2}(\catname{#1})}
\newcommand{\sub}[1]{\mathsf{Sub}_{\catname{#1}}}
\newcommand{\subm}[1]{\mathcal{M}/{{#1}}}
\newcommand{\set}[1]{\mathsf{Inj}/{\abs{#1}}}
\newcommand{\dom}{\mathrm{dom}}
\newcommand{\cod}{\mathrm{cod}}
\newcommand{\co}[1]{\{\abs{#1}\}}
\newcommand{\equ}[2]{[#1 \equiv #2]}
\newcommand{\id}[1]{\mathsf{id}_{#1}}
\newcommand{\sat}[0]{\mathsf{cl}}
\newcommand{\escape}[3]{#2 \mathfrak{E}_{#1} #3}
\newcommand{\es}[3]{ \mathsf{isER}_{#1}(p,#2, #3)}
\newcommand{\esca}[4]{ \mathsf{eRoute}_{#1} (#4, #2, #3)}
\newcommand{\cov}{\vartriangleleft}
\newcommand{\pos}[1]{\mathsf{Pos}_{\mathfrak{#1}}}
\newcommand{\rk}[2]{\mathrm{rk}_{#1}({#2})}
\newcommand{\lind}[1]{\mathcal{L}({#1})}
\newcommand{\class}[1]{\catname{Cl}({#1})}
\newcommand{\hoclass}[1]{\catname{HoCl}({#1})}
\newcommand{\propo}[0]{\mathsf{Prop}}
\newcommand{\term}[2]{\mathbf{Term}({#1}, {#2})}
\newcommand{\formu}[1]{\mathbf{Form}_{#1}}
\newcommand{\ex}[1]{\mathcal{ex(#1)}}
\newcommand{\inte}[1]{\llbracket{#1}\rrbracket}
\newcommand{\typ}[1]{\mathsf{Type}({#1})}
\newcommand{\coalg}[1]{\catname{CoAlg}(\mathcal{#1})}
\newcommand{\modd}[0]{\catname{Mod}}
\newcommand{\des}[1]{Des({#1})}
\newcommand{\prof}[2]{\mathsf{Proof}({#1},{#2})}
\newcommand{\sur}[3]{{#2}\mathfrak{S}_{#1}{#3}}
\newcommand{\er}[3]{\mathsf{EscR}_{#1}({#2},{#3})}
\newcommand{\ab}[0]{=-Adj}
\newcommand{\ba}[0]{=-E}
\newcommand{\su}[2]{\mathfrak{s}_{#1}^{#2}}
\newcommand{\ipos}[1]{\catname{Pos}(\catname{#1})}
\newcommand{\Su}[2]{\mathfrak{S}_{#1}^{#2}}
\newcommand{\Pro}[2]{\mathfrak{P}_{#1}^{#2}}
\newcommand{\surr}[0]{\mathsf{surr}}
\newcommand{\propag}[0]{\mathsf{propag}}
\newcommand{\nxt}[0]{\mathfrak{s}}
\newcommand{\integ}[1]{\mathsf{int}_{#1}}
\newcommand{\scat}[1]{\mathsf{nec}_{\gamma_{#1}}}
\newcommand{\diam}[1]{\mathsf{cl}_{\gamma_{#1}}}
\newcommand{\fix}[2]{\mathsf{Fix}_{(#1, \cdot)}({#2})}
\newcommand{\stab}[2]{\mathsf{Stab}_{(#1, \cdot)}({#2})}
\newcommand{\stabi}[2]{\mathsf{Stab}_{(#1, \cdot_{#1})}({#2})}
\newcommand{\inva}[1]{\mathsf{Inv}_{(#1, \cdot)}}
\newcommand{\invar}[1]{\mathsf{Inv}_{(#1, \cdot_{#1})}}
\newcommand{\fuz}[0]{\mathcal{FzSub}}
\newcommand{\cont}[0]{\mathsf{cont}}
\newcommand{\bound}[0]{\mathsf{ext}}


\def\E{\mathcal{E}}
\def\epi{\mathsf{epi}}
\def\reg{\mathsf{reg}}
\newcommand{\rg}[2]{\mathsf{reg}_{#1}([#2])}
\newcommand{\ep}[2]{\mathsf{repi}_{#1}([#2])}
\newcommand{\vcat}{\V\textbf{\textup{-Cat}}}


\def\A{\textbf {\textup{A}}}
\def\B{\textbf {\textup{B}}}
\def\C{\textbf {\textup{C}}}
\def\D{\textbf {\textup{D}}}
\def\X{\textbf {\textup{X}}}
\def\V{\textbf {\textup{V}}}
\def\Y{\textbf {\textup{Y}}}
\def\e{\textbf {\textup{E}}}


\title{Generalizing Mitchell's Theorem to $\vcat$}

\author{}


\begin{document}
	
	\maketitle

	\section{Preliminaries: tensor product of algebras}	
	\begin{definition}
Let $(\X, \otimes, I)$ be a symmetric monoidal category, and $(T, \eta, \mu)$ a monad on $\X$. A \emph{strength} is a natural transformation $\sigma:(-)\otimes T \to T((-) \otimes (-))$  such that the following diagrams commute

\begin{gather*}
\xymatrix@C=20pt{ I \otimes T(A) \ar[rr]^{\sigma_{I,A}} \ar[drr]&&T(I\otimes A) \ar[d]& &A\otimes B \ar[drr]_{\eta_{A\otimes B}} \ar[rr]^-{\id{A}\otimes \eta_B}& &A\otimes T(B)\ar[d]^{\sigma_{A,B}}\\
	&& T(A) && & &T(A\otimes B)}\\
\xymatrix@C=8pt{A\otimes B \otimes T(C)\ar[rr]^{\sigma_{A\otimes B, C} }\ar[dr]_{\id{A}\otimes\sigma_{B,C}} &&  T(A\otimes B \otimes C)\\
& A \otimes T(B\otimes C)\ar[ur]_{\sigma_{A,B\otimes C}}}
\end{gather*}
Let $\tau$ be the twist natural transformation, so that $\tau_{A,B}:A\otimes B\to B\otimes A$, and suppose that $\sigma$ is a strength for $T$, then $(T,\sigma)$ is \emph{commutative} if the following diagram commutes
\[\xymatrix@C=50pt{ T(T(A)\otimes B)\ar[r]^{T(\tau_{T(A),B})}& T(B\otimes T(A)) \ar[r]^{T(\sigma_{B,A})} & T(T(B\otimes A)) \ar[r]^{T(T(\tau_{B,A}))} & T(T(A\otimes B)) \ar[d]^{\mu_{A\otimes B}}\\
	T(A) \otimes T(B) \ar[u]^{\sigma_{T(A),B}}  \ar[d]_{\tau_{T(A),T(B)}}& & &T(A\otimes B)\\
	T(B)\otimes T(A) \ar[r]_{\sigma_{T(B), A}} &T(T(B)\otimes A) \ar[r]_{T(\tau_{T(B),A})}&T(A\otimes T(B)) \ar[r]_{T(\sigma_{A,B})} & T(T(A\otimes B)) \ar[u]_{\mu_{A\otimes B}}}\]	\end{definition}
	
	
	\begin{remark}
	It is well known that if $\X$ is $\Set$ then for every monad $T$ there exists a unique strength $\sigma$ on it. Specifically,
	\[\sigma_{A,B}: A\times T(B) \to T(A\times B) \qquad (a,b)\mapsto T(i_a)(b)\]
	 where 
	 \[i_a: B\to  A\times B \qquad b\mapsto (a, b)\]
	\end{remark}

\begin{remark}There is a description of the algebraic theories which induces a commutative monads: they are the ones in which every operation commutes with every other.
\end{remark}
\begin{example}
	Given the previous remark, examples of commutative monads are given by the one of $R$-modules for some commutative ring $R$ and the one associate to the actions of a commutative monoid.
\end{example}
	
	\begin{definition}
		Let $(\X, \otimes, I)$ be a symmetric monoidal category and $(T,\sigma)$ a commutative monad on $\X$, given $(X, \xi_1), (Y, \xi_2)$ and $(Z, \xi_2)$ in $\eim{T}$, an arrow $f:X\otimes Y\to Z$ is called a \emph{bihomomorphism}  if the following diagrams commute
		\[\xymatrix@C=35pt{&X\otimes T(Y)  \ar[r]^{\sigma} \ar[d]_{\id{X}\otimes \xi_2} & T(X\otimes Y)  \ar[r]^{f}& T(Z) \ar[d]^{\xi_3}\\
	&	X\otimes Y \ar[rr]_{f} & & Z\\
T(X)\otimes Y \ar[d]_{\xi_1\otimes \id{Y}}\ar[r]^{\tau_ {T(X), Y}} & Y\otimes T(X) \ar[r]^\sigma& T(Y\otimes X) \ar[r]^{T(\tau_{X,Y})} & T(X\otimes Y) \ar[r]^{T(f)} & T(Z)\ar[d]^{\xi_3}\\
X\otimes Y \ar[rrrr]_{f} & & & & Z}\]
We will denote  by $\bi{(X,\xi_1)}{(Y,\xi_2)}{(Z,\xi_3)}$ the set of bihomomorphisms $X\otimes Y\to Z$.
	\end{definition}
	
	Notice that for every $f\in \bi{(X,\xi_1)}{(Y,\xi_2)}{(Z,\xi_3)}$ and $h:(Z, \xi_3)\to (W, \xi_4)$, $h\circ f$ is again a bihomomorphism, so that we can deduce the following.
	
	\begin{proposition}
	For every pair of Eilenberg-Moore algebras $(x, \xi_1)$ and $(Y, \xi_2)$, there exists a functor $\eim{T} \to \Set$ sending $(Z, \xi_3)$ to $\bi{(X,\xi_1)}{(Y,\xi_2)}{(Z,\xi_3)}$. 
	\end{proposition}

\begin{definition}
The \emph{tensor product} $(X, \xi_1)\otimes_T(Y, \xi_2)$ of two Eilenberg-Moore algebras is a representative objcect for  $\bi{(X,\xi_1)}{(Y,\xi_2)}{-}$.
\end{definition}

\begin{remark}
	We can spell out more explicity the universal property defining $(X, \xi_1)\otimes_T(Y, \xi_2)$. It comes equipped with a bihomomorphism $t:X\otimes Y\to U_T((X, \xi_1)\otimes_T(Y, \xi_2))$ such that, for every other bihomomorphism $f:X\otimes Y \to U_T(Z, \xi_3)$ there exists a unique $\overline{f}:(X, \xi_1)\otimes_T(Y, \xi_2)\to (Z, \xi_3)$ making the following diagram commutes
	\[\xymatrix@C=50pt{X\otimes Y  \ar[r]^{f} \ar[d]_{t}&  Z\\
	U_T((X, \xi_1)\otimes_T(Y, \xi_2)) \ar@/_0.3cm/[ur]_-{U_T(\overline{f})}}\]
\end{remark}

	\begin{theorem}\label{thm:mon}
	For every commutative monad $(T, \eta, \mu)$ on $\Set$, $(-)\otimes_T(-)$ endows $\eim{T}$ of a symmetric monoidal structure, with unit $F_T(1)$. Moreover this structure is closed.
	\end{theorem}
	
	We are interested in closedness since we will exploit it to apply the following, rather technical, results. 
	
	\begin{lemma}\label{lem:pres}
		If $F:\X_1\times \X_2\to \Y$ is a functor which preserves reflexive coequalizers in each variable, then $F$ preserves reflexive coequalizers.
	\end{lemma}

Let now $F, G:\X^{op}\times \X \rightrightarrows \Y$ be two functors with $\Y$ cocomplete, then we have arrows
\[\omega^F_X:F(X,X)\to \int^{C\in \X}F(X, X) \qquad \omega^G_Y: G(Y,Y)\to \int^{Y\in \Y} G(Y, Y) \]
On the other hand, if $\Y$ has a monoidal structure $(\otimes, I)$ we also have a functor
\[F\otimes G: (\X\times \X )^{op}\times (\X\times \X) \to \Y \qquad ((X, Y), (X', Y')) \mapsto F(X, X')\otimes G(Y, Y')\]
which has its own terminal cowedge
\[\omega^{F\otimes G}_{(X,Y)}: F(X, X)\otimes G(Y, Y) \to \int^{(X,Y)\in \X\times \X} F(X,X)\otimes G(Y,Y)\]
Thus there exists a unique \emph{comparison morphism}
\[\phi:\int^{(X,Y)\in \X\times \X} F(X,X)\otimes G(Y,Y) \to \int^{C\in \X}F(X, X) \otimes \int^{Y\in \Y} G(Y, Y)\]
which makes the following diagram commutative.
\[\xymatrix{& F(X, X)\otimes G(Y,Y) \ar[dl]_{\omega^{F\otimes G}_{(X,Y)}} \ar[dr]^{\omega^F_X\otimes  \omega^G_Y}\\\int^{(X,Y)\in \X\times \X} F(X,X)\otimes G(Y,Y) \ar[rr]_{\phi} & &\int^{C\in \X}F(X, X) \otimes \int^{Y\in \Y} G(Y, Y) }\]

\begin{remark}
Given a functor $F:\X^{op}\times \X \to \Y$  and an arrow $f:X_1\to X_2$, we have arrows
\[F(f, \id{X_1}):F(X_2, X_1) \to F(X_1, X_1)  \qquad F(\id{X_2}, f ):F(X_2, X_1)\to F(X_2, X_2)\]
which induces $F^*$ and $F_*$  which, by the general theory of coends, fit into a coequalizer diagram
	\[\xymatrix{ \sum_{X_1, X_2\in \X} \sum_{\X(X_1, X_2)}  F(X_1, X_2) \ar@<.5ex>[rr]^-{F^*} \ar@<-.5ex>[rr]_-{F_*}&& \sum_{X\in \X} F(X, X) \ar[r]^{\omega^F}& \int^{X\in \X} F(X,X)  }\]
		where $\omega^F$ is induced by $\omega^F_{X}$. Notice that this coequalizer is reflexive, i.e. $F^*$ and $F_*$ have a common section: for every $X\in \X$ we can consider the composition
		\[\xymatrix{F(X,X) \ar[r]^-{i_{\id{X}}}& \sum_{\X(X, X)} F(X, X) \ar[r]^-{j_{X,X}}& \sum_{X_1, X_2\in \X} \sum_{\X(X_1, X_2)}  F(X_1, X_2)} \]
	and define $s_F$ to be the induced arrows. 
\end{remark}


\begin{proposition}\label{cor:refcoeq}
	Let $(\Y, \otimes, I)$ be a closed monoidal category with $\Y$ cocomplete. Then for every $F, G:\X^{op}\times \X \rightrightarrows \Y$, the comparison morphism previous defined is an isomorphism.
\end{proposition}
	\begin{proof}Take two coequalizer diagrams
	\[\xymatrix{ \sum_{X_1, X_2\in \X} \sum_{\X(X_1, X_2)}  F(X_1, X_2) \ar@<.5ex>[rr]^-{F^*} \ar@<-.5ex>[rr]_-{F_*}&& \sum_{X\in \X} F(X, X) \ar[r]^{\omega^F}& \int^{X\in \X} F(X,X) \\ \sum_{X_1, X_2\in \X} \sum_{\X(X_1, X_2)}  G(X_1, X_2) \ar@<.5ex>[rr]^-{G^*} \ar@<-.5ex>[rr]_-{G_*}&& \sum_{X\in \X} G(X, X) \ar[r]^{\omega^G}& \int^{X\in \X} G(X,X) }\] 
	Since they are reflexive, it follows from \ref{lem:pres}
 that $\int^{X\in \X}F(X,X) \otimes \int^{Y\in \X} G(Y,Y)$	is the coequalizer of 
\[\xymatrix{ \Sum\limits_{X_1, X_2\in \X} \hspace{3pt}\Sum\limits_{\X(X_1, X_2)}  F(X_1, X_2) \otimes \Sum\limits_{Y_1, Y_2\in \X} \hspace{3pt} \Sum\limits_{\X(Y_1, Y_2)}  G(Y_1, Y_2) \ar@<.5ex>[r]^-{F^*\otimes G^*} \ar@<-.5ex>[r]_-{F_*\otimes G_*} & \Sum\limits_{X\in \X} F(X, X) \otimes \Sum\limits_{Y\in \X} G(Y, Y) }\]
and applying the fact that $\otimes$ preserves colimits  we have a diagram

\[\xymatrix{ \Sum\limits_{X_1, X_2\in \X} \hspace{3pt}\Sum\limits_{\X(X_1, X_2)}  F(X_1, X_2) \otimes \Sum\limits_{Y_1, Y_2\in \X} \hspace{3pt} \Sum\limits_{\X(Y_1, Y_2)}  G(Y_1, Y_2)  \ar[d] \ar@<.5ex>[r]^-{F^*\otimes G^*} \ar@<-.5ex>[r]_-{F_*\otimes G_*} & \Sum\limits_{X\in \X} F(X, X) \ar[d] \otimes \Sum\limits_{Y\in \X} G(Y, Y) \\ \Sum\limits_{(X_1, Y_1), (X_2, Y_2) \in \X\times \X} \hspace{2pt}  \Sum\limits_{\X\times \X((X_1, Y_1), (X_2, Y_2))} F(X_1, X_2) \otimes G(Y_1, Y_2) \ar@<.5ex>[r]^-{(F\otimes G)^*}\ar@<-.5ex>[r]_-{(F\otimes G)_*}  & \Sum\limits_{(X, Y)\in \X\times \X} F(X,X)\otimes G(Y,Y)}\]

in which the vertical arrows are isomorphism. Since $\int^{(X,Y)\in \X\times \X} F(X,X)\otimes G(Y,Y) $ is the coequalizer of the bottom row we get the thesis.
\end{proof}
	\section{Mitchell's  Theorem}
	In the following we will assume $\V$ to be $\eim{T}$ for some commutative monad $(T, \eta, \mu)$ on $\Set$. By Theorem \ref{thm:mon} we know that it comes equipped with a symmetric monoidal closed structure $(-)\otimes_T (-):\V\times \V\to \V$. We can then consider the category $\vcat$ of category enriched over $\V$.
	
	\begin{remark}We will denote by  $(-)_0:\vcat\to \Cat$ the functor sending a $\V$-category to its underlying category. Notice that, for every pair of objects $X, Y$  in a $\V$-category $\X$
	\[\X_0(X,Y)=\V(F_T(1), \X(X,Y) )\simeq\Set(1, U_T(\X(X,Y)))\simeq U_T(\X(X,Y))\]	
	\end{remark}
	\begin{definition}
		Let $F:\X\to \Y$ be a $\V$-functor, let also $P$ and $Q$ be objects of $\Y$, we define $\e_F(X, Y) $ putting
		\[\e_F(Y_1, Y_2):=\int^{X\in \X_0} \Y(Y_1, F_0(X)) \otimes_T \Y(F_0(X), Y_2)\]
		which exists since $\V$ is cocomplete as a category monadic over $\Set$.
	\end{definition}

\begin{proposition}\label{prop:comp}
	If $F:\X \to \Y$ is a $\V$-functor which is the identity on objects, then there exists a $\V$-category $\e_F$  with the same objects of $\Y$ and in which the objects of arrows between $Y_1$ and $Y_2$ is given by $\e_F(Y_1, Y_2)$. 
 \end{proposition}
\begin{proof}
Given $Y\in \Y$ define an identity in $\e_F$ as the composition
\[\xymatrix{I \ar[r]^-{\id{Y}}& \Y(Y,Y) \otimes_T \Y(Y,Y)\ar[r]^-{\omega_Y} &\e_F(Y, Y)}\] 
Using\ref{cor:refcoeq} we can see that $\e_F(Y_1, Y_2)\otimes_T \e_F(Y_2, Y_3)$ is isomorphic to
\[\int^{X, Y\in \X_0} \Y(Y_1, X)\otimes_T \Y(X, Y_2)\otimes \Y(Y_2, Y)\otimes_T \Y(Y, Y_3)\]
and we can take as composition the arrow induced by
\[\xymatrix{\Y(Y_1, X)\otimes \Y(X, Y_2)\otimes_T \Y(Y_2, Y)\otimes \Y(Y, Y_3) \ar[r] & \Y(Y_1, Y_2)\otimes_T \Y(Y_2, Y_3)}  \]
The axioms for a $\V$-category follows from the properties of $\otimes_T$.
\end{proof}

\begin{definition}
	
	INSERIRE DEFINIZIONE DI DOMINIO QUA
\end{definition}

\begin{remark}
	Under our hypotheses on $\V$ the following hold:
	\begin{enumerate}
		\item a $\V$-functor $F:\X\to \Y$ is mono if and only if it is injective on object and mono on arrows, where this last condition means that
		\[\sum_{X, Y\in \X} \X(X,Y)\to \sum_{X, Y\in \X} \X(F(X),F(Y)) \]
		is a mono in $\V$;
		\item the family $\{\reg_{\X}\}_{\X\in \vcat}$ forms a closure operator $\reg$ on $\vcat$.
	\end{enumerate}
\end{remark}

\begin{lemma}
Qui mostrare che reg è ereditario per sottocategorie piene
\end{lemma}
\begin{proof}
	contenuto...
\end{proof}

\begin{remark}
	reg NON è ereditario in generale (altrimenti epi regolari comporrebbero)
\end{remark}

\begin{theorem}\label{thm:mitch}
	Let $F:\X\to \Y$ a $\V$-functor which is mono and the identity on objects, then $\reg_{\Y}(F)$ is the inclusion of the subcategory $\A$ with the same objects of $\Y$ and such that the inclusion $\A(Y_1, Y_2)\to \Y(Y_1, Y_2)$ makes $\A(Y_1, Y_2)$ an equalizer for
	\[\xymatrix@C=50pt{ I \otimes_T \Y(Y_1, Y_2) & \Y(Y_1, Y_1)\otimes_T \Y(Y_1, Y_2)\\ 
		\Y(Y_1, Y_2)  & \int^{\X\in \X} \Y(Y_1, F(X) ) \otimes_T \Y(F(X),  Y_2) \\
		\Y(Y_1, Y_2)\otimes_T I & \Y(Y_1, Y_2) \otimes_T \Y(Y_2, Y_2)} \]
\end{theorem}

\begin{remark}
	GLi equalizzatori si calcolano come in $\Set$
\end{remark}

\begin{proof}
Suppose that $H, G:\Y\to \e$ are such that $H\circ F=G\circ F$, then, for every $X\in \X$ we can take the composition
\[\xymatrix{\Y(Y_1, F(X))\otimes_T \Y(F(X), Y_2)\ar[r] & \e(H(Y_1), H(F(X))) \otimes \e(G(F(X)), G(Y_2)) \ar[r] & \e(H(Y_1), G(Y_2))}\]
which induces 
\[f:\int^{X\in \X_0}\Y(Y_1, F(X))\otimes_T \Y(F(X), Y_2) \to \e(H(Y_1), G(Y_2))\]
Now, if $s:Y_1\to Y_2$, taking $X_1$ and $X_2$  such that $F(X_i)=Y_i$ we get 
\[f(\id{Y_1}\otimes s)=H(s) \qquad f(s\otimes \id{Y_2})=G(s)\]
So, if $s$ belongs to $\A$ then it must belong to $\reg_{\Y}(F)$.

Viceversa, we can define $H,G:\Y\to \e_F$ which are the identity on objects putting, for every $s:Y_1\to Y_2$
\[H(s):=s\otimes \id{Y_2} \qquad G(s):=\id{Y_1}\otimes s \]
Notice that, if $s=U(t)$ for some $t:X_1\to X_2$ then $H(s)=G(s)$  since the following diagram commutes
\[\]
On the other hand, if $s$ doesn't lie in $\A$, $G(s)\neq H(s)$ and thus $s$ doesn't belong to $\reg_{\Y}(F)$.
\end{proof}

\section{Semicategories}

\end{document} 

